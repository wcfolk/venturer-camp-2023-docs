\documentclass[a4paper, 10pt]{report}

\usepackage{geometry}
\geometry{
a4paper,
total={170mm,257mm},
left=20mm,
top=20mm,
}
\setlength\parindent{0pt} % get rid of the stupid indent

% INCLUDE PACKAGES

\usepackage[utf8]{inputenc}
\usepackage[dvipsnames]{xcolor}
\usepackage{float}
\usepackage{graphicx}
\usepackage{tabularx}
\usepackage{fontawesome}
\usepackage[colorlinks=true, linkcolor=Magenta, breaklinks=true]{hyperref}
\usepackage{ragged2e}
\usepackage{cancel} % used to cancel out numbers in maths mode.
\usepackage{amssymb} % gives more maths symbols
\usepackage{enumitem} % gives ability to have different enumerate bullets.
\usepackage{multirow}

\usepackage{longtable}

\usepackage{fontspec}
\setmainfont[Ligatures=TeX]{Montserrat}
\setsansfont[Ligatures=TeX]{Poppins Bold}
\usepackage[raggedright,bf,sf]{titlesec}

\setcounter{tocdepth}{0}
\renewcommand{\chaptername}{Section}

\usepackage{fancyhdr}
\pagestyle{fancy}
\fancyhf{}
\fancyhead[R]{\leftmark}
\fancyfoot[C]{\thepage}
\renewcommand{\footrulewidth}{0.4pt}
\addtolength{\topmargin}{-1.59999pt} % This and line below changes margins to accomondate for headers and footers
\setlength{\headheight}{13.59999pt}

% NOW CREATE TITLE PAGE

\newcommand{\bookTitle}[4]{
    \begin{titlepage} % Suppresses headers and footers on the title page
	
        \centering % Centre everything on the title page
        
        \rule{\textwidth}{0pt} % Thick horizontal rule
        
        \vspace{0.3\textheight} % Whitespace between the top rules and title
        
        % \rule{\textwidth}{1pt}\\
        \vspace{15pt}
        {\LARGE \textbf{VENTURER CAMP 2023}}\\[1em] % Title line 1
        %{\Large - }\\[0.5\baselineskip] % Title line 2
        {\LARGE #1} \\[15pt]% Title line 3
    
        {\large #2 \\[5em]}
    
        {\large #3} % Date
        \vspace{15pt}
        % \rule{\textwidth}{1pt}
        \vspace{3em}
        %\vfill % Whitespace between the author name and publisher
        
        
        \begin{figure}[H]
            \begin{minipage}{0.45\textwidth}
                \centering
                \includegraphics[width=0.45\textwidth]{../vc23.png} %this will be replaced with vc23 logo when we have one
            \end{minipage}\hfill
            \begin{minipage}{0.45\textwidth}
                \centering
                \includegraphics[width=0.45\textwidth]{../wcf.png}
            \end{minipage}\hfill
        \end{figure}
    
        \vfill  
        
    \end{titlepage}

    \fancyhead[L]{#1}
    \fancyfoot[L]{\footnotesize{\texttt{#4}}}
}

\newcommand{\backPage}{
    \begin{titlepage}
        \centering
        \rule{\textwidth}{0pt}
        \vfill
        \begin{figure}[H]
            \begin{minipage}{0.45\textwidth}
                \begin{flushright}
                \includegraphics[width=0.1\textwidth]{../vc23.png} %this will be replaced with vc23 logo when we have one
                \end{flushright}
            \end{minipage}\hfill
            \begin{minipage}{0.45\textwidth}
                \includegraphics[width=0.1\textwidth]{../wcf.png}
            \end{minipage}\hfill
        \end{figure}
        \rule{\textwidth}{2pt}
        \small{Venturer Camp 2023, a project by \href{https://woodcraft.org.uk}{Woodcraft Folk}, will bring together 13-17 year olds from across the UK to camp together and live by the Woodcraft Folk values for a week in the summer of 2023.\\
        Check out our website (\href{https://venturercamp.org.uk}{venturercamp.org.uk}) and our social media pages for more information.}
    \end{titlepage}
}

% include header footer on new chapter pages by redefining \chapter command
\makeatletter
\renewcommand\chapter{\if@openright\cleardoublepage\else\clearpage\fi
\thispagestyle{fancy}%
\global\@topnum\z@
\@afterindentfalse
\secdef\@chapter\@schapter}
\makeatother

% \usepackage{draftwatermark}
% \SetWatermarkText{\textsc{proof}}
% \SetWatermarkScale{1}


\begin{document}
\bookTitle{Info Pack v0}{\textit{An introduction to all things Venturer Camp 2023}}{16th December 2022}{info-pack-v0}
\tableofcontents

\chapter{Welcome From Thomas}
Welcome to the Info Pack V0!

We're putting together this info pack to bring you all the information which you might need at this stage of planning for Venturer Camp 2023 within your district.
We'll try to keep this as short as possible, cramming as much into it as we can.

If you want to get in touch with me for any reason, feel free to drop me an email to \href{mailto:info@venturercamp.orf.uk}{\texttt{info@venturercamp.org.uk}}. 

\section{Find us on the Internet}
Venturer Camp lives across the internet!

Check out our website \href{https://venturercamp.org.uk}{venturercamp.org.uk}, our Instagram \href{https://www.instagram.com/venturercamp/}{@venturercamp} and our Facebook \href{https://facebook.com/wcfventurercamp}{/wcfventurercamp}.

We're always listening to feedback and trying to improve. One piece of feedback we've heard from Common Ground is that there were too many different places people could find information. We'll be publishing all important information to our website (and social media), minimising important information which is just published to social media. When we release information packs, like this one, we'll email that to all Venturer Leaders and once bookings open, all booking contacts too!


\chapter{Meet The Team}
\section{Thomas Boxall - Camp Coordinator}
Hey there, I'm Thomas and I'm the coordinator for Venturer Camp 2023! Outside of Venturer Camp, I'm on General Council and was recently on DF Committee. I'm really excited to be coordinating the next Venturer Camp and can't wait for you to see what we've got planned!
\section{Millie Burgh - Woodcraft Folk Events Assistant}
Hiya, I'm Millie! Lots of you will have met me over the last few years as I co-coordinated Venturer Camp 2019 and worked on Common Ground as the camp assistant. I'm looking forward to using everything I've learnt from these two events to support another volunteer team on another big Woodcraft Folk camp!
\section{Wilf and Sadie Lamont and Emma Britton-Voss - Central KP Team}
We are the central KP team, responsible for designing the menu and organising all the food for Venturer Camp 2023. We are a group of three DFs, including a sibling duo.

Hi, I'm Emma. I helped with central food at Common Ground and was co-KP for DF Camp 2022. I am also on the DF committee and General Council.

Hi, I'm Wilf. I also worked on the central food team at Common Ground. You may have met me on the Standing Orders Committee.

Hi, I'm Sadie. I have attended many Woodcraft Folk camps and events, and I'm excited to be working with Emma and my twin brother, Wilf.

We are looking forward to meeting everyone and providing the food for Venturer Camp 2023.
\section{Joe and Chris Bowler - Coordinators of the Café and allergy kitchen }
Hey we're the Bowler brothers, lifelong woodcraft busybodies. We're testing out a lot of new ideas to improve food accessibility and easing the burden on village KPs. We're excited to see you all again, so come say hi at the cafe if you want to talk over a nice coffee.
\section{Sapna Agarwal and Mollie Saunders - Volunteer Support}
Hi there, we're the PEB team - here to support volunteers at Venturer camp. Our point of contact, Sapna, will be helping those coming to camp to prepare, and Mollie will be available to assist on camp. (For anyone who came to Common Ground, PEB is back!)

The PEB, or Positive Energy Bubble, is a new initiative focused on creating a supportive environment for volunteers. The team will provide a space where volunteers can relax, recharge, and connect with others. We will also offer refreshments and potentially even massages or aromatherapy.

In the past, there have been instances of volunteer burnout at camp, and we want to prevent this from happening at Venturer camp. We encourage open communication and offloading, so feel free to vent about even the smallest issues.

As a brand-new team, we are flexible and willing to adapt to the needs of volunteers. We invite you to join us and help shape the PEB's Woodcraft legacy.

\section{Bx Muller, Will Tuffrey and Imke Hoffmann - Site Services, Logistics \& Production}
Hey, we're Bx, Will and Imke.

We'll be forming the site services team. We will be encouraging the self sufficiency of the Venturers and will get them involved in some of our day-to-day tasks through a rota (including cleaning toilets and showers).

\chapter{Bookings Information}
We will be using a modified version of the Common Ground booking system for this camp.

In January, we will publish template forms which groups can use to gather information on their participants coming which can then be inputted into the booking system.
\section{Bookings Timeline}
\begin{table}[H]
    \centering
\begin{tabularx}{\textwidth}{ll}
    \textbf{27th January 2023} & Bookings Open \\
    \textbf{12th April 2023} & Early Bird Booking Deadline \\
    \textbf{26th May 2023} & Final Booking Deadline
\end{tabularx}
\end{table}

\section{Who Can Come?}
Taking inspiration from the DFs, we are introducing a \textit{Covid Clause} for Venturer Camp 2023! This will enable those aged between 13 and 17 inclusive to come as a participant.

Even if you're not part of a Venturer Group, your still able to come. More information this will be released closer to when bookings open. 

Anyone over the age of 18 must take on a volunteering role, whether this be in a village or centrally for the camp.

\section{Early Bird Bookings}
We will be doing Early Bird Bookings slightly differently for Venturer Camp 2023.

Rather than a reduction in price, those who book \& pay before the Early Bird Booking Deadline will receive a limited edition t-shirt. To claim this t-shirt, you will still have to be booked in after the final booking deadline. We'll be in touch with contacts for those bookings to collect t-shirt sizes. 

We will be giving the t-shirts out at pre-camp!
\section{Individual Bookings}
To simplify the bookings process, we will request individual adults who are booking themselves in, do so through the lone adult booking secretary. This is to reduce the administrative burden of chasing people to find out where they want to camp when we get closer to camp.

Information on who the lone adult booking secretary is and how they can be contacted will be made available when bookings open.

If you are a lone adult who wishes to camp with a particular group, please contact that group to book through their booking secretary. If you are unsure of who this, get in touch with the Info team via \href{mailto:info@venturercamp.org.uk}{\texttt{info@venturercamp.org.uk}} and we can put you in contact.

\chapter{Finance}
This section will outline the cost of camp as well as a number of other financial related points.
\section{Cost Of Camp}
Venturer Camp 2023 will cost £150 for the entire camp for under 18s and £50 for the entire camp for over 18s. This price difference is to recognise the significant contribution made by our volunteers, without whom the event would not be possible. Under 18s who are volunteering with significant roles are also able to get the reduced fee. An access fund will be launched to reduce any financial barriers young camp participants might face. Please contact \href{mailto:info@venturercamp.org.uk}{\texttt{info@venturercamp.org.uk}} for more information on this.

For those wishing to come for shorter spans of time, this is broken down as follows:
\begin{table}[H]
    \centering
\begin{tabularx}{\textwidth}{ll}
    \textbf{Under 18s} & £21.50 per night \\
    \textbf{Over 18s} & £7.50 per night 
\end{tabularx}
\end{table}
Like Venturer camp in 2019, you will be able to select the nights which you wish to be on site for while booking. This is completely flexible and you will be able to select any combination of nights.

\section{Fundraising}
The following table has some ideas of grants to apply to/ways to fundraise. The links will take you to pages on the Common Ground website which include more detailed guidance on each idea. Additional advice and support is available from \href{mailto:fundraising@woodcraft.org.uk}{\texttt{fundraising@woodcraft.org.uk}}. 


\begin{longtable}{|p{7em}|p{5em}|p{5em}|p{5em}|p{10em}|}
\hline
\textbf{Funding Source} & \textbf{Difficulty} & \textbf{Likelyhood of Success} & \textbf{Amount you could raise} & \textbf{Conditions}\\
\hline
\hline
\href{https://www.commonground.camp/fundraising/local-giving/}{Local Giving Magic Little Grants} & Easy & Medium & £500 & The application process could take as little as 10 minutes\\
\hline
\href{https://www.commonground.camp/fundraising/gift-aid/}{Gift Aid} & Medium & Guaranteed & 25\% of event fee (donation) & Parents paying for subs or camps must pay some income tax or capital gains tax\\
\hline
\href{https://www.commonground.camp/fundraising/awards-for-all/}{National Lottery Awards For All} & Medium & Medium & Up to £10,000 & Post-pandemic, the number of applications is quite high. Applications are accepted all year round, with turnaround time around 12 weeks\\
\hline
Donation-based crowdfunding & Easy & High & Varies & Crowdfunding is a way of raising donations by asking a large number of people each for a small amount of money. You have several platforms to choose from. Consider pages such as GoFundMe, Justgiving or Crowdfunder. Check carefully that the page fee is 0\%\\
\hline
\href{https://lionsclubs.co/Public/}{Lions Club} or \href{https://www.rotary.org/en/}{Rotary Club} & Easy & Medium & Varies & Your local philanthropic organisations may donate towards camp costs. Call your local club to enquire. It's good for raising our profile too! \\
\hline
\end{longtable}

\section{Travel Access Fund}
Bibilins is rurally located, despite being in the midlands! As a result of this there will be a vast difference in the amount which campers pay to get to the site. To help campers cover the cost of travel, we will be introducing a Travel Access Fund. In February we will publish guidance on how the fund will work, but in essence those with the shortest journeys will be asked to make a solidarity contribution to support the costs of those campers travelling the furthest.
\section{Additional Costs}
When advertising the cost of Venturer Camp to young people, districts should take into account additional costs which they may incur in the process of getting to Venturer Camp. These costs might include (however are not limited to):
\begin{itemize}
    \item Resources for village-level activities
    \item Subsidies for travel to \& from the site
    \item Replacing or purchasing village camp equipment (e.g. first aid kit, lighting)
    \item Snacks whilst travelling to \& from the site
\end{itemize}
\section{Paying For Things On Site}
We will have a cafe which will sell food and drinks (including proper coffee!) as well as a merchandise stand which will sell Limited Edition Venturer Camp 2023 merch. More information on what will be sold and the approximate costs of items will be released closer to the time. 

Whilst on site, we will request card payments where possible. This is to reduce the administration required for cash handling and banking of cash every day. We will, of course, take cash as we understand that not all Venturers will have access to bank cards.

\chapter{Site \& Infrastructure}
We are excited to announce that we will be returning to Biblins for Venturer Camp 2023. Biblins is a Woodcraft Folk run campsite in the Wye Valley, Forest of Dean. 

The team at Biblins are working hard on improvements to the site, installing a solar array and moving Camp Koodoo to the west side, freeing up the middle pitches to become our central area. 
\section{Contact On Site}
Biblins valley is a dead-zone of mobile signal.

On site there is a landline telephone which will be staffed for the duration of the camp as well as a camp WhatsApp number.

There will be limited charging facilities and young people should be prepared to spend the majority of the week without a charged phone, unless they require it for medical reasons.
\section{Shuttle Busses}
On the first and last day of camp, we will be running shuttle buses to and from Hereford train station. Once camp bookings have opened, we will open shuttle bus bookings. Keep an eye out for more information coming soon.
\section{Village Infrastructure}
We are really pleased that a number of local districts are happy for us to use their camping equipment to fully equip each village with a kitchen and a dining/activities marquee! The core team are in contact with these districts to coordinate transport of equipment and exactly what we can borrow. We have also put money aside in the budget to support the districts with paying for van hire to transport the equipment. If your district hasn't been contacted by the team and you would like to discuss your district's equipment being used at Venturer Camp 2023, please email \href{mailto:info@venturercamp.org.uk}{\texttt{info@venturercamp.org.uk}}.


\chapter{How You Can Help}
\section{Get Involved}
Venturer Camp 2023 is run almost entirely by volunteers. We wouldn't be able to put it together without their support.

We are always on the lookout for more volunteers to make  the camp awesome! If you want to get involved, drop \href{mailto:info@venturercamp.org.uk}{\texttt{info@venturercamp.org.uk}} an email.

We are currently looking for volunteers with an interest in
\begin{itemize}
    \item Communications
    \item Finance
    \item Bookings Administration
    \item Running Centrers
    \item Accessibility On Camp
    \item Atmosphere Coordinator
    \item Site Services
\end{itemize}
More information on roles can be found via our website \href{https://venturercamp.org.uk/get-involved/}{\texttt{venturercamp.org.uk/get-involved/}}. 

If there's not a role listed which suits you and you want to get involved, please do drop us an email and we'll find a role for you! 

\section{Support for Volunteers}
The Positive Energy Bubble, PEB, is back for Venturer Camp 2023! They are here to support volunteers so the volunteers feel supported in bringing young people to Venturer Camp 2023. PEB will offer suggestions over the months leading up to the event on how Volunteers can reduce the burden on themselves and at Venturer Camp, provide a space where volunteers can relax and destress. For more information on their work, contact \href{mailto:volunteer.support@venturercamp.org.uk}{\texttt{volunteer.support@venturercamp.org.uk}}.

\chapter{Food}
The menu will be organised by our central KP team, ingredients delivered to your village daily. Contact \href{mailto:food@venturercamp.org.uk}{\texttt{food@venturercamp.org.uk}} if you want to be a village KP.
\section{Cafe \& Allergy Team}
The KP team has an additional division for Venturer Camp 2023, the Allergy kitchen team can provide partial or complete replacements of the main camp menu to meet individual dietary needs. Contact \href{mailto:cafe.allergy@venturercamp.org.uk}{\texttt{cafe.allergy@venturercamp.org.uk}} if, due to complex dietary needs, you will need an altered menu for camp. 

\chapter{Accessibility}
We are really keen to make Venturer Camp 2023 as accessible to as many people as possible. Bunkbed accommodation is available in The Burrow. Disabled access wet rooms are available on site.

To help us with this, we need to hear from you. Is there something we can do to make the camp more accessible for you? Get in touch to \href{mailto:info@venturercamp.org.uk}{\texttt{info@venturercamp.org.uk}}.

\chapter{Safeguarding \& DBS}
All campers aged 18 years and over will need to follow Woodcraft Folk's \href{https://woodcraft.org.uk/resources/volunteer-screening/}{Screening \& Vetting procedures}, this includes:
\begin{itemize}
    \item Submitting two suitability references
    \item Completing an enhanced DBS
    \item PVG membership (in Scotland only)
\end{itemize}
All volunteers are advised to check their DBS status before April 2023.

The Safeguarding team will support camp organisers with a risk assessment, which will be shared with all Village Co-ordinators after pre-camp.

If you would like to discuss any safeguarding or child protection issues please contact\\ \href{mailto:safeguarding@woodcraft.org.uk}{\texttt{safeguarding@woodcraft.org.uk}}

\chapter{Programme}
We are taking feedback from Common Ground and running the mornings as Village Mornings (where little-to-no central program is run) and the afternoons as Central Afternoons (where centres will run workshops and sessions). Centres may create activity packs for villages that lead into future workshops. The cafe and chill spaces will be open in the morning as well as Safe Spaces and Volunteer Support spaces. 

Following feedback from Venturer Camp 2019 we also plan to offer activities such as climbing and canoeing, fully benefiting from our location. 

If you want to get involved with a centre, either to coordinate a centre or run workshops in one, drop an email to \href{mailto:programme@venturercamp.org.uk}{\texttt{programme@venturercamp.org.uk}}.
\section{Theme}
As you may have seen on social media and our website, we have chosen the theme Mythology for Venturer Camp 2023! We are really excited to see what programme comes out with that theme and what villages will do around that theme.

\chapter{Sustainability \& Environment}
In 2019, Woodcraft Folk declared a climate emergency. As an organisation, we are committed to reducing its greenhouse gas emissions. Venturer Camp 2023 is a great opportunity to educate young people on sustainable values.

The organising team will be taking actions in a number of ways to minimise our carbon footprint across all areas of activity before, during and after the event. However, this is not just the responsibility of the organising team, everyone who comes to Venturer Camp and is a part of it is responsible for reducing their individual carbon footprint. More resources to support this will be made available in the new year.

The Sustainability and Environment team will be working with volunteers to ensure the camp is enjoyable for all whilst keeping the environmental impact as low as we can. 

The Programme team will run a series of Sustainability workshops throughout the camp where young people will be able to engage at a practical level about what Sustainability is and how best it can be incorporated into everyday life.

Biblins Campsite, is not on the mains gas or electricity grid therefore all our power will be provided from renewable sources, such as the new solar array being installed this spring or small solar systems brought by districts. We are currently exploring ways to minimise our bottled gas usage.

We encourage you in your group settings to discuss the option of our Veggie or Vegan menu, thinking about the environmental impact of these versus a meat-containing menu. We will be using local suppliers to source fresh produce where possible as well as looking for ways in which we can reduce food waste.

Furthermore, we will be exploring ways in which groups can travel to Biblins, reducing their carbon footprint. More information on this will be made available in the new year.

Our decarbonisation work will be led by the Sustainability and Environment team. They can be contacted via \href{mailto:info@venturercamp.org.uk}{\texttt{info@venturercamp.org.uk}}

\chapter{Pre-Camp}
The Pre-Camp for Venturer Camp 2023 will mostly be held virtually. We will also be hosting a weekend on site from the 30th June to 2nd July 2023 as a Board Meeting where we will invite leaders and 1 or 2 Venturers from each group booked in to come and meet the other groups in their village.

More information on Pre-Camp, both the on site Board Meeting and virtual sessions, will be made available on our website in Spring next year. 

\backPage
\end{document}
