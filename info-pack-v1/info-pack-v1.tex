\documentclass[a4paper, 11pt]{report}

\usepackage{geometry}
\geometry{
a4paper,
total={170mm,257mm},
left=20mm,
top=20mm,
}
\setlength\parindent{0pt} % get rid of the stupid indent

% INCLUDE PACKAGES

\usepackage[utf8]{inputenc}
\usepackage[dvipsnames]{xcolor}
\usepackage{float}
\usepackage{graphicx}
\usepackage{tabularx}
\usepackage{fontawesome}
\usepackage[colorlinks=true, linkcolor=Magenta, breaklinks=true]{hyperref}
\usepackage{ragged2e}
\usepackage{cancel} % used to cancel out numbers in maths mode.
\usepackage{amssymb} % gives more maths symbols
\usepackage{enumitem} % gives ability to have different enumerate bullets.
\usepackage{multirow}

\usepackage{longtable}

\usepackage{fontspec}
\setmainfont[Ligatures=TeX]{Montserrat}
\setsansfont[Ligatures=TeX]{Poppins Bold}
\usepackage[raggedright,bf,sf]{titlesec}

\setcounter{tocdepth}{0}
\renewcommand{\chaptername}{Section}

\usepackage{fancyhdr}
\pagestyle{fancy}
\fancyhf{}
\fancyhead[R]{\leftmark}
\fancyfoot[C]{\thepage}
\renewcommand{\footrulewidth}{0.4pt}
\addtolength{\topmargin}{-1.59999pt} % This and line below changes margins to accomondate for headers and footers
\setlength{\headheight}{13.59999pt}

% NOW CREATE TITLE PAGE

\newcommand{\bookTitle}[4]{
    \begin{titlepage} % Suppresses headers and footers on the title page
	
        \centering % Centre everything on the title page
        
        \rule{\textwidth}{0pt} % Thick horizontal rule
        
        \vspace{0.3\textheight} % Whitespace between the top rules and title
        
        % \rule{\textwidth}{1pt}\\
        \vspace{15pt}
        {\LARGE \textbf{VENTURER CAMP 2023}}\\[1em] % Title line 1
        %{\Large - }\\[0.5\baselineskip] % Title line 2
        {\LARGE #1} \\[15pt]% Title line 3
    
        {\large #2 \\[5em]}
    
        {\large #3} % Date
        \vspace{15pt}
        % \rule{\textwidth}{1pt}
        \vspace{3em}
        %\vfill % Whitespace between the author name and publisher
        
        
        \begin{figure}[H]
            \begin{minipage}{0.45\textwidth}
                \centering
                \includegraphics[width=0.45\textwidth]{../vc23.png} %this will be replaced with vc23 logo when we have one
            \end{minipage}\hfill
            \begin{minipage}{0.45\textwidth}
                \centering
                \includegraphics[width=0.45\textwidth]{../wcf.png}
            \end{minipage}\hfill
        \end{figure}
    
        \vfill  
        
    \end{titlepage}

    \fancyhead[L]{#1}
    \fancyfoot[L]{\footnotesize{\texttt{#4}}}
}

\newcommand{\backPage}{
    \begin{titlepage}
        \centering
        \rule{\textwidth}{0pt}
        \vfill
        \begin{figure}[H]
            \begin{minipage}{0.45\textwidth}
                \begin{flushright}
                \includegraphics[width=0.1\textwidth]{../vc23.png} %this will be replaced with vc23 logo when we have one
                \end{flushright}
            \end{minipage}\hfill
            \begin{minipage}{0.45\textwidth}
                \includegraphics[width=0.1\textwidth]{../wcf.png}
            \end{minipage}\hfill
        \end{figure}
        \rule{\textwidth}{2pt}
        \small{Venturer Camp 2023, a project by \href{https://woodcraft.org.uk}{Woodcraft Folk}, will bring together 13-17 year olds from across the UK to camp together and live by the Woodcraft Folk values for a week in the summer of 2023.\\
        Check out our website (\href{https://venturercamp.org.uk}{venturercamp.org.uk}) and our social media pages for more information.}
    \end{titlepage}
}

% include header footer on new chapter pages by redefining \chapter command
\makeatletter
\renewcommand\chapter{\if@openright\cleardoublepage\else\clearpage\fi
\thispagestyle{fancy}%
\global\@topnum\z@
\@afterindentfalse
\secdef\@chapter\@schapter}
\makeatother

\newcommand{\nl}{\newline}
\usepackage{ragged2e}

\usepackage{draftwatermark}
\SetWatermarkText{\textbf{PROOF}}
\SetWatermarkScale{1}

% define wcf colours
\definecolor{wcfGreen}{RGB}{112, 143, 65}
\definecolor{wcfRed}{RGB}{197, 20, 21}
\definecolor{wcfYellow}{RGB}{252, 215, 96}
\definecolor{wcfDarkGreen}{RGB}{24, 90, 71}

\begin{document}
\bookTitle{Info Pack v1}{Subtitle}{5th May 2023}{info-pack-v1}
\tableofcontents
\chapter{Introduction}
Welcome to info pack v1!\nl

This info pack comes to you a bit later than originally anticipated, but fear not as it is crammed full of useful information about Venturer Camp!\nl

As of writing this section, we're sitting at a whopping 265 bookings! If you haven't booked yet and you would like to, be quick as bookings close on 26th May 2023.\\
Camp costs £150 for those aged under 18 and £50 for those aged 18 and up.\nl

If you've got a question or would like to know some more information, feel free to drop us a DM on our social media or send us an email (\href{mailto:info@venturercamp.org.uk}{\texttt{info@venturercamp.org.uk}}).

\section{Venturer Camp 2023, a reminder}
Back for 2023, Venturer Camp will once again be returning to Woodcraft Folk's Biblins Youth Campsite in Wye Valley near Hereford from the 5th to 12th August 2023. Venturer Camp is open to participants aged 13-17 due to the Covid pandemic and will have the theme of Mythology. 

\section{Find Out More}
Venturer Camp lives across the internet. You can find out more about us on our website (\href{https://venturercamp.org.uk}{\texttt{venturercamp.org.uk}}), on the Woodcraft website (\href{https://woodcraft.org.uk}{\texttt{woodcraft.org.uk}}), on our Instagram (\href{https://www.instagram.com/venturercamp/}{\texttt{@venturercamp}}) and on our Facebook (\href{https://www.facebook.com/wcfventurercamp}{\texttt{/wcfventurercamp}}).
We will also send emails to those who have booked and to Venturer group contacts with key information in it.

\section{Future Publications}
We are aiming to publish Info Pack v2 by 15th June 2023. This will contain most of the information which you will need before you come to camp including the finalised kit list; finalised travel information; expectations of villages including central camp clan jobs and village officeholder roles; and lots more!\\
We will be publishing the preliminary village handbook by 16th July. This will be provided in a printed form to villages on site. 

\chapter{Pre-Camp}
Venturer Camp 2023 will be hosting the majority of its pre-camp sessions online to save money, reduce the environmental impact and make them more accessible!\nl

We will also be hosting an on-site weekend in early July to finalise logistics.

\section{Virtual Pre-Camp}
Virtual sessions will be held each week from 17th April to the end of June through Zoom. Each session has a different topic and recordings of the sessions along with any presentations used will be made available on our website.\nl 

Upcoming sessions can be found on the \href{https://venturercamp.org.uk/calendar/}{calendar page of our website}.

\section{On-Site Pre-Camp}
On-Site Pre-Camp will take place from the 30th June - 2nd July at Biblins Youth Campsite. \nl

The weekend will not be a taster of the camp, it will be an opportunity for up-to one leader and one venturer (or one person from a non-venturer containing booking) to visit the site, review logistics and meet with other leaders who will be in your village!\nl

Information on booking for on-site pre-camp will be sent to booking contacts by mid-May. 

\chapter{Travelling To Site}
The full address of the site is as follows:\\
Biblins Youth Campsite\\
The Doward\\
Whitchurch\\
Ross-on-Wye\\
HR9 6DX\\
OS grid reference: SO 549 145

\section{Shuttle Buses}
We will have coaches from Hereford station to the camp on Saturday 5th August and from camp to Hereford station on Saturday 12th August. Unfortunately we won't be able to provide any transport on the days in between.\nl

Organising the coaches is always a little chicken and egg, as we know groups want to organise trains to fit with the coaches but we also want to organise the coaches to both fit with train arrival times and ensure everyone who needs one has a seat. We know no one will be booking trains for a couple of months yet but it would be incredibly useful if groups could let us know if they are considering travelling by train so we can start to have a vague idea of who might need spaces on the coach. \nl

Please take a moment to email \href{mailto:coaches@venturercamp.org.uk}{\texttt{coaches@venturercamp.org.uk}} if you know you are, are likely to, or might be travelling by train on either Saturday and if you have it the rough number of people travelling and rough arrival time (this could be eg some time between 1 and 4 pm). We really don't mind very vague information, some knowledge will help us with planning.

If you have any questions about coaches please also do drop Jeni and David an email at \href{mailto:coaches@venturercamp.org.uk}{\texttt{coaches@venturercamp.org.uk}}

\section{Site Access}
The main site access track is unsuitable for coaches, vehicles weighing over 7.5 tonnes or vehicles over 6ft wide. There is a coach access route on the Gloucestershire side of the river which is graded and rated for 40 ton timber lorries. This can be used for coaches. Please note that all luggage/ equipment will need to be carried across the suspension bridge.\nl

If you require use of the coach access route, please contact the Coordination \& Event Admin team via \href{mailto:info@venturercamp.org.uk}{\texttt{info@venturercamp.org.uk}} as we will need to ensure the access gate is unlocked for you and co-ordinate the arrival of vehicles to ensure no traffic jams!

\chapter{Communications During The Event}
The Venturer Camp 2023 coordination team recognises that for many young people being away from home can be a nerve wracking experience both for the young people and for the parents/ carers of those young people.\nl

The site has extremely poor mobile network coverage, it should be expected that no mobile service is available on site. There is limited service a short walk away from the site, we anticipate that young people will walk to this location as part of their morning activities a few times throughout the event.\nl

In the case of a serious emergency, the Coordination \& Events Admin team can pass messages to participants \& volunteers at the camp. They will be contactable via the following methods:
\begin{table}[H]
    \centering
    \begin{tabular}{p{0.3\textwidth} p{0.6\textwidth}}
        \hline
        Email & info@venturercamp.org.uk\\
        \hline
        WhatsApp & \textit{number to be announced in next info pack}\\
        \hline
        Phone & 01600 890 850\\
        \hline
    \end{tabular}
    \caption{Contact methods during the event}
\end{table}

\section{WiFi Access}
There is extremely limited WiFi available at the site. This will only be available to those who present a legitimate requirement for needing access to it. Obtaining exam results is a legitimate reason - so all of those who are expecting their Scottish Higher results on August 9th will be given access on that day.

\section{Electricity Access}
There is no access to electricity for the general population at camp. Participants and Volunteers should ensure they bring enough portable charging capabilities to last the duration of the event.

\chapter{Equipment}
\section{Village Kit}
A number of local districts/ districts camping at Biblins before or after Venturer Camp's kit will be used to supply the four villages with kit.\nl

The central camp will contribute to the cost of hiring a van to move their kit so there is no expectation for groups within the village to contribute to this.\nl

The Coordination \& Event Administration team will be in contact with the kit providing districts to coordinate transportation/ storage on site.

\section{Sleeping Tents}
It is expected that all participants \& volunteers provide their own sleeping tents. It may be the case that participant/ volunteer's district will provide enough tents for all to sleep in, or that individuals are expected to bring their own.\nl

If bringing sleeping tents for yourself/ your delegation to camp is a problem, please get in touch with the Coordination \& Event Administration team via \href{mailto:info@venturercamp.org.uk}{\texttt{info@venturercamp.org.uk}}.

\section{Kit List}
Anything brought to camp is brought at the owners risk. We recommend not bringing valuables or keeping them on your person at all times.
\subsection{It is suggested that everyone on site brings}
\begin{itemize}
    \item Sleeping bag
    \item Pillow, roll-mats etc
    \item Eating Kit
    \item Wash Kit
    \item Clothes (be prepared for all weathers)
    \item Torch
    \item Waterproof jacket \& trousers (if you have them)
    \item Boots/ outdoor shoes which are comfortable to hike in
    \item Flip-Flops/ sandals
    \item Towel
    \item Something to put dirty laundry in
    \item A refillable water bottle
    \item Sun Cream
    \item Hat
\end{itemize}
\subsection{Participants wishing to take part in the Adventurous Activities will also need}
\begin{itemize}
    \item Clothes you don't mind getting wet \& dirty in the river
    \item Spare bin bag for wet clothes
    \item Towel you don't mind using straight out of the river
    \item Snug fitting trainers
\end{itemize}
\subsection{Do not bring}
Anyone found on site with the items listed below will have the item confiscated
\begin{itemize}
    \item Nuts or products containing nuts
    \item Hi-vis jackets
    \item Walkie Talkies/ handheld radio communication devices
    \item National Flags
    \item Hair Straighteners/ Dryers
    \item Unnecessary Electrical items
    \item Items of high sentimental/ monetary value.
\end{itemize}

\chapter{Food}
\section{Draft Menu}
\begin{table}[H]
    \centering
    {\RaggedRight
    \begin{tabular}{p{0.2\textwidth} p{0.2\textwidth} p{0.2\textwidth} p{0.2\textwidth}}
        \textbf{Day} & \textbf{Breakfast} & \textbf{Lunch} & \textbf{Dinner}\\
        \hline
        Saturday 5th & \cellcolor{gray!25} & \cellcolor{gray!25} & Pesto Pasta with halloumi and sweetcorn\\
        \hline
        Sunday 6th & Eggy Bread & Leek and potato soup with rolls and cheese & `Korma' curry and rice \\
        \hline
        Monday 7th & Porridge & Sandwiches with egg mayo and coleslaw & \cellcolor{red!25} Pasta with tomato sauce and meatballs\\
        \hline
        Tuesday 8th & Fried Breakfast & Falafel and wraps & Chilli and potato with kidney beans and veggie mince \\
        \hline
        Wednesday 9th & Bircher Muesli & \cellcolor{red!25} Bagels with cold cuts, cream cheese and coleslaw & Lentil dahl; Cauliflower curry; and green bean curry with rice\\
        \hline
        Thursday 10th & Garlic fried bread & Lentil soup with rolls and cheese & \cellcolor{red!25} Sausages and mash with gravy, fried onions and peas\\
        \hline
        Friday 11th & Porridge & \cellcolor{red!25} Hot dogs with sauerkraut & Stir-fry with rice noodles\\
        \hline
        Saturday 12th & Leftovers & Leftovers & \cellcolor{gray!25}\\
        \hline
    \end{tabular}
    }% end of \RaggedRight
    \caption{Draft menu}
\end{table}
Puddings will rotate so each village will have a different pudding on each day.
The pink cells are meals that contain meat, vegetarian and vegan options will be available for all meals.

\chapter{Programme}
Below is an itinerary for Venturer Camp! Please note this is subject to change, we have included it to give you an idea of what will happen during the week but timings are not confirmed yet. \nl

There is no central programme in the mornings to give you a chance to bond and do clan in your villages. You should decide together when to do breakfast, morning circle, lunch, clan, and if you wish village activities, just make sure there's time to fit everything in before central activities start at 14:30! Each village will also be allocated one day where all their participants get the chance to do either canoeing or climbing (or maybe both subject to final numbers), and this will include a morning and an afternoon session, so there will be one day where your village has central programme in the morning as well as the afternoon.
\begin{table}[H]
    \centering
    {\RaggedRight
    \begin{tabular}{p{0.2\textwidth} p{0.2\textwidth} p{0.07\textwidth} p{0.2\textwidth} p{0.08\textwidth} p{0.08\textwidth}}
        \textbf{Day} & \textbf{14:30} & \textbf{16:00} & \textbf{16:30} & \textbf{17:30} & \textbf{18:00} \\
        \hline
        Saturday 5th & \multicolumn{4}{l}{Arrivals} & Dinner \\
        \hline
        Sunday 6th & \cellcolor{wcfGreen}Centres (hiroshima day focus) & Break & \cellcolor{wcfGreen}Centres (hiroshima day focus) & Break & Dinner \\
        \hline
        Monday 7th & \cellcolor{wcfGreen}Centres and adventure activities & Break & \cellcolor{wcfGreen}Centres and adventure activities & Break & Dinner \\
        \hline
        Tuesday 8th & \cellcolor{wcfYellow}Centres and adventure activities & Break & \cellcolor{wcfYellow}Centres and adventure activities & Break & Dinner \\
        \hline
        Wednesday 9th & \multicolumn{3}{l}{\cellcolor{wcfYellow}Wide game} & Break & Dinner \\
        \hline
        Thursday 10th & \cellcolor{wcfRed}AGM, centres and adventure activities & Break & \cellcolor{wcfRed}AGM, centres and adventure activities & Break & Dinner \\
        \hline 
        Friday 11th & \cellcolor{wcfRed}Centres and adventure activities & Break & \cellcolor{wcfRed}Centres and adventure activities & Break & Dinner\\
        \hline
    \end{tabular}
    }% end of \RaggedRight
    \caption{Afternoon Programme Schedule}
\end{table}

\begin{table}[H]
    \centering
    {\RaggedRight
    \begin{tabular}{p{0.2\textwidth} p{0.1\textwidth} p{0.2\textwidth} p{0.1\textwidth} p{0.2\textwidth}}
    \textbf{Day} & \textbf{19:30} & \textbf{20:30} & \textbf{22:00} & \textbf{22:30} \\
    \hline
    Saturday 5th & News & \cellcolor{wcfDarkGreen}Central Programme  - Band(s) & Sign in & \cellcolor{wcfDarkGreen}Programme continues until 23:30 \\
    \hline
    Sunday 6th & News & \cellcolor{wcfGreen}Village Programme & Sign in & \cellcolor{wcfGreen} \\
    \hline
    Monday 7th & News & \cellcolor{wcfGreen}Central Programme - Ceilidh & Sign in & \cellcolor{wcfGreen}Programme continues until 23:30 \\
    \hline
    Tuesday 8th & News & \cellcolor{wcfYellow}Village Programme & Sign in & \cellcolor{wcfYellow} \\
    \hline
    Wednesday 9th & News & \cellcolor{wcfYellow}Merry moot, with an exciting guest host & Sign in & \cellcolor{wcfYellow}Programme continues until 23:30 \\
    \hline
    Thursday 10th & News & \cellcolor{wcfRed}Village Programme & Sign in & \cellcolor{wcfRed} \\
    \hline
    Friday 11th & News & \cellcolor{wcfRed}Central Programme - Band(s) & Sign in & \cellcolor{wcfRed}Programme continues until 1am\\
    \hline
    \end{tabular}
    }% end of \RaggedRight
    \caption{Evening Programme Schedule}
\end{table}
The colours in this table reflect mini themes, based on the main theme of Mythology, which will change every few days. We'll announce these soon so you can prepare your costumes! 

\section{Centres we have confirmed}
\begin{description}
    \item[Mythology] coordinated by Eastbourne Venturers and DFs - a centre to explore mythology from all over the world and all time periods, through discussions, craft and more
    \item[Activisim] coordinated by Cherry Tucker - a centre to explore activism regarding the climate, anti-racism, LGBTQI+ rights, feminism and more
    \item[Media] coordinated by Gus - a centre where you can create your own news pieces, adverts, or other videos to be shown on the camp news each night
    \item[Solar Cinema]  coordinated by Ash Taylor - a centre where you can see blockbusters and smaller releases, including some bigger events where discussions and Q\&A sessions follow the film
    \item[Radio]  coordinated by Tyler Eckersall - a centre where venturers can run their own radio station, be that a talk show, an interview, simply djing, or a mixture
\end{description}
\section{Centres we would like but need more volunteers for}
\begin{description}
    \item[MEST-UP] Woodcraft Folk DFs' peer support project, where participants can get advice on topics such as sex and relationships, drugs and alcohol, mental health and more
    \item[Nature/ Bushcraft] a space to explore our gorgeous surroundings at Biblins, and/or learn some bushcraft skills
    \item[Arts]  a mixture of drop-in craft activities and jamming, and more structured performing and visual arts workshops 
\end{description}

There will also be activities in the cafe, and a wide game on the Wednesday afternoon. The wide game coordinator would appreciate some more help though, especially from anyone who has run a wide game at a big camp before! 

If you are interested in volunteering with any of the above, or would like to offer something else to the programme, we'd love to hear from you! Get in touch on 
\href{mailto:programme@venturercamp.org.uk}{\texttt{programme@venturercamp.org.uk}}.

\chapter{International Volunteers}
Venturer Camp will be supported by a team of international volunteers. This year we will be supported by two European Solidarity Corps volunteers, who are supporting activities at both Cudham and Biblins until September. Sabrina and Noemie are really excited about supporting Venturer Camp activities during their placement.\nl

The ESC volunteers will also be joined by a team of 15 international volunteers recruited by Concordia. The team will come from Slovenia, France, Spain, Mexico, Armenia and Turkey. The Concordia team will arrive at Biblins before camp to help set up and will remain on site after camp to help take down.\nl

Please span the world with friendship and give our international volunteers a warm Woodcraft Folk welcome.

\chapter{Volunteer Wellbeing}
Volunteer wellbeing is just as important as the safety and wellbeing of the young people. Without volunteers the camp wouldn't be happening.
If you are coming to Venturer Camp 2023 as a volunteer there are a number of things you could put in place before the camp
\begin{itemize}
    \item Get to know the other adults you are coming to camp with so you feel more able to talk to each other if/when you're feeling stressed or overwhelmed.
    \item Spread the workload of preparing for the camp. There are many jobs volunteers who are not intending to attend the camp can take on such as fundraising, organising transport, helping young people acquire camping equipment etc. The people doing the planning don't need to be the same people actually going to the camp.
    \item Arrange in advance, between you, for each facilitating adult in your group to have some time `off' being responsible for the young people and the camp during your week at Biblins. That could be an afternoon or even a full day, depending on your collective capacity.
    \item Plan to bring something to do to camp that you enjoy - a book to read, a craft project, sketchbook, musical instrument, whatever you would do in your spare time at home.
    \item Remember to pack some home comfort whether that's your slippers, your favourite mug to drink out of, a hot water bottle etc.
\end{itemize}
\section{Positive Energy Bubble}
During the camp itself all volunteers are invited to spend time at PEB - the Positive Energy Bubble - for some quiet time, to have someone to talk to, or even just as somewhere to get a hot drink and have a sit down


\chapter{Safeguarding \& Risk Management}
\section{Safeguarding Team}
Safeguarding support and response will be provided by an on-site Safeguarding Team, which includes:
\begin{itemize}
    \item Debs McCahon
    \item Felix Pepler
    \item Catherine Tuffrey
\end{itemize}
Each Village will be also required to nominate a Safeguarding Lead.

\section{Site Stewards}
Volunteer stewards will also be available on site - answering questions and checking people on and off site.\nl

More information about stewards, and their role will be available in the next Information Pack.

\section{DBS/ PVG Screening}
All campers aged 18 years and over will need to follow Woodcraft Folk's \href{https://woodcraft.org.uk/resources/volunteer-screening/}{Screening \& Vetting procedures}, this includes: 
\begin{itemize}
    \item Submitting two suitability references
    \item Completing an enhanced DBS 
    \item PVG membership (in Scotland only) 
\end{itemize}
All volunteers are advised to check their DBS status before May 2023 and a minimum of 4 weeks before the event. Please contact your District Membership Secretary or \href{mailto:membership@woodcraft.org.uk}{\texttt{membership@woodcraft.org.uk}} for help with your DBS or PVG application.

\section{Adult Ratios}
Each group attending camp will need to bring enough adults to supervise the young people in their group - supervision is best done by adults who know the young people, understand their support needs and already benefit from a positive relationship.\nl

Like a typical group night, every group must have a minimum of two screened adults present at all times (travelling to and from the site and throughout camp). The coordination team recommends a minimum of three screened adults accompany groups, as this will allow for volunteers to have time off and will better support their own wellbeing as well as the wellbeing of the young people they are supporting.\nl

When calculating the most appropriate ratio of adults to young people you should consider:
\begin{itemize}
    \item The 24/7 nature of the event
    \item Travel time and method
    \item The environment and any planned activity risks
    \item The strong recommendation that every volunteer gets some time off
    \item The experience of the volunteers and how well they know the Venturers
    \item The experience of the Venturers
    \item The individual and group support needs of the Venturers
\end{itemize}
Advice and support on ratios and effective supervision is available from \href{mailto:safeguarding@woodcraft.org.uk}{\texttt{safeguarding@woodcraft.org.uk}}.

\section{Responding to incidents, accidents and disclosures}
If you would like to discuss any safeguarding or child protection issues please contact \href{mailto:safeguarding@woodcraft.org.uk}{\texttt{safeguarding@woodcraft.org.uk}}.

\chapter{Site Safety}
Group leaders are reminded that they are responsible for the safety and well-being of the young people in their group at all times. Please see Woodcraft Folk's latest residential \href{https://woodcraft.org.uk/group-guidance/camping-and-residentials/}{guidance on the Woodcraft Folk Website}. \nl

There are some aspects of the site which Group leaders should be aware of including:
\section{Public Access}
There are a number of rights of way through the site, including access to the bridge in the centre of the campsite, which are well used by walkers. Thought should be given to the positioning of tents, marquees and cars to create natural barriers to the camping circle. Anyone in the camping circle without a Venturer Camp wristband should be challenged and asked to return to the public footpath.

\section{The River Wye}
Many of our camping pitches back directly onto the river, while the riverbank is clearly marked, it is not fenced. \nl

The river is fast moving and not suitable for swimming in. Canoes may launch from the launch on the eastern end of the site.

\section{Vehicles on Site}
Groups are asked to keep driving on site to the absolute minimum needed to arrive/leave site with their group.  There is a speed limit of 10 miles per hour and drivers should be aware that the path through the site is very popular with local walkers as well as being used by other campers.\nl

Vehicles must be parked in the designated car park and not adjacent to villages during the event, unless the vehicle is required for accessibility reasons.


\backPage
\end{document}
