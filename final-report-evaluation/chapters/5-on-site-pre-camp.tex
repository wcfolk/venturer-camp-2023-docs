\chapter{On-Site Pre-Camp}
On-Site Pre-Camp was held at Biblins on the weekend of Friday 30 June - Sunday 2nd July. Not many members of the Core Team attended the event, in part due to dates and in part due to travel times. No leaders from delegations attended, despite being invited. We believe this to be due to the fact that many people in Woodcraft know Biblins well and wouldn't want to make the trek down there just to be told the same things they could find out online.\\

The dates were chosen to minimise exam clashes, which given the young nature of the Coordination team was for the best however it left us with very little time to do tasks and pick up actions after the event. This, whilst okay for some team members, left others with many actions to complete and little time to do them well in. The late date choice impacted other people not on the direct coordination team, for example getting the camp map designed.\\

There had been some discussion in the weeks leading up to pre-camp about if the event should go ahead or not, ultimately it did go ahead however some people felt the go or no-go decision needed to be made sooner as it resulted in confusion for some. It was also noted by attendees of the weekend that there wasn't an obvious booking contact for the weekend in the event that someone needed to cancel / amend their booking - while this did stop someone from cancelling their booking, it should be considered in the future. Furthermore, attendees who were in the central organising team but weren't core team didn't get any advanced information due to the Discord Structure. This is an oversight on the camp coordinators part, and the issues were rectified as soon as possible. \\

The agenda for the weekend was designed with fluidity and a chance for teams to take the time to do what they needed to do in mind. This resulted in a good balance of time where all the people present on site were taking part in site-wide activities and taking time to do what they needed to do in small groups (there were a series of bonus tasks completed throughout the weekend).\\

The pivotal moment during Pre-Camp was deciding the layout of the central area. Having this discussion on site enabled a participative \& cooperative approach where the majority of the relevant teams were able to contribute. Not having anyone with direct oversight of programme made this process slightly more complicated as it meant the people on site were relying on data in a number of spreadsheets to be up-to-date and from conversations had between team members to be able to decide placement for structures. Ultimately, this worked out fine - however it should be noted that when doing site layout, it's advisable to have members from the programme team present to determine marquee layout. \\

Another especially useful session was around teams sharing updates. This is something the coordinator had struggled to get teams to do in the leadup to this so having teams all in one place presented the perfect opportunity to share updates. Updates were varied, with some teams being basically ready for camp tomorrow and others still having lots of work - the Pre-Camp weekend displaying this to them and them then being able to get on with it.\\

The final major session of the weekend was Open Space. This time was utilised by some teams for them to have a team meeting where they could determine an actionable plan to take forward and complete; other teams did some maintenance at Biblins; and the remaining people took an inventory of the ex-Common Ground equipment and the ex-Crampton St. Office rooms within the Bunkhouse. In feedback from the weekend - some people found the inventorying to be a waste of time, however others thought it was invaluable as we now fully understood the equipment stored at Biblins, should we need to use any in the event of an emergency.\\

Jack Brown, working week \& takedown coordinator, led a session on what is happening during Working Week and Takedown - which was a very valuable use of time. Having this as the last session of the day proved very beneficial as we were able to use knowledge gained and decisions made earlier in the day to feed into this conversation, making it more productive.\\

Finding a KP for the weekend proved challenging. We were unable to source an external volunteer, which resulted in Noemi and Sabrina (ESC Volunteers based at Biblins) being tasked with KPing for us. They were supported by Lauren Karstadt with menu planning and ordering of food.\\

Throughout the weekend, the people on site at pre-camp were `left to it' with the food, after Noemi and Sabrina dropped it off for us on the Friday night. This was not how it had been sold to us, which caused some confusion. However we were able to make do given the number of competent Woodcrafters present! It was also noted during the evaluation that some of the specialist diets' food wasn't quite right - notably the Gluten Free options. \\

A good suggestion came out of the evaluation whereby for future events like this at Biblins, a member of the team should be encouraged to warden as this would then mean we have relative-unrestricted access to the Warden's Cabin with its electricity and WiFi! We were able to make do sitting on the balcony or on the grassy bank however had the weather been worse - we would've struggled. \\

It was felt that whilst the weekend on site proved extremely useful, it would be also extremely useful to have the entire team together. It was suggested that two in-person meetings could be held in the year of the event, one of them being in an easy-to-access meeting room in a city and the other being a site visit. From the coordinators perspective, this would be much more valuable as only having half the team present was a very difficult experience given the time pressure placed on us to pull off Venturer Camp in a year. 
