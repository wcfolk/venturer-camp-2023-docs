\chapter{Team Structure}

\section{Core vs Wider Team}
When the Coordinator began recruiting for Volunteers to fill out the majority of the team, an idea was vaguely mapped out that a smaller group of people would form the `core team', with support from the `wider team'. Work to recruit volunteers did fill the majority of the core roles initially, with many volunteers who hadn't been involved in a large camp before being recruited. Very quickly, the concept of a `core' and `wider' team was blurred, with it only remaining in the Discord server - roles such as centre coordinators \& stewards weren't assigned `core' access.\\

From a coordinator's perspective, it would've been beneficial to keep this structure in place - with a more defined support structure, as this would have reduced some of the repeated conversations, which may have saved some time. Generally speaking, the volunteer teams felt this way too. Volunteers agreed that a structure including team leads and supporting volunteers makes sense, considering the division of responsibilities and roles; furthermore it allows for more effective decision making by the core team. 

\section{Teams}
Broadly speaking, the individuals who were part of the Venturer Camp 2023 team fit into one of the following teams:
\begin{itemize}
    \item Coordination \& Event Administration
    \item Site Services \& Production
    \item Food
    \item Finance
    \item Volunteer Support
    \item Programme
    \item Stewarding
    \item Communications
    \item Transport Logistics
    \item Merchandise
    \item Sustainability, Environment \& Decarbonisation
\end{itemize}
Each team was structured slightly differently, their evaluations in the subsequent chapters will detail this. \\

There were a number of individuals who worked across a number of teams, for example a number of volunteers were involved in the Solar Power system (under Site Services \& Production's jurisdiction), and the Sustainability, Environment \& Decarbositation team. Considerations should be given here as to who they report to, we managed for Venturer Camp as the team supported each other however in the future plus the volunteers are extremely experienced; however in the future, we need to be more considerate of cross-team working like this. 

\section{Team Unity}
Due to the accelerated time scale in which Venturer Camp 2023 was organised, the unity \& bonding of the team was not as strong as it had been in the past. This was felt throughout the core \& wider team. Unfortunately, due to time pressures - we were unable to take the time to get to know each other and build relationships which would help to make the camp even better.\\

It was generally felt that the lack of in-person meetings contributed to the isolation some members felt. The only in-person meeting offered was On-Site Pre-Camp which had extremely low attendance due to the difficulties of getting to Biblins. For those who did attend Pre-Camp, there was a greater feeling of unity 

\section{Volunteer Recruitment}
Volunteers were recruited for Venturer Camp 2023 through a number of mechanisms, the primary being word of mouth at Common Ground. Having an international camp where the idea of a Venturer Camp is conceived works really well as lots of awesome Woodcraft Folk members are in a field who can be gently persuaded to take on a role at the next camp.\\ 

Further recruitment was done off the successes of Common Ground, using social media and email to reach members, inviting them to join an open meeting which enabled them to ask questions before getting involved. This worked well and filled out many of the remaining spaces on the team.\\

Unfortunately, not all roles were able to be filled. This included some significant roles such as Communications and Administration. These roles were less critical to fill as for the first time with a Venturer Camp - we had a dedicated member of staff working on it who could fill the gap. However, due to shortcomings in other teams, the member of staff's time was filled very quickly. This led to the Coordinator taking on some Production, some Communications, some Administration and some Finance tasks in addition to the Coordination role itself. Some of these additional tasks were shorter term due to either late recruitment or peaks in work. Having additional team members who could be used to plug these gaps would be extremely useful. \\

As would be expected with a large Woodcraft camp, recruitment continues right up until the end of camp for volunteers to take on a variety of roles. These late recruited roles ranged from Wide Game volunteers to stewards. The carefully calculated late-leaving of recruitment works, however it should be noted that for roles with age restrictions, for example, stewarding - can result in having to turn some people away. We ran into the fortunate circumstance where too many under-18s applied during camp for a shift. This left us overstaffed with under-18s and not enough over-18s to pad out numbers with the restriction in place that anyone aged under-18 needs to be paired with an over-18 while stewarding. \\

Venturer Camp 2023 also began to develop a suite of Role Descriptions for key roles. This is something which Woodcraft SLT requested that we do, as managing expectations for Volunteering roles is critical. For future events, it would be suggested to set the dates for any compulsory attendance meetings in advance, and require people to be available to attend to be able to take on the role. \\

We ran into an issue with a volunteer taking on a local group role as well as a significant central role. Due to ensuring the young people had sufficient leadership capacity, we were unable to allow the volunteer to take on both roles. This resulted in one of the teams loosing a core volunteer with lots of experience \& knowledge of the site. For future events, it would be suggested that the expectation that core team members only take on that role is made clear.

\section{Volunteer Onboarding}
Venturer Camp had no real Volunteer Onboarding process. This is something which it would be recommended to explore for future camps.\\

It was found that through not having a volunteer onboarding process, beyond ``join our Discord server'', we ran into issues where team members' contact details weren't available and it wasn't always clear what people were signing up for, either to themselves or to members of other teams. It would have helped greatly to have a short Google Form which new team members are expected to complete that asks for their basic information (email address, telephone contact number, role, etc) as well as any access needs they have and sets out expectations for meeting attendance etc. From this form, relevant data would be passed to relevant people which would hopefully enable people to feel better supported in the team as well as having more ways to contact them. 
