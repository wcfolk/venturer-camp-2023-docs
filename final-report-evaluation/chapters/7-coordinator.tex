\chapter{Coordinator}
The role of coordinator at a large Woodcraft Folk camp is a unique one, with it being a pivotal role in its own right as well as potentially needing to `mop up' and `fill in gaps' for other roles on the Core Team.

\section{Team Structure \& Evaluation Of It}
The role of Coordinator was taken on solely by Thomas Boxall.\\

At the start of the project, it was aimed to recruit a second person to take on the Coordinator role alongside Thomas, thus sharing the workload and ensuring that there would be more support available to teams. This did not happen and therefore Thomas was the only coordinator.\\

Plans had been made where Thomas would line manage the team leaders for individual teams, who would in turn support their teams. However, due to the rushed nature of the project - this structure was never able to be put into place. It was a continual struggle for the coordinator as to what they could get involved in to support teams without overburdening themself. In the end, the coordinator directly supported the majority of the core team, which was much more intensive in relation to communication and slowed progress somewhat as the coordinator had to contact many people who would often take some time to respond before anything could happen.

\subsection{Staff Support}
A number of different staff members were involved in supporting the functioning of Venturer Camp at different times. The Coordinator was expecting that, from autumn 2022, Woodcraft Folk's Events Assistant would have Venturer Camp as their sole focus. However, this was not specifically discussed or agreed as part of the project kick-off. This led to some underlying confusion regarding the Events Assistant's role supporting camp. Workflow and the allocation of tasks were complicated by the Coordinator not having line management responsibility for the Events Assistant. While it is not usual in Woodcraft Folk for line management of staff to be undertaken by volunteers, a more detailed agreement at the outset about project structure and management would have reduced the potential for confusion.\\

Having the Events Assistant available was a massive relief; they were able to be assigned some tasks which took burdens off of the teams and the coordinator. However, the distribution of their working hours, and the impact of flexible working meant that urgent tasks which could otherwise have been picked up by the Events Assistant ended up being completed by the Coordinator - especially when there were interdependencies with other tasks. \\

Staff Support also came from Woodcraft Folk's Chief Executive and Programme Manager. Both of these members of staff had very clear roles and responsibilities which they took on.

\subsection{Support In The Lead Up To Camp}
The Coordinator did not feel well supported during the project - neither by his team of volunteers, nor by the organisation more widely, and felt that he was putting in more work than the majority of other volunteers working on the project (rather than sharing in a truly cooperative effort). This has had a significant and detrimental impact on other areas of The Coordinator's life, which shouldn't be underestimated. Through the evaluation interviews, it transpired that other individuals were in similar situations - this further emphasises the need for more face-to-face interaction within the core team alongside better-managed deadlines and key milestones being established in advance of camp.\\

The lack of support was potentially made worse by the knowledge that at the end of a day, a staff member could `log off' and stop working whereas there is an underlying expectation for Volunteers to continue working with little-to-no encouragement for time off. The Coordinator's main working hours were in the evening - thus feeling like they were `Always On'. The coordinator found that coordination began to take over their life, for example while in a lecture - they would be responding to emails or discord messages, or while waiting for their dinner to cook - they would be catching up on edits to documents. \\

The Coordinator noted that the experience of working to Coordinate teams felt at times like they were ``screaming into the void'' with little return from teams. At times, it felt as though the Coordinator was the only one doing significant work to progress the camp forward whereas other teams were taking their time to do tasks. This should be rectified for future events through setting a clear timeline in advance of the project - and ensuring this is kept to and that the deliverables noted on it are actually delivered on time or a good excuse is at the very least.\\

There were large periods of time when The Coordinator felt he was putting lots of messages out to volunteers, generally and individually, and getting nothing back. While individually, volunteers might have had a handle on the things that they were expected to do, there was little understanding or acknowledgement of the impact on others of things being pulled together late or at the last minute, and the lack of feedback from people in key positions raised levels of stress and concern for The Coordinator in particular. We need, as an organisation, to find approaches to sharing goals and holding each other collectively and supportively to account, rather than what feels like a top-down management of volunteers in a project of this nature.\\

The Coordinator's mental health suffered as a result of their role in Venturer Camp 2023, as did their first year final grades of their degree. They had numerous breakdowns with little support from Woodcraft Members, most of their support came from people outside Woodcraft.

\section{Supporting Events}
\subsection{Virtual Pre-Camp}
This was a lot of stress and hassle to organise from the Coordinator's perspective. Chasing the team to run their sessions when they were already behind in what they were doing was difficult and led to lots of additional stress on the Coordinator's part.\\

The sessions the coordinator ran went fine, with relatively low attendance. The coordinator passed off quite a few of the sessions which were due to be ran by themself to other members of the team due to capacity limitations.

\subsection{On-Site Pre-Camp}
The on-site pre-camp weekend felt like the most useful meeting that the project team had - although lack of commitment to the date from the team as a whole meant that all the right people were not there. Actually being on site was hugely helpful, and the coordinator regrets not doing this earlier - they reflected that there is a significant difference between being familiar with a site from having visited / camped there previously, and seeing it through the eyes of an organiser of a camp on a significant scale.\\

The pre-camp weekend would have benefited hugely from having at least one person from every team present - and from team leaders ensuring that this person was briefed to speak on behalf of the team as a whole. It would also have been beneficial had one or both of the ESC volunteers on placement at Biblins attended the pre-camp weekend to share knowledge about the site and provide additional detail to support the team's understanding.

\subsection{Working Week}
The early part of the week, where The Coordinator \& Working Week Coordinator were on site alone moving equipment and slowly beginning setup was quite relaxed and enjoyable. Having this time for site acclimatisation for an unfamiliar site proved extremely valuable as the Coordinator was able to spend the time getting to know the site before the masses descended.\\

Much of Working Week for the Coordinator was spent `putting out fires' and dealing with issues that arose throughout the week. This wasn't unexpected however the frequency and complexity of issues further emphasised the need for better planning ahead of Working Week. An example of this was where the locks on the Bunkhouse storage were changed by the BIblins Staff Team to prevent unauthorised access and on one of the days in Working Week, no members of the staff team were working from site. The issue of gaining access to these stores fell to the Coordinator who spent the morning phoning different people trying to work out who knew the padlock code. It ultimately transpired that one of the ESC volunteers knew it and they were able to unlock it, though this had not been communicated to the camp volunteers.\\

The Coordinator felt that by Wednesday (when the majority of people arrived for Working Week), `all hell had broken loose'. This feeling came from key volunteers arriving sporadically and there being a mismatch between stress experienced by people who had been on site for a few days and those arriving and realising that tasks they thought had been done, in fact hadn't while they were feeling the way into their role and getting acclimated to the site.\\

The Coordinator found it extremely helpful to have a separate Working Week Coordinator. This position was held by Jack Brown - who was able to deal with much of the operational decisions and planning for Working Week, leaving the Coordinator to plan further into the camp. 

\subsection{Takedown}
By the time that campers had departed the site, the small number of volunteers left on site were exhausted and burnt out. It was hoped that Concordia volunteers, who had less reason to be exhausted, would add valuable capacity at this point, but they didn't and it was generally easier to do things ourselves than try and wrangle Concordia to do it, which contributed to further ill-feeling among Woodcraft Folk volunteers.\\

Takedown provides an opportunity for volunteers to celebrate and debrief informally, but by this point some volunteers were (physically and emotionally) unable to contribute meaningfully, but still wanted to be part of it. More of a contract or briefing to give people clarity about what is expected of them would help, and the takedown phase would also benefit from dedicated volunteers to provide `fresh blood', though this is hard to achieve in practice. Managing these volunteers fell to the Coordinator - a task which they did not feel prepared to complete or able to tackle maintaining positive social interactions. \\

Takedown for Venturer Camp 2023 was a blur of exhaustion and the VengaBus for many attendees. This was extremely unfortunate given the success of the camp - that for some volunteers it ended this way. Having fresh volunteers available who are able to support the exhausted central team is critical for a successful takedown. 

\section{On-Camp}
The Coordinator found that Village Coordinator meetings in the morning were a positive part of the routine - it was good to see people face to face and hear what's going on for them. The Village Coordinators were a lovely group of people who were positive and constructive at the meetings. Sometimes it felt like other members of the team who attended these meetings, led a bit too heavily - where they would decide something without deferring to the Coordinator. \\

The core team didn't meet up at camp - it didn't feel there was time. The idea of an `on shift' co-ordinator was a good one, but there was insufficient handover when this changed from one person to another - possibly due to the timing of this shift-change, but it's possible that there isn't a `good' time. In conclusion it didn't really achieve what it set out to (and The Coordinator still felt `on duty' even when he wasn't wearing an orange jacket) - though with a pair of co-ordinators doing more of a double-act as equal partners, and a stronger `core team' this might be more successful.\\

The Coordinator found that they mostly experienced camp from the inside of the Camp Office, and lacked the time and space to walk around the site and soak up the atmosphere. This may have contributed to a tendency to see the problems and issues that were arising, rather than appreciating that the majority of camp was running well and most everybody (or at least all the young people) was having a good time and a positive experience.\\

The Coordinator did manage to get some time off, however this was mostly spent sleeping or off site. Being off site felt like the only way to get away from the demands of the role for the Coordinator. 

\section{Coordinator Specific Insights}
Taking on a Coordinator role at a large Woodcraft Folk Camp is both a blessing and a curse. It's a great opportunity to meet new people and try new things but it's a huge responsibility which comes with an unpredictable and unwieldy amount of work. Often Woodcraft takes the mentality that once a Coordinator is found, the event can go ahead - with details to be ironed out later. This is potentially the wrong approach to take and future Coordinator's shouldn't allow planning to commence, until they have a team sourced and inducted.\\

Much of the time at the start of the project was spent dealing with what dates to host the event in. We settled on 2023 in the end due to Venturer's decisions, not decisions based on the capacity or willingness of volunteers. This fundamental decision underpins quite literally everything - from how quickly planning progresses to the wellbeing of the team. It's the wellbeing of volunteers that should be at the heart of everything we plan like this. The Coordinator notes that no one has walked from the organisation as a direct result of Venturer Camp 2023, and the stress it puts on individuals; however we can't rely on the good will of volunteers to stay in an organisation which works them to the breaking point frequently.
\subsection{What The Role Entails}
This is a really difficult question to answer. The role of coordinator at a large Woodcraft Folk camp is hugely varied and will change minute-to-minute. \\

For Venturer Camp 2023, due to not recruiting anywhere near enough Volunteers the Coordinator ended up dealing with lots of the Event Administration, Communications, some equipment wrangling, early finance activities and monitoring finance alongside all the standard Coordinator tasks. This was a huge task which the Coordinator often would spend between 15 and 20 hours a week working on relatively constantly from the word `go' in September 2022 to June 2023 where it ramped up considerably.\\

After camp - the role continues. The Coordinator holds responsibility to ensure that the evaluation is completed as well as the finances are left in an orderly condition, with all outstanding invoices \& expenses paid and all income has been gathered; and that any loose ends have been tied up. This took a considerable amount of time for Venturer Camp due to the Coordinator's limited capacity after camp.
