\chapter{Takedown}

Takedown formally started on the afternoon of Saturday 12 August, after the main camp had concluded \& attendees left. Due to site restrictions, we had instructed all attendees to be off site by midday as there were a number of other groups scheduled to be camping at Biblins that night. Takedown was scheduled to run until Wednesday 16 August, and the Burrow had been booked as such - however due to a desire to get away as quickly as possible, Takedown formally concluded at around lunchtime on Monday 14 August. Takedown was coordinated by Jack Brown, who is also one of the coordinators of Woodcraft Folk's next big camps. Having a dedicated Takedown Coordinator proved extremely helpful as they were able to take the lead on planning when equipment should be taken down and manage the storage of it, ensuring it was stored in a safe way for the 2025 camp.\\

On paper, it would seem that we had a good number of people booked to come to Takedown, with 25 coming. This figure included the 12 Concordia volunteers, so in reality - we had about 13 people at Takedown. Some of these people were only able to stay for one day, which left a very small group of people doing takedown. This was extremely difficult, as many of these people had been to Working Week and held key (high stress) roles on camp, which lead to a number of delirious moments throughout Takedown. Ultimately, we made it work - however for larger events where there would be more to do, this would be too great a task for the people left behind. For future camps, it may be worth exploring asking a Venturer or DF group to stay behind and support with Takedown. \\

To make the takedown crew's lives easier \& due to inclement weather, a decision was taken to strike as much of the central area as possible on Friday 11 August. This, whilst leaving the last night to have a strange looking central area, proved extremely valuable as we were able to drop 4 of the 5-by-10 m marquees after the central program finished. Whilst the Camp Coordinator \& Takedown Coordinator did end up orchestrating the taking down, extra people were able to be involved in getting the marquees down, sharing the load of the takedown.\\

The main food depot tent was also able to be taken down earlier in the day on Friday 11 August. This was returned to New Barnet overnight in the Luton Van, a decision which left us without the van on site to use to move the smaller centre marquees after taking them down - however a smart decision nonetheless, as it saved the hire of another van just to collect the marquee after camp. One of the parents of a young person drove the van to New Barnet and back, after being added to the insurance by the hirer. We were able to use a minibus on site to move the marquees, enabling them to be moved to the bunkhouse with ease. \\

The Luton Van had been scheduled to return Ealing's equipment to Ealing on Saturday morning, then return Lewisham \& Greenwich equipment to south London on Sunday where it would stay and not return to site. This left the takedown team with a shortage of van for the majority of Saturday and Sunday - something which had been planned for and we made work, however was another complication. Due to the lack of a van, we had to work in the earlier mornings and later into the evenings to ensure that all the required movements of equipment were managed; this was accompanied by using a car to transport some of the smaller, lightweight equipment. Having a van available isn't the only piece of equipment which is vital for a successful takedown - you also need some basic tools and ladders available for striking marquees, as we had hung lights high in the Ex-Swindon spiky marquee which we needed to de-rig. We were able to find a ladder on site to use, fortunately. \\

Villages generally were very good about getting their equipment taken down and off site by the specified time. Some villages chose to strike their large infrastructure the night before, in part due to early departures of some groups and in part due to inclement weather scheduled for Saturday 12 August. Villages were also generally pretty good about not leaving their pitches messy, other than one village who left a large amount of rubbish on the site. This was dealt with by the site staff team, however it should be noted for future that doing a full litter pick of the site may take a lot of people, a lot of time. \\

Some of the villages also opted to take down their sleeping tents, with the aim of minimising what had to be dried out after a potentially damp morning on 12 August. The Coordinator had given permission for the young people to bivy in the main marquee on the night of 11 August on the understanding that an appropriate ratio of adults would also be in the marquee. At 1am when the central programme finished, a member of Woodcraft Folk SMT overturned this decision which frustrated many campers, souring what had been a positive experience for the most part. For future camps, it would be beneficial to publish guidance and a suggestion around this ahead of camp. \\

In the evening of Sunday 13 August, the team (including the Concordias) had a meal from a fish \& chip shop for dinner then went to the pub for a celebratory drink. It is a tradition within Woodcraft camps to go to a pub for a meal, however we were unable to book a table large enough for all of us. This was a lovely way to conclude the main part of Takedown and gave the team some downtime before the final rush of activity on Monday Morning before the last people left the site. \\

Throughout the takedown weekend, meals were served out of the Burrow Kitchenette, which could just about cope with the size of our group. Leftover food which hadn't been donated to the Foodbank had been taken to the burrow towards the end of camp, which was turned into a number of simple meals. An oversight had been made in relation to the special diets, where not enough equipment or a sanitary space was available to cook these meals in - which is unacceptable. This was rectified by a small number of individuals cooking \& eating from the Warden's Cabin, until we had to close it for cleaning.\\

For future Takedowns, it's suggested that the Vengabus needs to be played more. This may sound strange to people who didn't come to takedown - however for those who were there, we can all still hear it. On a serious note, having a loud speaker and some good music can really help with takedown and morale. 
