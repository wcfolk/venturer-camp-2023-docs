\chapter{Transport Logistics (Coaches)}
\section{Team Structure}
Before camp, the Transport Logistics team was composed of Jeni Dixon and David Neibig. David took a lead on the technical aspects of coach scheduling while Jeni led on customer services and communications to the wider movement. This was an informal structure, which was defined through exchanging emails. David had previously held this role for events on his own so handed off some responsibilities to Jeni.\\

On arrivals \& Departures day - two Concordia Volunteers supported with the on-site operations. They were briefed by Jeni and they were responsible for communications to Jeni, meeting \& greeting coaches and managing the crossing of the bridge.

\subsection{Evaluation of Team Structure}
The structure worked well, with a clear division of responsibilities between David and Jeni. They had previously worked together in Woodcraft, which aided this easy-collaboration. They didn't have any capacity issues, as the right combination of people brought balance.

They experienced challenges on arrivals and departures day, in part due to many things outside of their control and poor communication between the different involved parties.

They found that their position in the `Core Team' was slightly unnecessary, however they do need to be kept in the loop about certain things including the first and last day plans. 

\subsection{Support Ahead of Camp}
The team felt supported by the Coordinator ahead of camp. The team had considered their boundaries of what they could commit and as such managed their workload well. 

They were left to figure out how they were getting to the station themselves, this fortunately worked out fine however in the future, its suggested that a standby car is sourced.

\section{Supporting Events}
\subsection{Pre-Camp}
The Transport Logistics Team did not have an online pre-camp session, nor did either of them attend On-Site Pre-Camp.
\subsection{Working Week}
The team came to the final few days of Working Week. They made this decision to ensure that they would have internet connectivity to enable them to do any last minute administrative tasks. Having a few days on site enabled them to do any last minute planning on site and to brief helpers.

\section{On-Camp Operations}
\subsection{Arrivals Day}
\begin{itemize}
    \item Got a lift to station, arrived half an hour before first train so could meet the station staff and manager. This gave time for really good cooperation between us and station staff for things like opening barriers etc
    \item Hung around at the station to meet groups when came off the trains. Explained what was happening about waiting or getting straight on a coach. Explained about return journey 
    \item Worked well there were 2 people there - One could do group wrangling and one could do driver wrangling. One could go to the shop and one could do people wranging
    \item Would have been nice to have a packed lunch sent with them in the morning
    \item Came back to site on the last coach, thanked drivers etc
\end{itemize}

\subsection{Intervening Days}
During camp, the Transport Logistics Coordinators had conversations with groups and individuals about their return journey. They then refined their coach schedules and ensured that everything was printed and the right people had access to it.\\

They found it was about a day and a half's work spread across the entire camp. For future events - it's recommended to have drop-ins at set times rather than getting coach-users to leave messages at the cabin or tracking either Jeni or David down.\\ 

They also spent a sizeable portion of time negotiating Scotland's transport Logistics (see below).
\subsection{Departures Day}
\begin{itemize}
    \item Had a strict plan as most villages were leaving early
    \item Village coords need to know when people from their villages are leaving (rough idea even!)
    \item Got up really really early to see over the scotland coach
    \item Took camping chair and sat on the other side of the bridge
    \item Met venturers at bridge and let things happen (which they did!)
    \item Filled coach number 1 and left David to it
    \item Much harder than arrivals as didn't go to bed until half three (This was due to an issue with a change of plan about where venturers were allowed to sleep, nothing to do with this role)
\end{itemize}

\subsection{Support \& Time Off}
The role was not a full time role while camp was taking place. This meant the volunteers were able to do other activities, and were able to use their time as they felt best. They had one evening `fully off'. 

\section{Transport Logistics Specific Insights}
\subsection{Biblins' Coach Access}
Biblins' Coach Access is through a Forestry Track on the Welsh side of the River. The entrance to this track is protected by a padlocked gate. The track is not wide enough for a coach to pass another, meaning only one could be on the track at any given time. The turning circle on the Welsh side of the River Wye is only big enough for two coaches. \\

We solved these issues by instructing a ``coach leader'' to telephone Jeni when they arrived at the top of the track, who either instructed them to wait for a coach to come up or allowed them to proceed down the track. Someone on the campsite would then message Jeni to inform her that the track was clear, once the coach was visible in the turning circle. Jeni would also be informed when a coach departed the turning circle and began to drive up the track. This system worked within the Woodcraft Volunteers however working with different Coach Companies caused some difficulties with managing expectations for driving down the track. 

\subsection{Scotland's Transport Logistics}
The Scottish groups had contracted their own coach to transport all their members, leaders and equipment to and from Biblins. They had lots of issues with their driver not liking driving his coach down the Coach Access Track on arrival day, including him refusing to do it again on departure day. This resulted in a lot of conversations and planning to resolve. \\

We resolved the issues with getting Scotland \& their Equipment to the top of the Coach Track, and to a suitable meeting point for their coach by using one of the Venturer Camp coaches to shuttle them. This worked fine and the Scottish members were dropped in ``the layby''. This is by no means a perfect solution and we were fortunate to have such an accommodating coach company.\\

This situation also highlights the importance of having a small team working to solve problems, as at one stage there were potentially five people working on finding a solution which may have lead to incorrect information being shared internally or externally to Coach Companies. 
\subsection{Timeline}
\begin{itemize}
    \item Oct - Events Assistant booked coach company for a minimum number we would definitely need, with a view to being able to increase if necessary
    \item Dec - asked to put question in booking form
    \item Feb / march - david started to contact the station
    \item Apr onwards - started to have an idea of data and working out worst case scenario
    \item May onwards - started proactively emailing people to ask them if they knew anything then funneling it down
    \item 2 or 3 weeks off from camp - numbers, trains booked and schedule with company made
    \item Week before - almost chilled!
    \item On camp - rechecking data then working out data
    \item Post-camp -  Thanking the coach company, proactively. Dealt with general thank you emails. Could have been some additional admin bits but we didn't need them (around coach stuff e.g vomiting kids)
\end{itemize}

\subsection{Miscellaneous Insights}
The team had a question added to the booking system about whether people would need a coach space or not, this made it much easier for them to plan and assign people coaches.\\

They found having information on who's in what village at the station helpful. This enabled them to group groups together while they were waiting for their coach. \\

They found that the Venturer Camp Coaches email address become a secondary General Enquiries inbox after Working Week had started. They think this is because campers only ever got direct communications from those two email addresses. It would be worth improving communications about who to contact about what for future events, to ensure the questions can be answered by the right people!\\

The Transport Logistics Team should have some basic safeguarding information, including about ratios and ensuring young people have adequate leaders. The Woodcraft Folk Safeguarding team is able to advise on this. 

\subsection{Tips for the Future}
You need a combination of people who are good at logistics and understand the group model within Woodcraft Folk and have Safeguarding knowledge. This has now been proven to be an extremely effective combination!\\

It would be nice to have a structured way to give away unwanted train tickets. This may be more of a thing localised to areas such as London however it would save money for groups. For international camps, this role could be expanded to include to support booking UK Train Tickets for International delegations and potentially commissioning a train.\\

When planning coaches, you need to factor in driver breaks. Each coach company will do this slightly differently, so speak to the company first!\\

Be proactive with communications! Contact each district and ask how many will need the coach and when their trains are / when they're planning on booking them. This works much better than waiting for people to come to you.\\

The Coach Company we used, Jones Coaches, are phenomenal! They are very interested in working with Woodcraft Folk in the future, including at non-Biblins' locations. 
