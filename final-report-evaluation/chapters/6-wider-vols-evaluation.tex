\chapter{Wider Volunteer Team Evaluation}

All volunteers at Venturer Camp 2023 were invited to give feedback through a Google Form, which was circulated to them after camp concluded. For some questions, the form asked volunteers to rate their experiences out of 10. This chapter will analyse that data only - as the dataset is unique with its reach.

\section{Preparation and Support}
Volunteers generally felt adequately prepared ahead of camp, with an average rating of 7.3. However, three individuals expressed lower levels of preparedness, spanning different roles. Notably, the virtual meet your village evening received positive feedback, indicating its effectiveness in strengthening relations within villages and exciting people for the event. Those who felt less prepared also reported feeling less supported, suggesting a correlation between perceived support and preparedness.

\section{Communication Before Camp}
Communication from and with the central team received an average rating of 7.9. However, there were discrepancies, with two respondents expressing dissatisfaction, particularly regarding clarity on early bird rewards and reliance on social media for updates. The delayed initiation of stewarding recruitment might have contributed to lower satisfaction among some respondents.

\section{Role Performance}
Overall, respondents felt they met their role's aims well, with an average rating of 8.4. Notably, no respondent rated their performance lower than a 6. However, a suggestion arose that the camp-wide designated safeguarding leads should not both be in the same village.

\section{Support During Role During Camp}
Despite the generally positive performance ratings, a few individuals felt unsupported in meeting their role's aims, resulting in an average rating of 8.2. Additional support from the central team and clearer boundaries on role responsibilities were highlighted as areas for improvement.

\section{Communication During Camp}
Communication on camp received a slightly lower average rating of 7.4, with concerns raised by both leaders and central volunteers. While some perceived communication issues, others, like a village coordinator, found it effective. This suggests a need for consistent and reliable communication channels to ensure smooth operations.

\section{Personal Enjoyment}
The camp garnered a high enjoyment rating of 8.1, indicating a fulfilling experience for most volunteers. However, feedback highlighted areas for improvement, such as clearer information on village safeguarding leads, better support for MEST-UP, and better off-duty spaces for volunteers.

\section{Willingness for Future Roles}
While the majority expressed willingness to undertake their roles in future camps, some respondents, particularly Village KPs and programme volunteers, expressed reservations. Suggestions for improvement included simplified menus, more socializing opportunities for KPs, and increased volunteer numbers across teams.

\section{Suggestions for Improvement}
Participants provided valuable insights for enhancing future camps, including streamlining signing-out protocols, having a proper opening ceremony, prioritizing sustainability, and incorporating nightly entertainment activities. Additionally, suggestions for clearer communication and designated evening programme MC's leaders were noted.

\section{Highlights and Memories}
Memorable aspects of the camp included the camaraderie and friendships forged, exciting adventure activities, the atmosphere of the merry moot, and appreciation for the food team's efforts. Participation in the central team was also highlighted as a meaningful experience.