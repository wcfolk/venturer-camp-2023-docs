\chapter{Deciding on Dates}
\section{2023 v 2024}
As part of the project kick off, dates for the event had to be decided. However, before we could decide exactly which dates to host the camp on, we had to decide on a year first. This was a complex debate, with many people weighing in on the decision, ranging from Trustees to venturer group leaders to venturers themselves.\\

The ultimate decision was made that the camp should be hosted in 2023. We came to this decision based on the preference of Venturer Leaders and Venturers themselves to host the camp in 2023. This data was captured through a survey which ran for a few weeks in September 2022, the results of which can be seen in Table \ref*{tab:year-survey-results}

\begin{table}[h]
    \centering
    {\RaggedRight
    \begin{tabular}{p{0.2\textwidth} p{0.2\textwidth} p{0.2\textwidth} p{0.2\textwidth}}
    \textbf{} & \textbf{2023} & \textbf{No Preference} & \textbf{2024}\\
    \hline
    \hline
    Leaders & 56\% & 11\% & 33\% \\
    \hline
    Venturers & 67\% & 11\% & 22\% \\
    \hline
    \end{tabular}
    } % end of rr     
    \caption{Results of Year Survey}
    \label{tab:year-survey-results}
\end{table}

The survey also provided space where the respondents could share any thoughts, feelings, or suggestions. The responses to this varied were varied, some of the responses are shown below:
\begin{itemize}
    \item ``I think another national camp would not be supported. We need to have district summer camps to get young people back into camping. We struggled like lots of districts getting pioneers to common ground without a summer camp next year we will struggle to get children back into summer camps after the disruption of covid''
    \item ``Some venturers want a `proper camp before they are too old.' Adults want a break in 2024 before 100 camp''
    \item ``It would be good to have a date to be able to add to the calendar and to try and not book family holidays at the same time.''
    \item ``Personally, I think sooner is best as delaying by another year will inevitably mean some venturers won't get the chance to go. The venturers were mixed in responses, with a small majority favouring 2023 but others saying either or 2024. If it is next summer, please avoid clashing with the international camp in Finland (24-31 July 2023). Also, a question from our venturers is whether DFs who missed their chance to go to VCamp because of covid would be able to attend? Thanks''
    \item ``I have a venturer and a DF happy to help''
\end{itemize}

At the time of making the decision, we did not have a fully fleshed out team. There were significant gaps of knowledge and experience in the following teams:
\begin{itemize}
    \item Food
    \item Site Services \& Production
    \item Programme
    \item Communications
    \item Access \& Inclusion
    \item Event Administration (however we expected this role to be done by the Woodcraft Folk Events Assistant, so weren't worried about recruitment)
\end{itemize}
The lack of some of these teams presented a challenge for project initiation as once we had decided on 2023, they couldn't influence it and as such this deterred people from joining the team.\\ 

A volunteer close to the coordinator who supported him a lot said ``it's very dangerous when we organise the camp in a very short timeframe with a big dependence on one individual as it puts them in a vulnerable position and goes against our aims and principles. Empowering people to take on roles they've not done before is good but they need support in place. We need to ask questions about how they are supported.'' It was these questions around support networks which were answered when they were asked; when the coordinator was struggling, not pre-emptively. Pulling off a Venturer Camp in such a short amount of time, with such a limited capacity team. was not an easy thing to do (yes we did have a large number of people on it but many were limited in their capacity to be involved due to other commitments). Woodcraft Folk put too much pressure on the Camp Coordinator who was also juggling many other things, see introduction, which should never happen again.

\section{Which Dates In 2023}
The decision of what dates we wanted to host the camp on came down to four things:
\begin{enumerate}
    \item The dates which Biblins was available
    \item Festivals \& other attractive-to-young-people summer events
    \item School Term Dates (taking into account the early return of Scottish Schools)
    \item Finnish International Camp Dates
\end{enumerate}

At first, we chose the dates Monday 7 August to Monday 15 August. These dates were put to the Coordination team who decided that we would rather start and end on a weekend to reduce the amount of annual leave adult volunteers would have to take.\\

After some deliberation, we decided on Saturday 5 August to Saturday 12 August. These dates did't clash with any major festivals, were early enough that Scottish Young People would have a few days between camp finishing and their term dates starting, there were a few days between the Finland International Camp finishing and Venturer Camp starting, however the whole of the campsite was not available for all of these dates. There was a group booked onto pitch 1a for the night of Saturday 5 August. The decision was made that we wouldn't need that part of the site for the first night and so could press on with publishing the dates and working out the rest of the timeline.

\section{Group on Pitch 1a}
We took a gamble with the group on pitch 1a being a nice group who wouldn't mind 450 people descending onto the site. For the most part, the gamble paid off - the group were lovely and were interested in what we were doing. However, at first they weren't keen on the numbers of people who were camping to the West side of The Spill. We gave them a wide berth after they indicated this and had no further complaints or comments from them. 