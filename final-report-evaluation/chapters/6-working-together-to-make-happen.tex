\chapter{Working Together To Make Venturer Camp Happen}
\section{Meetings}
\subsection{Online Meetings}
Monthly online check-in meetings were for the majority of the year leading up to Venturer Camp itself. These meetings were designed to check the progress of each team, understand where they have gotten to and work out where they are going next; a way of building shared accountability for the project. It was felt that the meetings would've been more engaging if there had been substantial discussions built into them - rather than sharing completed work. \\

The date and time to hold these meetings varied, a decision taken to ensure that the maximum number of people could be present. Unfortunately, some members of the team didn't engage in many (if any) meetings which lead to them not knowing anyone before they got to camp. \\

It was generally felt across the volunteer team that the meetings would be better if there were more decision-making and bonding activities planned as part of the agenda. It was suggested to have fewer meetings, with each being longer and more meaningful. \\

The online meetings were useful from a Coordination perspective as they were a very good time to get updates from teams, ensuring that teams were making required progress. Meeting with team members seemed to be the most productive way to get updates out of people, as when discussing team progress on Discord - teams would rarely respond which is extremely difficult to gauge progress from. 

\subsection{In Person Meetings}
To save time and money, it was decided that the only in person meeting held for Venturer Camp 2023 would be On-Site Pre-Camp, this would also double as the site visit. Whilst this was a good decision in terms of saving volunteer expenses for in-person meetings, it was a bad decision in terms of team unity \& building relationships between team members.\\

Woodcraft Folk members got very good at using technology to support their engagement through the Covid Pandemic where that was the only option. However, there is very little which compares to having people in a room building relationships, sharing ideas and feeling the excitement from each other about this thing which we are all working towards. \\

It is generally felt across the volunteer team that at least one more in-person meeting would have been beneficial. It was suggested to use a Youth Hostel Association Meeting Room in a large city or convenient location for the members of the team, as Woodcraft has access to these for free, for a day meeting or to have another residential meeting. For either of these options, careful considerations need to be taken as to who gets travel expenses paid to come - and this needs to be made clear in advance of the meeting.\\

It was also suggested that the in-person meetings could `tak-onto' another residential meeting taking place. For example, a General Council or Venturer Committee meeting could run in parallel to a camp committee meeting. This would reduce the overall cost per head for venue hire as well as provide space and time for lots of un-scheduled conversations which are often the most productive.\\

A suggested timeline for the in-person meetings, for Venturer Camp 2023, is shown below:
\begin{itemize}
    \item Late 2022 / Early 2023 - kick off meeting in a YHA in a city
    \item Spring 2023 - Team suggested to join a working weekend at the site to get to know the site
    \item June 2023 - On-Site Pre-Camp on site with team meeting element
    \item August 2023 - Camp itself
\end{itemize}

\section{Communications}
\subsection{Discord}
It was decided early on during the project conceptualisation phase that Discord, a popular instant messaging platform which is similar to Slack or Microsoft Teams, would be used as the primary communication platform for members of the Coordination team. This decision was taken due to the high levels of flexibility with which the platform can be used, and the Coordinator's familiarity with it. A positive side effect was that teams would be given a space in which they could converse with each other, without having to deal with setting up WhatsApp groups if they chose to use it.\\

Opinions about using Discord varied. Some volunteers found Discord effective and user-friendly, particularly for keeping discussions organised (through using channels \& threads) and for facilitating quick responses. However, there were a number of challenges which need to be addressed before Woodcraft uses Discord again. Some team members found it difficult to use and there was a need for better instructions around how to use it \& what the different areas of it were for which was highlighted during the evaluation. It was also sometimes felt that the information was scattered throughout it - making it hard to track down a critical piece of information. \\

It was generally felt that having a category per team, with a few communal categories worked well. Each category had a number of channels, which teams could edit and manage as they so chose. It was suggested that precisely what channels were made available to teams should be explored further as for some users the number of channels available to them was overwhelming. Again, users felt the need for better education around the use of the platform. \\

When new people were added to the server, they landed in a channel which required them to send a message containing their name and role on camp. This was used to assign the team members to the right places within the server. The Coordinator managed administration for the server, with another volunteer also having Administrative access should it be required.\\

A bot (carl-bot) was used to provide some additional functionality, such as auto assigning roles and reaction roles. It was suggested to look into a bot which can also send an email to a mailing list when a message is posted in an announcements channel. This is due to the fact that people would miss critical announcements as a result of not having their notifications configured correctly. 
\subsection{Inter-Team Communications}
In the lead-up to camp itself, some teams needed to contact other teams for a variety of reasons. This proved more challenging to teams than it should have. Through the Discord server, team members were able to identify members of the team to contact - however they weren't always sure who from a team to contact. A suggestion was made to have a who's who document linked from the Discord server identifying who to contact about different things. \\

It was also suggested that core team members should be given access to other team's general conversation channels, on Discord, to improve communication between teams.\\

Once team members were able to gain contact with the right person - the inter-team communication experience was positive, as you would hope within Woodcraft volunteers! 

\subsection{Other Communication Methods}
Some teams found it easier to communicate using other platforms such as Email \& WhatsApp. They used these other platforms as they would be their platform of choice or to avoid using crowded Discord channels. \\

All staff members were formally communicated with through their Woodcraft Folk email addresses. Many staff members engaged in the Discord server where they were given access to everything so they could communicate with teams there. For future camps, it would be suggested to do this in the same way, as having staff members accessible over an instant messaging platform works well. However, guidance should be given to volunteers on appropriate communications to staff as to not overburden them.


\section{Google Workspace Use}
As Woodcraft Folk utilises the Google Workspace as its cloud based productivity suite and email solution, it made sense for us to also utilise it. As part of the domain registration in Autumn 2022, the domain was re-configured as part of the Woodcraft Folk Google Workspace. 
\subsection{Google Drive Folder}
A folder for Venturer Camp 2023 was created within the Events shared drive. All team members were given access to this folder and instructed to store all their files in this folder. They were also instructed not to store secure files or files with sensitive content in this folder.\\

Feedback regarding the use of the shared folder was mixed, some members of the team found it efficient while others had difficulties in locating files due to folder naming and save-location issues. Members of the team would've been appreciative of a clearer folder structure and a more user-friendly approach.\\

A sub-folder was created for each team, as well as supporting events. This also applied for teams where there weren't people taking on those roles which may have led to the confusion. \\

With the Venturer Camp folder existing within the Events shared drive, this meant that lots of people had access to the Venturer Camp folder just by having access to the Events drive, which had been assigned for a previous event. This is a potential GDPR risk, especially with no volunteers receiving adequate Data Protection training. Due to this, it would not be recommended to follow this file saving approach in the future; with the recommendation being that each individual event gets its own Shared Drive created. After the event has concluded, the files can be moved to the main Events shared drive, for archival purposes. 

\subsection{Email Addresses}
As already discussed, we made use of the Woodcraft Folk Google Workspace, which included access to Gmail. We decided to use an individual email address system, rather than sharing passwords. This involved each member of the core volunteer team having their own personal @venturercamp.org.uk email address, with these being assigned delegate access to the shared team inboxes (for example, info@venturercamp.org.uk). As people understood how the system worked, they got the hang of it. However, it took members of the team some time to appreciate how the system works.\\

With the way Google manages delegate access to inboxes, you are unable to add this as an account on a mobile device without knowing the password, for this reason - it would be recommended to further explore how to do email addresses. Perhaps rather than a single individual holding all the passwords for all the accounts and dealing with all the delegation, it would be useful for a member of each team to know the password but be expected to use delegate access for the most part. 
