\chapter{Introduction To The Project}
\section{Idea Conceptualisation}
Venturer Camp normally happens every 3 years, as a national camp for Venturers (the 13-15 year olds in Woodcraft Folk). 16 year olds are also normally allowed to come as participants if they haven't experienced a Venturer Camp before.\\

Typically a volunteer team provides infrastructure for the central area, a central menu, some central programme in the form of workshops in the daytime and some evening entertainment, and put groups in `villages' where they will eat, sleep and do clan. Group leaders bring their young people and organise their village including infrastructure, clans and activities.\\

For a long time Venturer Camp happened at Drum Hill Scout Campsite in Derbyshire, but in 2019 for the first time we held it at Biblins, Woodcraft Folk's own site in the Wye Valley. \\

There were two key things we wanted to do differently from past Venturer Camps this time round. Firstly, due to Common Ground being postponed 2 years because of COVID, this camp was to be 4 years after the Venturer Camp before opposed to the usual 3. For this reason, we expanded the participant age range to 17, to ensure as many young people as possible get to experience a Venturer Camp. \\

Another focus of this camp was volunteer support. Building on Common Ground, where for the first time there was a volunteer wellbeing role on the camp team, we wanted to ensure all volunteers (both central and village volunteers) were well supported on camp. We didn't manage to do as much as we wanted in this respect as we were only able to recruit one person for the volunteer support team who could support ahead of camp and on site, but the majority of the central team were able to get a day off with planning and support, which definitely hasn't been the norm in the past.

\section{Planning Timeline}
The decision was made in summer 2022 to hold a Venturer Camp the following August (we go into more detail why in the `What Dates' section). Because of this we only had a year to plan the camp when preparation usually starts around 2 years in advance. While most things don't happen until the final year, this extra year is pretty key when it comes to recruiting volunteers, building trust and care among the team and getting started with some key decisions and actions. Much of what could have been improved this camp came down to not having enough volunteers or volunteers not feeling confident/part of a team, which may not have been fully rectified by having an extra year but this almost certainly would have helped.\\

We have shown a Venturer Camp can be planned in a year, but not without either having a full team from early on or seriously overworking some members of the team. Therefore we would recommend always beginning plans 2 years in advance where possible.
