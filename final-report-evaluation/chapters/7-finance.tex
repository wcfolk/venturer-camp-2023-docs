\chapter{Finance}

\section{Team Structure}
The volunteer finance team consisted of DJ Hanson. DJ wasn't recruited until a few months into the project, and there was never a designated handover point - meaning there was always some confusion over who was responsible for what aspects of finance. The Woodcraft Folk Head of Resources and Finance Assistant did all of the pre-camp money handling with DJ approving expenses and money tracking.\\

On Camp - it felt like just the treasurer working on finance. This involved running the camp bureau de change, managing tills (cashing up etc), paying cash into the bank and supporting spending money on site. This was a lot of work for one person; an easy off-load of work would have been to train other people on bureau de change operations. 
\subsection{Support}
Overall, DJ felt well supported by the Coordinator and Volunteer Support in the run up to the event, however some delays were felt with getting access to all the right systems, most notably the finance email address. 
\subsection{Team Redesign Thoughts}
It's suggested to redesign the team, making roles more clearly defined. There also needs to be a stronger link between the people managing and spending the money.

\section{Supporting Events}
\subsection{Pre-Camp}
The Finance team attended On-sIte Pre-Camp which they found useful as it was a chance for them to test the payment card machines on the WiFi. It was also useful for the team to get to know the site and the other members of the team.

\section{On-Camp Operations}
\subsection{Daily Structure}
The finance team had quite a varied daily structure, with shop management, running the camp bureau de change, dealing with money and paperwork taking the majority of their time. They were able to use some time in the afternoons to support other things.\\

A sizable amount of their time was spent travelling to and from various internet-enabled locations such as the Pub or Cafe, so they could complete internet-based forms.
\subsection{Time Off}
As mentioned above, the finance team were generally able to take some time off in the afternoons to support other things on site, including their Venturer Group. They also had to walk somewhere most days to get access to the internet. 
\subsection{Support}
The team felt well supported by the Coordinator and Volunteer Support team. They felt that they didn't need much support and emphasised the non-intimidating role of the treasurer.

\section{Finance Specific Insights}
See the Finance chapter for a detailed breakdown of everything money related.
\subsection{Timeline}
Once the treasurer was brought into the project, there was quite a lot of faffing to get access to things sorted. This was partially due to the access requirements of the treasurer - needing access to Woodcraft Folk finance systems as well as Venturer Camp systems. They did some budget stuff, some collecting and chasing of payments as well as checking in with other teams about spending before camp. The treasurer was also involved in the idea, conception and implementation of the V-Coin currency.

\subsection{Areas for Improvement}
The systems and processes that are put in place currently to manage finances are extremely dependent on having good internet access. This is an obvious issue for a site like Biblins where there is a poor connection coming in which is required by so many people. For future events, either better internet connectivity is required or designing processes which don't require internet connection. \\

The budget lines not matching the Central Woodcraft Folk Finance Monitoring caused no end of issues - this needs to be looked into for future camps, with one fitting into the other.\\

There were lots of journeys off-site, it would be more time efficient if these were consolidated into fewer journeys per day. 
