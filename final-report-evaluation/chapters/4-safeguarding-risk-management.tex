\chapter{Safeguarding and Risk Management}
Woodcraft Folk's Chief Executive and member of the Woodcraft Folk Designated Safeguarding Team, Debs McCahon, took the role of on-site Safeguarding lead. She was supported on-site by Catherine Tuffrey and Felix Pepler who both took on shifts on some of the days as well as having mostly off-site and some on-site support from Owen Sedgwick-Jell.\\

In the lead up to the camp, a Risk Register was written. This was produced primarily by Debs, in collaboration with different teams - ensuring that teams were involved in the creation and management of risks. Overall, this worked very well, especially taking an iterative approach to the production of this document as we were able to ensure that it was as accurate as possible at the start of the event. \\

On-Site, there was always a nominated individual who was `on-shift' for Safeguarding. This took the burden of this off of the On-Call Duty Coordinator which worked well. It would be recommended to repeat this at future events. Furthermore, at set times of the day, the on-call safeguarding lead would be available in the Camp Office for a drop-in conversation. With the dynamic nature of Safeguarding, this worked well when combined with always having someone from the safeguarding team on-call. All members of the safeguarding team managed to get at least one day off from Safeguarding throughout the event.

\section{Near Misses \& Incidents}
As could be expected at an event like Venturer Camp, we had a number of incidents, near misses or disclosures. These were effectively managed at the time by the on-site team with support of the off-site team. We did not have any major incidents at Venturer Camp 2023. Shown below is an outline of the types of incidents and near misses which were reported throughout the event:
\begin{itemize}
    \item 5 relating to alcohol and intoxicating substances
    \item 2 relating to young people being AWOL (see the appendices for Missing Young Person Procedures)
    \item 7 relating to incidents of unacceptable behaviour by participants
    \item 6 relating to incidents of unacceptable behaviour by volunteers
    \item 6 relating to first aid / medical issues
    \item 9 relating to health \& safety, data protection or the wild escapades of members of the public
\end{itemize}
All incidents were managed effectively on camp, with young people being signposted to a range of support services, and local safeguarding leads maintaining a watch on the situation after camp. There was one incident, involving an allegation against a volunteer which had a significant impact on young people and volunteers in a single village - this was addressed with the whole village via a workshop facilitated by two members of staff towards the end of camp. The safeguarding team continued to manage the situation following camp, but it transpired that the allegation was unfounded.

\section{Policies \& Procedures}
As Venturer Camp is, at a fundamental level, a Woodcraft Folk Project - we utilised Woodcraft Folk's suite of policies and procedures, with a few policies added on top. This was especially useful given that we did not have anyone filling the Admin role therefore work on Policies fell to members of the team who also had other responsibilities.\\

The Venturer Camp team wrote a number of additional policies including:
\begin{itemize}
    \item Payment Policy
    \item Code of Conduct
    \item Missing Young Person Procedures
    \item Under 18 Volunteer Price Policy
\end{itemize}
These were published to our website and linked to from a number of sources throughout the lead up to the event.
