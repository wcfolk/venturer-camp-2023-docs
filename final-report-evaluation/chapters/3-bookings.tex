\chapter{Bookings}
\section{Booking System}
After discussions between the Camp Coordinator and Woodcraft Folk Chief Executive where different booking system options were reviewed, it was agreed to use Ralph Sleigh's custom system which had been first used at Venturer Camp 2019, then Common Ground 2022. Ralph and the Camp Coordinator had initial conversations about what features the booking system would require as Ralph had started re-writing the booking system to use different, more cost effective, technology.\\

During Autumn 2022, the Food and Special Diets teams were consulted about what data they wished to collect from those booking in. As the focus on special diets was new to Woodcraft Camps, a larger amount of data was captured about each individual booked on. Through using the custom booking system, the implementation of this was exactly as we wished, while using an off-the-shelf system we may not have been able to collect and view this data in the same way.\\

During January 2023, the booking system was tested with the core team. These tests enabled the workflow of approving bookings, the mechanics of booking and ensuring that the wording used in the system was clear. Also during this period, the Booking Handbook was written.\\

The decision was made to require ``applications to book'' before enabling people to book. This decision was made mostly by the fact that Common Ground used this feature. In reflection, it was the right decision to use as it enabled us to ensure people booked in a way that was convenient to us. Rather than individuals booking, we had larger groups booking and we had the say to stop individuals if needed where there was already a group booking for their group. This also enabled us to ensure there were no bookings from people unknown to Woodcraft, but we either received no or very few requests of this nature.\\

Throughout the use of the booking system, a number of bugs and issues arose with it. Ralph responded quickly to all of these, deploying a fix usually within 12 hours of the issue being reported. Ralph was also able to implement features which improved the usability of the system, for example, messaging manager messages to the Coordination Team Discord server which prevented the need to use the emails the system also sent.\\

Overall, the use of a booking system which was developed ``In house'' gave us greater flexibility, greater options and overall a much easier experience than that of an off the shelf system. \\

The booking system allowed users to be able to be assigned backend access to view some or all of the data entered by users. To ensure volunteers who were being given this access had some basic understanding of GDPR, a Data Protection Declaration was created which all backend users of the booking system were expected to sign before being granted access. This worked and while Woodcraft Folk is not providing basic GDPR training for its volunteers, is something which should be repeated for future events. The transcript of the declaration can be found in the Appendix.\\

One thing that wasn't so great about Ralph's system is that it couldn't automatically match up who has / doesn't have membership. Or at least we thought it couldn't. When we discussed this at one point Ralph said there was something he could try to rectify this, which is definitely worth exploring for future events which use this system. 

\section{Bookings Administration}
After the bookings opened - Thomas and Millie processed most of the administration around bookings. This involved things such as: managing applications to book, reviewing bookings to locate any access needs.\\

Before bookings opened - there were conversations around how we want people to book. The decision was taken to try to get people to book as groups, with the general principle of one booking per group; with central adult volunteers booking separately.
This system worked, mostly. There was one case in particular where a region decided to come together and due to the disconnected nature of the young people attending - they took the decision for all the parents to book their children on. Unfortunately, they decided to start making their booking on the date of the early booking deadline. This led to the booking applications being accepted.\\

With having the booking deadlines at midnight, this meant that the responsibility of being on standby to approve the booking applications which come in after the members of staff had finished for the day fell to a volunteer, which for this event fell to the Camp Coordinator. For future events, the booking system needs to be able to specify the time of a booking deadline so that it can be set for a more sociable hour! \\

The booking system generates a unique booking reference for each booking. This then means that when the money is transferred to the Woodcraft Folk Bank Account, we should be able to match up the payment to the booking. This works reasonably well, so far as people used the references and we were able to match up payments. The difficulties experienced with this system came from the structures of the Woodcraft Folk Finance Administration Systems, see below for a more detailed analysis of this. 

\section{Early Booking Deadline}
A decision was taken that the incentive for booking by the early-bird deadline was to receive a free limited edition t-shirt. This decision, whilst good in theory, creates a large financial overhead - where each t-shirt costs more than the financial discount would have been. Once this fact was discovered, it was agreed that we would expect people to have booked and paid their deposit by the Early Bird Booking Deadline to receive a free t-shirt.\\

The communications around the requirements for payment before the t-shirts were given out was not the best due to a number of factors. Primarily, it was down to time pressures that all locations promoting the early bird deadline weren't updated to reflect the payment requirement. The requirement to pay was communicated through the Payment Policy and through social media content leading up to the deadline.\\

Unfortunately, some groups didn't receive the message about the payment requirements. These caused tensions where group leaders had promised their young people free t-shirts and there were no free t-shirts for them as they had not paid by the deadline. These tensions were rectified by selling the group leaders t-shirts `at cost', rather than at the standard camp rate. Whilst not a perfect solution, it was accepted. \\

For those who did pay deposits \& book in advance of the deadline: a Google Form was used to gather the size requirements, which influenced the numbers of garments in each size we ordered. \\

It would not be recommended to do a free t-shirt as the reward for Early Bird bookings in the future. This is due to the complex administration requirements, and the difficulties experienced with advertising the offer

\begin{figure}[ht]
    \centering
    \begin{tikzpicture}
        \pie[text=pin]{56/Booked before EBD,
        44/Booked after EBD
        }
    \end{tikzpicture}
    \caption{Attendees booked before / after Early Bird Deadline (EBD)}
    
\end{figure}