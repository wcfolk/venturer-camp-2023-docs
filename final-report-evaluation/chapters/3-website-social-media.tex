\chapter{Website \& Social Media}
\section{Website}
For historical Venturer Camps, the domain \texttt{venturercamp.org.uk} has been used. Having a separate domain provides a number of pros and cons, this was the cause of a lengthy discussion in Autumn 2022.\\

Ultimately, it was decided that the domain would be re-registered and a simple WordPress website would be created using a cheap shared hosting provider. Ralph Sleigh and Thomas Boxall met virtually to register the domain, choosing 123-reg as the domain registrar and Hostinger as the hosting provider. Ralph managed the domain registration and therefore configured the DNS records to ensure that the domain was linked correctly to the Woodcraft Folk Google Workspace.\\

Having our own website gave us the flexibility to publish whatever content we wanted and allowed us to configure the website exactly how we wanted to. The decision was taken to leave WordPress configuration as simple as possible, exclusively using off the shelf components as this would make our life as simple as possible! \\

The configuration of the website was for the most part left unchanged during the year which it was active for. Only two major changes were made: adding reCAPTCHA and changing the Home Page. \\
Adding reCAPTHCA to the contact form plugin we used (WP Forms) was done out of necessity. We were beginning to receive an unmanageable amount of spam to the WordPress comment queue and by adding ReCaptcha, we were able to reduce this to a more manageable amount. \\
Changing the homepage was done to be able to give out the most accurate information during camp - something which the initial homepage didn't allow for. \\

For the most part, the operation of the website was managed by Thomas, however a number of others also had access to edit and manage the content on the site. There had been plans to expand the number of editors on the website, to allow teams to publish their own news articles and content - however this never happened due to time limitations.

\section{Social Media}
After Venturer Camp 2019, there is an already-established Venturer Camp instagram account which we gained control of during October 2022. At the time, the Camp Coordinator was doing the comms role and as such began posting semi-regular content to the account. At the time, there was no plan used as there was little time to devise a plan.\\

While gaining access to social media accounts, Thomas also got into the Facebook account. Through this, he learnt that the 2019 camp used a profile, rather than a page which meant we wouldn't have had any statistical oversight from it. Thomas created a page and published a link to it across all social media platforms and we gained followers.\\

The Facebook and Instagram pages were both linked to the same Meta Business Account, which meant we had oversight and control over them both from one place.\\

Thomas took the decision to not use Twitter. A tweet was published informing followers of this fact and signposting them to where they could find out more.\\

Through October, November, December and January, Thomas took a lead on producing content for the social media platforms. Posts were sporadic and it was generally used as calls to action for volunteers rather than beginning to excite people for bookings to open. For future events, it would be massively beneficial to have a dedicated communications manager in the team whose responsibility it is to deal with pushing content to social media.\\

As pressures on the coordinator ramped up in different aspects of the camp, Thomas handed over the managing of the social media to the Events Assistant who managed the social media until late spring / early summer where Alex Baird, national Communications Volunteer,  took over the managing of the social media alongside Woodcraft Folk's Communications Manager. \\

The use of social media for an event like this has many different purposes, ranging from telling people bookings are open, raising people's excitement about the event and giving them all the critical information.\\

There were times that we were unable to create and push content out due to not having capacity. This was unfortunate as the event is catering for those who probably spend the highest amount of time out of any age group on social media.\\

Without the role of communications being filled on the team - we were in a situation where social media often fell to an afterthought. Whilst not ideal, we managed like this. There were enough people who had access to the social media accounts that something could be pushed out when absolutely necessary. For future events, having a dedicated communications team (for larger events) or person (for smaller events) is imperative. It is not something which should fall to the Coordinator to do. The coordination team should also have the space to share content through the social media when needed, another thing which we were unable to do due to the size of our team.
