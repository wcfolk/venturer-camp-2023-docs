\chapter{Food}
\section{Team Structure}
The Food team was composed of Morgan Britton-Voss, Wilf Lamont and Sadie Lamont. Additionally they had support from Eiriol Evans, Ro L-Jones and Jess Nicholls on site. The Food team was also heavily supported by Joe and Chris Bowler, who took on roles in the Cafe \& Special Diets Team.\\

Before camp, Morgan, Wilf and Sadie planned the menu and made decisions around meals; Morgan took the lead on contacting Suppliers. Eiriol also supported a lot with the spreadsheet work.\\

On-Camp, Morgan continued leading on communicating with Suppliers and managing additional orders with Chris; Eiriol managed the physical food distribution, supporting shopping and managed the stores; and Sadie and Wilf led on communications with KPs. \\

The team was supported by Joe and Chris, who took on a number of tasks to support the Central KP team, including lots of the small details. They had planned to mentor the younger KP team but due to time constraints and the amount of work they had to do - this wasn't possible.
\begin{figure}[h]
    \centering
    \includegraphics[width=0.8\textwidth]{assets/some-of-food-team-ic.JPG}
    \caption{Some of the Central Food Team (\textit{IC})}
\end{figure}
\subsection{Evaluation of Team Structure}
There were a number of challenges the Food team faced. On reflection, they felt that Eiriol should have been involved from earlier, as their initial role had been planned to be purely on-camp but ended up being more organisational. They also felt that the team would have benefitted from having clearer defined roles. They also felt the communication with the Village KPs could have been smoother, as it took a few days to get into the swing of it. It was also commented that one year is not enough time to organise a Venturer Camp in, and that the Food team benefits from a longer lead in time.\\

Joe and Chris noted that their involvement in the Core Food team was more than they had initially intended it to be during their Cafe \& Special Diets Evaluation Interview. They had initially thought that they would be mentoring the Food team, however they ended up being instrumental in the success of the food team. Joe and Chris spent lots of their free time plugging gaps where the Food Team hadn't completed actions or needed support. This included facilitating a number of meetings and pushing decisions out of the food team. This was, for many members, their first experience in a large Woodcraft Folk project like this - which highlights the need for better mentoring with young volunteers.  The food team felt that the understanding of the mentoring and guidance was understood differently to them and Chris and Joe and thought that there would be more frequent communications than there were.  This, in part, led to the food team falling behind on some elements.  It also seemed as if the team was intended to be one unit then became almost two sub-units then one unit again on the event. Suggestion of defined roles for all volunteers and task groups, that may be subject to change as the project evolves, made clear in an accessible file and available for all to see to be clear on what is expected of everyone and to provide clarity of responsibilities, authorities, and duties.
\subsection{Areas for Improvement}
For future camps, it's recommended to have more people trained on how to read the Food Spreadsheets, as it worked out that all the people who could read them also had other pressures on their time at Venturer Camp. The food team also felt the need for better-defined deadlines and to have more support with driving on site. The type of driving support they need to have available is also important to consider; where a top-up shop could fit into a car for a Venturer Camp sized event but would require a van for anything larger. \\

The team also felt that changing their team structure to include a Leadership role would be beneficial. They found that not having a defined leader led to some of their difficulties they experienced with workload management and time balancing, both in the run-up to camp and on site. They also felt that having an `internal food communications' person would help. This person would be responsible for talking to KPs, and is in effect a buffer between the central KPs and the village KPs. \\

The team felt that the co-working and relationships between the Central Food Team and the specialised Cafe \& Special Diets team drifted over the course from January 2023 to April 2023. There was some ambiguity as to where the boundary between the two teams' roles were. For future events, a better understanding and clearer definitions of team remits would be recommended, as well as ensuring that where there is overlap - documentation is clear as to decisions made and any contributing factors to these decisions. 
\section{Supporting Events}
\subsection{Pre-Camp}
The online Pre-Camp Food session had low attendance, however this wasn't an issue as all they needed to present was the menu and make people aware of the Special Diets team. Team members found it a useful opportunity to catch each other up to where they were.\\

A good proportion of the food team made it to On-Site Pre-Camp. One team member described it as ``the best pre-camp I've ever been to''. The Food team used the open space time to have a team meeting, where they made leaps and bounds of progress. Lots of time was also spent discussing cooling of food to ensure we meet food hygiene standards. It was said, after the event, that the food team should have had suitable commercial standard refrigeration and not rely on a domestic fridge and cool boxes.
\subsection{Working Week}
The majority of the food team attended Working Week. This was very useful for them as they were able to get stuck in with some of the other Working Week tasks for a day or so, enabling them to get to know the wider team. Then, once their main marquee arrived - they were able to support erecting that then begin setting up their space. They were also able to work out niggles in their plan, which meant we were able to rectify problems. This included giving them an additional marquee for food distribution. \\

The team was also able to travel to suppliers to meet them in person. This proved very useful when communicating with them

\section{On-Camp Operations}
\subsection{Daily Structure}
The food team had a very defined daily structure, which would begin with them opening up the Distribution Centre in the morning, keeping an eye out for people turning up in panic about shortages of Breakfast supplies. On some mornings, they would arrive to see a delivery from the bakers had arrived. This would arrive at around 07:30, which meant it was often the Coordinator coming onto shift who would receive this. The food team would then pick the chilled foods for the morning distribution, then give out the food at the morning meeting. They then took a short break for lunch before picking the dry food for the following day's distribution.\\

This was interspersed with receiving deliveries in the morning and early afternoon. The food team worked with suppliers to negotiate the best times for deliveries, which worked well as they were able to manage the incoming food better. \\

Quite often, a team of people would be sent off site to procure more food. This would take place after lunch, and take a number of people off-site for multiple hours. 
\subsection{Time Off}
The Food team were able to take some time off during the event. Each member got at least one day off, with some taking additional time off due to medical issues. On one day, multiple members of the team took time off which meant a substitute needed to be drafted in. Sourcing this substitute was fortunately easy as they were able to call on someone who had been part of the food team at Common Ground, so understood the systems and processes. 
\subsection{Support}
The Food team had a difficult experience trying to gain support from anyone. They found that the support from the Coordinator and Volunteer Support team was lacking; with the Coordinator too busy and stressed and the volunteer support space was inaccessible to many members of the team, due to distance and rock dust on the ground. They tried to visit the PEB tent in the evening, but were told they needed to leave due to noise concerns in the adjacent village. This was especially unfortunate to the team on the last night where due to the Main Marquee sleeping issues, Venturers were sleeping in the Koodoo bell tents - therefore volunteers were moved on from the Koodoo Marquee. Whilst this is understandable from the perspective of wanting young people to be able to sleep, it emphasises the need for future events to find a location for Volunteer Support which is available all the time to everyone, and away from campers if possible. 

\section{Food Team Specific Insights}
\subsection{Challenges On Site}
Due to the nature of internet ordering, and the lack of internet \& signal at Biblins - there were great difficulties with ordering food on the internet. This resulted in members of the team making trips into the local village to use a Pub's WiFi near enough every day to be able to order top-up foods. \\

Plans had been made to operate a Refills system whereby goods which the Villages would get through at a different rate would be made available to KPs to collect. These goods included milk, break, biscuits, cereals, sauces, etc. The food team had planned to log out the amount taken to gather some real-world data on this. However, due to time and capacity constraints, this was unable to happen.\\

Due to issues with suppliers not fulfilling the orders and unexpected needs arising, there was the need for daily off-site procurement trips. This was in part from attempting to use a domestic delivery service, Morrison's Home Delivery, for a commercial use where they don't have the capacity. The food team also experienced challenges in ordering \& predicting demand for certain products as usage varied by village and time of week.\\

The food team also experienced challenges when working out what food to serve. They received no guidance on specific food operations, for example how much meat to serve and the camp's stance on nuts. They expected this to be provided to them, despite the challenges it may have caused them. They went largely by Common Ground and DF proportions.\\

With regards to deliveries, the Food team had difficulties starting the camp on a Saturday. Have to start with three and a half days of fresh food and majority of ambient leading to the food arrivals being overwhelming, leading to less fresh milk due to capacity and no opportunity to do an emergency run on Sunday. Commercial suppliers usually only deliver on Mon-Fri.

\subsection{Tips for Future Central KPs}
When building KP teams in the future, it's important to ensure that there is significant prior KP experience. It also helps to have some people with logistics experience as quite a lot of the central KP role at a large camp is logistics. \\

It's recommended to anticipate forgotten tasks and work around this, building contingency plans. Planning in advance also should extend to the team members personal circumstances, especially relating to School Work and Exams - which will generally take over young people's lives, leaving less capacity for Woodcraft activities.\\

The food team also suffered from lack of capacity in the month leading up to camp, leaving a very small number of people the work of a very large group of people. It's recommended to ensure that the entire team is available in the month leading up to camp, as there's often things which will come up and need to be dealt with.\\

The team also recommends planning the menu early, and keeping it as simple as possible. Through having a menu available early, the Special Diets team are then able to work through booking data to ensure all diets are being catered for (to some extent as some allergens vary depending on which product is used e.g. falafel). Keeping recipes simple includes not having mange tout and green beans on the same menu, due to some people's difficulties in identifying vegetables.\\

It had been planned for the Food team to work with the Cafe \& Special Diets team to produce specialist training \& resources for Village KPs. It can be safely assumed that a village KP will not be as experienced as a central KP and as such communicating with them is a challenge. Through creating specialist training \& resources, the central KP team would be able to pass knowledge to the village KPs, upskilling volunteers who may be inspired to take on a central role at subsequent camps. \\

A major area for improvement for the Food team is requiring better Volunteer Support provision. By nature, the Food team is an extremely stressful environment due to its high emotional involvement in the camp's morale and success. The importance of supporting these volunteers who are not only physically doing a lot, but also carrying a massive mental burden should not be underestimated. A potential solution to this would be to have a dedicated member of the team who is responsible for the rest of the team's wellbeing, ensuring that they are fed, watered and have somewhere to sit down as this did not happen for the food team this time.\\

The food team not only needs to be able to purchase food, but also be able to purchase associated supplies. This includes things such as box cutters, weighing scales, bags, tape, markers, etc. For future events, a second budget line should be allocated to the food team for such consumables, rather than them having to compromise on food to be able to afford basic supplies. They should also be given the opportunity to order supplies through camp-wide stationary orders. 

\subsection{Menu}
{\RaggedRight \centering
\begin{longtable}{p{0.1\textwidth} p{0.17\textwidth} p{0.17\textwidth} p{0.17\textwidth} p{0.17\textwidth}}
\textbf{Day} & \textbf{Breakfast} & \textbf{Lunch} & \textbf{Dinner} & \textbf{Pudding} \\ 
\hline
\endhead

\multicolumn{5}{r}{\footnotesize\itshape continued on next page}\\
\endfoot 

\endlastfoot

Saturday 5 & \cellcolor{gray!25} & \cellcolor{gray!25} & Pesto Pasta & Coconut \& Cherry Traybake\\
\hline
Sunday 6 & Eggy Bread & Leek and potato soup & Korma curry and rice & Fruit and custard \\
\hline
Monday 7 & Porridge & Sandwiches& \cellcolor{red!25} Pasta with tomato sauce and meatballs & Banofee Pie\\
\hline
Tuesday 8 & Fried Breakfast & Falafel and wraps & Chilli and potato & Chealsea Buns \\
\hline
Wednesday 9 & Bircher Muesli & \cellcolor{red!25} Bagels & Lentil dahl; Cauliflower curry; and green bean curry with rice & Apples \& toasted oats\\
\hline
Thursday 10 & Garlic fried bread & Lentil soup & \cellcolor{red!25} Sausages and mash  & Strudel\\
\hline
Friday 11 & Porridge & \cellcolor{red!25} Hot dogs & Stir-fry with noodles & Fruit \& Custard\\
\hline
Saturday 12 & Leftovers & Leftovers & \cellcolor{gray!25} & \cellcolor{gray!25}\\
\hline

\caption{Menu at Venturer Camp 2023}
\end{longtable}
}% end of \RaggedRight

Feedback on the menu found that attendees were neutral towards it's taste, with a slight lean towards they liked it. It was found that the quantities of food were slightly lacking, however the reasoning behind this is unclear. 

\subsection{Menu Alteration Due To Gas Supply Issues}
Due to potential issues with Gas supply, a number of last minute changes had to be made to the menu. This caused a great deal of stress within the food team as it was too late in the process to be able to make any significant changes, however the changes made did contribute to the overall success of the camp.
\subsection{Food's route From Van To Mouth}
\begin{enumerate}
    \item Van arrives
    \item As food is taken off the van, food team sorts it into piles in the Main Food Distribution Tent
    \item The food is then `picked', where its taken from the Main tent in to the Giving out tent in village sized piles
    \item Village KP comes to the meeting and after they've listened to the meeting, they can collect food. The KP has to come to this meeting, no one else unless agreed prior with food team.
    \item KP cooks food and serves in Village
\end{enumerate}

\subsection{Food Safety}
Food safety is always a challenge at large Woodcraft Folk events due to the living-in-a-field nature of them. This being said, with the amount of people at Venturer Camp, we should have done better than we did. \\

To actively keep food cold, a single 300-ish litre Pantry Fridge was used. A number of cold boxes and broken fridges were used, filled with ice to passively cool food. This is unacceptable and should not be repeated in the future. The fridge, being a domestic fridge, isn't designed to have its entire contents changed every day and have the door opened for such long periods of time. It was decided to purchase this fridge and run the camp in this way to prevent the need for hiring a diesel generator and fridge trailer, for Carbon Emission and Cost reasons. For future events, we should acknowledge the carbon footprint of such an appliance and still hire it to ensure we are being food safe. We were fortunate enough to be able to situate all the cool storage in a smaller store tent adjacent to the Main Food Depot, in the shade. This combined with the cool weather experienced throughout the camp may have been the reason that we didn't have any major food safety issues throughout the camp. It's also important to note that there were not enough thermometers provided to the Food Team, alongside other equipment which they need. \\

Outside of the cold stores, generally food safety was okay. With the short-termed nature of the Food Distribution Centre's marquee, combined with the floor coverings and elevated storage used within it - no pest incidents were reported. There were, however, a few mice were spotted once the marquee had been taken down, which suggests that it takes mice about a week and a half to work out where food is stored. Mice were spotted getting into food within the Cafe Kitchen however. This was expected due to the Cafe Kitchen being the Old Camp Koodoo's standing kitchen. A car roof-box was used to prevent Venturers and Mice getting into stocks overnight.\\

Within villages, we have no sure data on this however it was felt that Food Safety was at an acceptable level. This excludes chilling food however as there was no provision for ice packs. Chilled foods were given out on the day it was due to be consumed, which reduced the time any food was out of the fridge. Meals with frozen ingredients were staggered so the frozen item could be used to help keep cool the cool items (e.g. frozen cauliflower to cool milk). There was a large number of wasps in villages. Village KPs were very good at controlling this, through minimising the amount of open sticky, sweet substances. \\

The Food team were able to make use of leftover online Food Safety \& Hygiene Courses purchased for Common Ground. They experienced difficulties in ensuring that all village KPs had completed relevant training. For future events, they suggest to require information of village KPs further in advance than we did as well as proof of certification.

\subsection{Food Team's Timeline}
\subsubsection{Autumn 2022}
\begin{itemize}
    \item Establish Team
    \item Team meeting includes role division
    \item Discussions around menu
\end{itemize}
\subsubsection{Winter 2022}
\begin{itemize}
    \item Menu set around December, pending minor change
    \item Recipes trialled at DF's Winter Wonderland and Spring Awakening
    \item Action log created, however not used consistently
\end{itemize}
\subsubsection{February / March / April 2023}
\begin{itemize}
    \item Search for Suppliers
    \item In person meeting
    \item Enhanced Team Communications \& Cross Team Communications with Cafe \& Special  Diets Team
    \item Further recipe tests conducted at DF's Spring Awakening
\end{itemize}
\subsubsection{May / June 2023}
\begin{itemize}
    \item Suppliers contacted
    \item Supplier accounts created by Woodcraft Folk's HoR
    \item In-Person Pre-Camp attended, team meeting held in Open Space time    
\end{itemize}
\subsubsection{July 2023}
\begin{itemize}
    \item In-Person team meeting, detailed planning conducted
    \item KP Handbook produced
    \item Spreadsheet of orders finalised, recommended to do this earlier and involve more people in it as it's a pain for an individual to do
    \item Orders placed
    \item Puddings planned (not done previously due to concerns over budget)
    \item Finalisation of products from suppliers \& re-distribution where necessary
    \item Gas supply issues led to menu changes
\end{itemize}

\subsection{Suppliers}
Suppliers were found through a combination of internet research and speaking to people familiar with the local supplier landscape for the Ross-on-Wye area. The suppliers used are as follows:
\begin{itemize}
    \item Evans of Monmouth Ltd - Incredible service, hard to return something that they have bought in specially but that is expected for a local supplier.
    \item Castell Howell - Mediocre, food team would rather have used a different supplier. Crisps were good
    \item Bookers - Good for 10 items (the basics like: beans, rice and biscuits), bad for rest.
    \item Neil Powell (butchers) - Communication informal, good value and quality. All meat was gluten free (sausages on request, the rest was naturally GF)
    \item Wigmores Of Monmouth (bakers) - Very good, returning the trays was a hassle due to KPs keeping the trays for a few days. They delivered very early.
\end{itemize}