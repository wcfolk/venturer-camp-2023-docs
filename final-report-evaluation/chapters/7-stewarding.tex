\chapter{Stewarding}
\section{Team Structure}
Before camp, the Head Stewarding team comprised of Paul Nolan-Paley and Adam Patterson. During Working Week, Ralph Sleigh also joined the team to ensure adequate experience with senior Stewarding was available on the team. Paul and Adam focused their pre-camp work on drafting a rota and supported with policy drafting, where this impacted Stewarding. \\

Woodcraft Folk's Chief Executive had been in contact with a number of Young People from Birkenhead, with the potential of them taking on Head Stewarding Shifts as they have lots of experience with roles like this. Due to this, we weren't focusing on recruiting others for this role. Communication with this group of people was extremely difficult as they were slow to respond through the Chief Executive to the Coordinator. Ultimately, their lack of timely responses and how unclear they were of if they would even be attending the event led us to recruit Paul, Adam and Ralph. These young people did arrive at the event, however due to unclear arrival dates, they didn't actually start Stewarding until well-into the event. \\

Some of the major Stewarding Events, especially arrivals day, had been planned by the Coordinator with support from previous Coordinators as Head Stewards weren't available for discussions.\\

On-camp, Paul, Adam and Ralph took it in turns on shift following a rota. The three mentioned above did eventually turn up and provided some respite support with this towards the end of camp. During the day, there would be one Head Steward on shift and sometimes there would be two. Having multiple head stewards on shift was especially useful in the evenings when we were running central programme.\\

The Head Stewarding team was supported by a team of Stewards who took on shifts throughout the event. These were recruited through a mixture of before-camp social media advertisement and on-camp having people turning up to Stewards HQ and wanting to sign up. Some of the Concordia volunteers' hours were assigned to Stewarding.

\subsection{Evaluation Of Team Structure}
This team structure worked as well as it could have given the difficulties in sourcing a Head Stewarding Team. The Head Stewards rota worked in a way such that there was no handover period, they have recommended that there needs to be a handover in the future, ensuring that things don't get dropped. \\

The team also noted that you can get by with 3 or 4 Stewards during the daytime, however in the evening you need 7 or 8. This was an issue at Venturer Camp as the Concordia Volunteers were rota'd in, however didn't show up. Then when someone went to look for them, they'd either be asleep or at the pub. When they did turn up, however, they needed direct supervision. It had been decided that a Concordia Volunteer needs to be paired with a Woodcraft Folk Volunteer aged over 18. This was also the case for those aged under 18, in that they needed to be paired with someone aged over 18. These decisions were made for Safeguarding reasons. We did get to a point during camp where lots of 16 and 17 year olds had volunteered to Steward and not enough over 18s, which caused a recruitment drive for people aged over 18 to steward. This was a complete pain for the Head Stewards as they found that people would turn up in pairs to steward; this led to some pairs of 16-17 year olds being stationed in the central area, near the Steward's HQ / Main Marquee, so they were able to steward however there were lots of adults supervising them.

\subsection{Support Ahead of Camp}
The team felt well supported by the Coordinator and Volunteer Support.

\section{Supporting Events}
\subsection{Pre-Camp}
The Stewarding Team didn't have an online pre-camp session.\\

No one from the Head Stewarding Team was able to make it to in-person Pre-Camp. This meant, for some of the team, that their first time seeing the site was during Working Week. This also meant that the Head Stewarding team missed conversations around arrival planning, which was held on site as part of the site walks.

\subsection{Working Week}
Some of the Head Stewarding Team attended Working Week. They used this time to have discussions around the rota and their operational plans. They also had a number of conversations with Owen Hayter, Common Ground 2022 Head Steward, who was attending Lewisham \& Greenwich's District Camp. \\

It's important for a Head Steward to be at Working Week for a day or two in order for them to have time to reacquaint themselves with the space and the entry \& exit points as well as the boundaries and any hazards on site. Something which didn't happen that it would have been good to do during Working Week was to do a Full Boundaries Walk with the Coordinator, Head Stewards and Site Manager present. This would have enabled the Stewards to be more prepared for different locations of the site, and ensure everyone was clear about where the boundaries were. \\

On the Final Night of Working Week (Friday 4 August), an Arrivals Briefing was held, which was aimed for anyone and everyone involved in the Arrivals Operation, the following afternoon. This was due to be delivered by the Camp Coordinator but due to the amount of work remaining, Jack Brown delivered this. For future events, it would be great if this can be handled by the Head Stewarding Team. 

\subsection{Takedown}
After the majority of campers have gone home, the Stewards role at large Woodcraft Events conclude. This means there is no obligation for Stewards to stay behind for Takedown. 

\section{On-Camp Operations}
\subsection{Daily Structure}
The daily structure for each head steward was slightly different, it was also different on each day depending on when they had shifts, time off, etc. Generally, their routine would involve paperwork and rounding up stewards. They didn't fuss too much if Stewards didn't arrive for morning shifts, as these were calmer. Focus was mostly on preparing for the evening and ensuring that Stewards would turn up for them. 

\subsection{Time Off}
All Head Stewards managed a day off, however these days weren't always restful as individuals had other responsibilities at camp. For example - one of Ralph's days off from Stewarding was the day of the AGM, where he has a role on the Standing Orders Committee which is heavily involved in the AGM; he also managed a day off for his Birthday, which he was able to completely get off site for. Others were given a day off after especially long days, meaning they didn't get to enjoy the event rather they were just recovering; some stewards preferred this as they needed less time to recover so were in fact able to enjoy the event. 

\subsection{Support}
The team felt well supported by the Coordinator and Volunteer Support. The team also commented that PEB wasn't accessible due to the distance from the central area and the villages which the Stewards were camping in. 

\section{Stewarding Specific Insights}
\subsection{Adult Social Space}
The PEB centre (Camp Koodoo's Marquee, located adjacent to Pitch 1), became a hangout zone for DFs in the evening as there hadn't been one designated. The stewards were happy with this and they added it to the rota, however for the future - it's important to also give adults a social space in the evening.

\subsection{Food}
The Stewarding team found that mischief wasn't caused over meal times, as everyone on site was eating at the same time. Due to this, they were able to take a break and eat at reasonable times every meal time. Sometimes they had their food brought to them in their Stewards HQ, and other times they trekked back to their village to collect it. \\

The Steward HQ was stocked with snacks, which were replenished throughout camp. However, they struggled to get them replenished at times.

\subsection{Successes}
There were no huge issues! The team reported that in the event of a missing person search being initiated, the individual was always found within 5 minutes, commonly in their village. \\

The Stewarding team also dynamically responded extremely quickly when the Coordination team decided that we would have a Fire Show on the final night with only a few hours notice. They were able to cordon off a safety boundary and had the right capacity to support keeping campers behind this.

\subsection{Miscellaneous Insights}
It was found that those who had signed up for an Under 18 Volunteer Priced Place with Stewarding as their role didn't do as much work as those with other roles. They could have probably been worked a bit harder, however due to their age - scheduling them presented more of a challenge. \\

There were scheduling conflicts with those who also took on MEST-UP duties. For the future, this could be managed better by developing the rota further in advance.\\

It's also useful to have more printed stuff in advance. This is documents such as the rota template, sign out sheets, etc.

\subsection{Tips For The Future}
This role is not about pleasing everyone. This role is about keeping people on site safe, and mostly happy. The role should not be taken on by people who don't handle stress well.\\

Stewarding Training and First Aid Training for Stewards should be run at Pre-Camp and / or Working Week. It was decided that due to the large investment into training at Common Ground International Camp 2022, we wouldn't invest beyond the legal essentials, which meant we couldn't afford either First Aid or Stewarding training.\\

When Volunteering Programmes, such as Concordia, are designed - their schedules should align to the Stewarding Timetable. Not the other way around. We didn't have a Stewarding Timetable at the time of sorting the timetable for the Concordia's which influenced the way we did this.\\

There needs to be a hard limit on how long an individual can steward for in a single day and across the event. This is to ensure that individuals are in a fit state to steward where they are not causing danger to themselves or the campers. Head Stewards also need to feel empowered to dismiss a Steward mid-shift should the need arise, and how to escalate this process to the Coordinator if needed. \\

It's also recommended to identify individuals who will be Stewarding in advance of camp. When a new Steward joins the team, they need to be given a briefing. A rule also needs to be included in the Code of Conduct that a leader needs to stay awake until their Venturers go to bed.
