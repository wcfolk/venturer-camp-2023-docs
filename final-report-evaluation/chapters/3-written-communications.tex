\chapter{Written Communications}
\section{Mass Emails}
Throughout the coordination of an event like this, communication with the membership is a vital process. Within Woodcraft Folk - the main way in which this is done is through mass emails pushed out through Constant Contact (mass mailing platform).\\

No one on the Venturer Camp team had direct access to Constant Contact, this caused a number of problems as we were unable to send the mass emails ourselves. Whilst understandable from a GDPR standpoint, it caused some difficulties. The main difficulty was the inability to send an email on-the-fly. Emails would have to be booked in with the member of staff sending the email at least a week in advance of needing to send them with the content deadline often at least multiple days before the target email send date - to give the staff time to schedule the email to be sent. There were a number of times where these tight deadlines meant that the emails didn't contain all the information in them that was originally intended due to not having enough time to write them. \\

Towards the end of the project, when time pressures were higher, Gmail's Multi Send Mode was utilised as this meant that emails could be sent to multiple recipients without using BCC (as this often results in emails going directly to people's spam box) while maintaining GDPR compliance. This proved extremely effective as the Coordinator had control over exactly when emails were sent and the content of them therefore reducing the dependence on others to send emails therefore reducing the lead time to send a mass email.\\

Another issue encountered using Constant Contact was the mailing list inconsistencies. Initially, we were using a ``All Venturer Group Leads'' list which had been pulled from Groop in the Autumn of 2022 however towards the end of the project, it was more relevant to email just the booking contacts. This data was only accessible within the booking system which therefore meant, a new mailing list would have to be created each time a mass email was to be sent within Constant Contact to ensure that any new booking contacts or additional booking contacts were included. Once the final booking deadline had passed, this was less of an issue however it still presented edge-case challenges where extra individuals would book on and we would then need to ensure they too received the mass emails.\\

As we didn't have capacity to, we didn't think about tailoring emails to the two different groups of people who booked on: group bookings and individual bookings. There was quite often confusion amongst the individual booking contacts where their entire contact with Woodcraft before had been as a parent of a child attending a weekly group night. This added some additional administrative burden to respond to the individuals when they questioned what the email meant to them. In the future - it may be useful to tailor the mass-emails to the two groups.

\section{Big Month Updates}
Towards the later months in the project, we began to push out `Big Month Updates'. These were one-stop-shop updates which contained all the relevant content either for the upcoming month or about the month-just-gone. It was unfortunate that these updates were only started towards the end of the project as it was felt that they were useful.\\

A newsletter-kind of update proved useful to the coordination team as it was a regular chance to push updates to the wider movement. Shorter updates would still need to be pushed during the times between the larger updates as sometimes there were news items too critical to wait. \\

For future large Woodcraft camps, it would be suggested to review and look to keep going with larger newsletter style emails \& blog posts on the website. Through publishing to the website, you also make the content available to those not in direct receipt of the email. It would also be advisable to share the highlights through social media, ensuring you reach as many people as possible.

\section{Information Pack}
Another staple within Woodcraft Folk camps are the Info Packs. For Venturer Camp 2023, we produced 3 (v0, v1 and v2). \\

The info packs were timed to be released at strategic points in the academic calendar, providing group leaders the information they needed at the right time. It was felt that less questions came through around topics covered in the info packs than would be expected, suggesting that leaders found them a useful resource.\\

Info Packs were co-authored by the entire Venturer Camp team, with the Coordinator and Events Assistant taking a lead. Once the content was complete, the coordinator typeset the packs using \LaTeX. The proof copies of the documents were then shared to the team for further contributions / comments, and once amended - the document would be published to the website. 

\subsection{Info Pack v0}
Info Pack v0 was published in December 2022 and aimed to provide a single location where the initial information about the event could be found. The document included:
\begin{itemize}
    \item An introduction from Thomas \& introductions to the different teams
    \item Information around the bookings process
    \item Information surrounding the cost of camp, suggestions for local group fundraising \& the access fund
    \item How to get involved
    \item Early information around food at camp, focusing on the differences to previous camps
    \item Information on the Sustainability of Venturer Camp 2023 \& its environmental impact
    \item The dates for on-site pre-camp
\end{itemize}

\subsection{Info Pack v1}
Info Pack v1 was published in early May 2023. There had been plans to release a few weeks earlier however due to delays in some content, the publication of the pack was delayed. \\

Info Pack v1 contained much more information about the event, still with some unclear details however. Much of the pack was dedicated to logistics surrounding equipment, communications, programme and food at the event. The table of contents included:
\begin{itemize}
    \item Pre-Camp details, both virtual and on-site
    \item Travel to site logistics
    \item Communications during the event, including emergency contact details
    \item Equipment
    \item Food
    \item Programme
    \item Introduction to the International Volunteers
    \item Decarbonisation
    \item Volunteer Wellbeing
    \item Safeguarding \& Risk Management
    \item Site Safety
\end{itemize}

\subsection{Info Pack v2}
Info Pack v2 was published in late June. This was the last major update group leaders got before the Village Handbook was released in July. the table of contents included:
\begin{itemize}
    \item Repeat of much of the info pack v2 content, with some details further fleshed out
    \item Site Layout
    \item Village composition
    \item Further detail on programme offerings
    \item Greater details on Safeguarding \& Risk Management
    \item Price list for the Cafe and Merch stand
    \item  Remaining roles to be filled
\end{itemize}

\section{Village Handbook}
For Venturer Camp 2023, we wanted to go back to the historical ways of doing things - publishing the Village Handbook well in advance of the camp, ensuring those who needed it had the information, before they got to site. This methodology worked, and while there were a few minor amendments made on site - the main village handbook document was published two weeks before the start of camp.
\subsection{Village Handbook Document}
The Village Handbook document was co-authored by many different members of the team, with the Coordinator bringing the sections together. The VH document followed the same workflow as the Info Packs, including typesetting.\\

Much of the handbook was dedicated to on-site logistics and sharing the details which are needed to ensure that everything works smoothly. There was also some information about the site, features of it, and how emergencies are handled as well as the consultation activities we were asking villages to run.\\

Many people commented on how comprehensive the document was, and were thankful for that fact. The aim of producing a longer document was to ease minds and ensure they had all the information they would need, which we achieved!\\

The full Village Handbook document can be found in the appendices.

\subsection{Village Handbook Folder}
Due to the nature of Biblins having very limited cellular connectivity, a decision was made to give each village a ring binder stuffed with information which Village Coordinators \& Village Volunteers would find useful. This worked well, however there was a lot of printing and folder stuffing. \\

Printed lists of members of each village was also provided to Village Coordinators as this ensured that they had the data to hand should it be required. The Camp Office also held a printed copy of the booking data, broken down in the same units as the Village copies, for quick reference.
\begin{itemize}
    \item Camp Map
    \item Village Handbook Document
    \item Safeguarding Documents
    \begin{itemize}
        \item Safeguarding Responsibilities \& Support
        \item Woodcraft Folk's safeguarding policy
        \item Venturer Camp 2023 Risk Register
        \item Missing Young Person Procedures
        \item Incident \& Disclosure Form
        \item First Aid Forms
    \end{itemize}
    \item Village Members lists
    \begin{itemize}
        \item Attendance list
        \item Consents list
        \item Dietary Data
        \item Medical List
        \item Central Role Holders list
    \end{itemize}
    \item Programme
    \begin{itemize}
        \item Daytime \& Evening programme itinerary
        \item Grab `n' Go Activity Pack
        \item Adventurous Activities Info Sheet
    \end{itemize}
    \item Consultation Activities
    \begin{itemize}
        \item Heading to 100 Session Plan
        \item Strategic Plan Session Plan
        \item Camp Evaluation Activity Session Plan
        \item EDI Exploration \& Discussion Session Plan
    \end{itemize} 
\end{itemize}

This was obviously a lot of paper to print and then dispose of after the event. The decision to do it this way was taken to ensure that Village Coordinators felt they had all the information they needed, without having to continually ask the same questions at the camp office. This worked, with very few village coordinators asking lots of questions at the camp office. For camps where there is phone signal, it would be suggested that the village handbook `folder' is provided as a web page with downloadable documents linked, then either a QR code or URL is provided in the main document signposting people to this.