\chapter{Volunteer Support}
\section{Team Structure}
Initially, the workload within this team had been divided between Sapna Argwal, focusing on pre-camp support and Mollie Saunders focusing on on-site support. In the end, Mollie was unable to make it to Venturer Camp; Sapna was fortunately able to run the Volunteer Support centre on site. Sapna was supported by a number of other volunteers, including Jeni Dixon and Margaret Fleming. 

\subsection{Support}
Sapna felt underprepared in the lead-up to the event, this was due to the lack of communication and support from the Coordinator, and lack of knowledge of how Venturer Camps work having never attended one before. She felt unsure about how often to contact the volunteers teams, as to not burden them as she was supposed to be supporting them!

\subsection{Team Redesign}
The Volunteer Support team suggests that the current model of having a small team of volunteers supporting the entire organisation's volunteers isn't fit for purpose. They suggest that regions should be responsible for their own volunteers. Through this idea, an individual or small group of volunteers from each region would support all volunteers in that region. The Scottish Nation operate like this currently, they know what's going on in each of the districts and are able to support each other without breaching privacy. Future events should take note of this idea in particular, as it may be time to resurrect the regions!

\section{Supporting Events}
\subsection{Pre-Camp}
The Volunteer Support Online Pre-Camp session was poorly attended. It's probably not worth doing a whole session dedicated to this in the future, rather tack it onto another session where attendance might be higher.\\

The Volunteer Support team wasn't able to make it to On-Site Pre-Camp due to the distance from their home to Biblins. They think that it would be useful for the team to be represented at Pre-Camps as it's a good opportunity to get to know the team and to have conversations which can influence future support. It's also a useful time to get to know the site.\\

The Volunteer support team noted that they wished they had got to know people better, so when they were in crisis or struggling, it would have been easier to have conversations with them.

\subsection{Working Week}
The Volunteer Support Team didn't attend Working Week due to travel arrangements. They were able to set up their space, PEB, on the first day of camp. For future events, it would be nice to make the PEB space more comfortable by having soft furnishing or sofas, etc. This would make a good Working Week task for someone with a van.

\section{On-Camp Operations}
\subsection{Daily Structure}
The team's daily routine involved moving between Elysium, where they were camping, to the Volunteer Support marquee (located adjacent to Asgard). On the walk along the entire length of the site, they would stop and talk to volunteers in all the villages they passed and the central area as well as stopping at the Camp Office to chat to the people on the balcony. 
\subsection{Time Off}
Sapna held a number of roles at Venturer Camp and managed to take a day off from each role, where she took on a different role for the day. 
\subsection{Support}
The Volunteer Support team felt supported by the Coordinator. 

\section{Volunteer Support Specific Insights}
\subsection{Positive Energy Bubble Tent}
The Positive Energy Bubble (PEB), was the home of Volunteer Support. It was situated in the Camp Koodoo main marquee. As well as offering support to Volunteers it was also planned to use the space to host a number of workshops aimed at volunteers around EDI (Equality, Diversity and Inclusion). These were cancelled during the camp as it was decided that the PEB tent wasn't the right place to host these workshops.\\

The use of the Camp Koodoo Marquee was the wrong choice for Volunteer Support as the marquee is sited on a hard standing which requires the use of steps to get up to it. \\

During the evenings, it was found that DFs used the PEB tent as a social space. This was approved by the Volunteer Support team as it was felt that they should have somewhere to go and socialise without feeling like they're on duty. They did need to be cleaned up after, however. This returns to the point made earlier of needing social spaces for volunteers as well as participants at large Woodcraft Camps.

\subsection{Timeline}
In the lead-up to Camp, the Volunteer Support team checked in with each of the teams every three weeks, as camp approached. Before this - teams were checked in with every 4-to-6 weeks. This seemed to work well. Most teams ignored the check-in messages or said that they were fine. Occasionally, the Volunteer Support team was able to offer support. After camp, the Volunteer Support team was reached out to by the MEST-UP team who had a challenging experience at camp. Post-Camp support would be worth exploring for future events.\\

The team noted that they needed better communication channels with adult volunteers. They found that the provided methods, Discord and Emails, weren't sufficient. This could be rectified through giving the Volunteer Support team access to other volunteers phone numbers, or enforcing a meeting structure between the Volunteer Support team and other volunteers, where a verbal check-in is had.

\subsection{Tips for the Future}
The team emphasises the need to think about stress points and how to alleviate them. This is something all teams experience throughout the planning process - so having a small group of volunteers at the stress points to offer practical and emotional support. It's encouraged to plan and get to know team members in advance of them hitting breaking point.\\

Future camps should strive to create a culture of people asking for help and receiving said help.

\subsection{General Areas for Improvements across the Camp}
Across the camp, it's felt that a quick win could be to look at straightforward, low-impact environmental practices at the village level. For example, hay boxes and solar showers. It was felt that the access fund for travel and attendance, and to have a reduced rate for volunteers was beneficial and should be repeated.\\

On the chill nights, it's suggested to play music in the central marquee, even when there's no organised activity or performances. 	This would need careful planning \& branding to ensure that it didn't turn into a rave every night however.