\chapter{On Camp Working}
\section{Volunteer Support}
Volunteer Support came in the form of the Positive Energy Bubble (PEB). PEB was located in the Camp Koodoo (adjacent to pitch 1a at the far-western end of the site) Marquee. This raised a number of concerns due to its location, being a 5 to 10 minute walk from the central area, and its size (a large marquee). Some volunteers were unable to utilise PEB due to the distance to get there and the requirement to climb steps to enter the marquee. \\

It was suggested that PEB should be centrally located, with the potential of it being used as a social space for adult volunteers in the evening. Volunteers also requested clearer communications about PEB's purpose and goals, which would enhance its effectiveness. \\

Volunteer Support would also have benefitted from additional capacity. There was one main leader of Volunteer Support who also had brought young people with them. They were supported by a number of other volunteers who ran small workshops. The lead Volunteer Support volunteer remarked that the trek from their Village (Elysium, pitch 9) to the PEB tent proved the most useful part of their day as they could stop and talk to people. 

\subsection{Support From The Coordinator}
The overall level of support from the coordinator was varied, with some teams feeling well-supported while others had limited engagement. Teams suggested that a consistent and proactive approach to support is necessary. \\

The Coordinator \& Events Assistant had hoped to split the teams between them, being a point of contact for the teams questions, concerns or ranting. However, due to the amount of additional work the Coordinator \& Events Assistant absorbed, they were unable to do this. Unfortunately, there was little capacity to support teams.

\section{Time Off}
A major focus for Venturer Camp 2023 was to ensure that all volunteers, whether they be a group leader or the coordinator, got some time off. This goal was built from feedback from Common Ground where volunteers didn't find that they had any downtime, leading to a large amount of burnout and exhaustion at the end of the camp. While we didn't get it right, as we still had a situation where the majority of a single team had to walk away for an afternoon on the same day, we definitely did better.\\

Many of the central volunteers managed to take some time off, with the majority opting for a day off. This took some planning and in some cases - drafting in of additional resources to plug gaps. However, the collective feedback indicates that having dedicated days off is essential for volunteer's wellbeing. Days off should be planned in advance of camp, not decided ad hoc, to ensure that adequate cover can be planned. It may be beneficial for teams to over-recruit volunteers to ensure that they can all take a day off without over-burdening others in the team. \\

Many volunteers found that socialising in the evening was a good way to relax. Some volunteers went off-site to the local village, Symonds Yat, before returning to site and looking for somewhere to socialise. When planning the camp, it hadn't been taken into consideration about the lack of spaces where volunteers could socialise. This lead to a number of instances where we had to ask Volunteers to move on and find somewhere else as they were disturbing sleeping Venturers. For future events, where there is not an obvious volunteer socialisation space, one needs to be provided. This may be the main marquee after the evening programme has concluded, or it may be the Volunteer Support space. 

\section{Feeding The Central Team}
Due to incidents at previous large Woodcraft Folk camps where central team members not camping as part of the Central Village didn't get food saved for them at meal times, we were very keen to improve on this. This was especially the case, given that we were not having a Central Village and having all the central team members camping in standard villages.\\

As part of the village handbook, KPs were given a list of people who needed food to be held back for them. This list was compiled by the Coordinator, ensuring that only the essential people were on this list as we recognised the large ask on KPs. Broadly speaking, the KPs were happy to oblige with this, understanding the need for it. \\

For future events - consideration should be given to those with allergies and how this is handled in relation to saving food. We had an unfortunate incident where a village had not saved food for one member of the core team, who had an allergic reaction as a result of cross contamination in the kitchen. \\

A number of members of the central team found that they needed more snacks provided. There were a few teams who had access to snacks, however there were teams who did not.

\section{Meetings}
Due to a shortage of time and capacity on site to organise a meeting, there was not a team meeting. Nearly all teams commented that having something like this would have been really nice and a good way to share experiences and increase team unity. It would have also been a time where teams could check in with each other, to ensure they all had enough capacity to get through camp.

\section{Shopping Trips}
Many teams had to take numerous trips off site to purchase additional resources. Some of these trips are unavoidable due to difficulties with deliveries, such as food deliveries, however some would've been avoidable with some slightly better planning. We tried to order as much as possible from online suppliers during camp, however due to Biblins being so isolated, deliveries took a long time to arrive. \\

Throughout the camp, there were a number of ideas raised which could've saved the number of individual trips off site, however none of these came through. The primary idea which would be great to see replicated in the future, is a single person going off site to do all the shopping. 
