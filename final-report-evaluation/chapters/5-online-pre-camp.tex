\chapter{Online Pre-Camp}
After seeing how Common Ground, Woodcraft Folk international camp held in 2022, ran their Pre-Camp sessions - it was decided to emulate this with a heavier emphasis on the online sessions. The online sessions took place from w/c 17 April to w/c 26 June with at least one session each week. Each session focused on a different component of the camp, ranging from a virtual site tour to a session where the attendees could meet the Site Services team.\\

Some teams found the online pre-camp sessions helpful, however on the whole - it was felt that they were stressful and time consuming. This point is further emphasised when reviewing the audience for each session. Most sessions only were attended by 2 to 4 attendees, with some gaining up to 20 additional views on YouTube after the fact. Some sessions, where recordings were actually taken, were published to the Woodcraft Folk YouTube channel.\\

No feedback was gathered beyond the core team, so we don't know the impact of Online Pre-Camp on the attendees.\\

On the whole, it would not be recommended to repeat Online Pre-Camp again in the same way in the future. However, taking learnings from this: it would be suggested that teams are given the opportunity to present an `update' which is recorded and published in some capacity, giving camp attendees information; with some teams still delivering a live session. The Food and Cafe \& Special Diets sessions were the best attended single-team sessions, it would be highly recommended to run similar sessions again. The General Q\&A \& broader sessions were also well attended, as would be expected - however many participants came along just to hear what was going on rather than ask specific questions, a testament to the team's excellent communications! The best-attended session was the Meet Your Village Evening, see below for further details around this. 

\section{How the Sessions Worked}
The majority of Online Pre-Camp sessions took a similar form:
\begin{itemize}
    \item Introductions to people hosting the session
    \item A short presentation, providing updates on that topic \& plugging future sessions
    \item Space for discussions and / or Q\&A
\end{itemize}

During each session, it was expected that notes were taken. The content of these notes was at the discretion of each session's facilitation team. For future sessions, it would be helpful to provide the facilitation team with a template they can insert their notes into rather than expect them to work out what to take notes on; thus ensuring what notes are returned to the coordinator after the session.\\

The sessions were promoted through our website. On the website, an events plugin enabled us to have a calendar which displayed upcoming events. An event was created for each Online Pre-Camp session, which gave us a space to link the Zoom Meeting from \& share any details about the session. A main Pre-Camp page also displayed information about Pre-Camp, should the movement want more information. Each week, a post was made to social media which detailed the upcoming sessions in that week - these were usually published in the first half of the week.\\

After each session, the event on the website would be updated to contain a link to the recording published to YouTube, any slides used, and any other relevant resources. 

\section{Sessions}
\begin{itemize}
    \item Welcome To Pre-Camp
    \item Virtual Site Tour
    \item Code of Conduct I (Day Scheduling and Leaving Site)
    \item Meet the Coordinator
    \item Booking Help
    \item Code of Conduct II (Respect, Clans \& Co-operation)
    \item Meet the Cafe / Allergy Teams
    \item Meet the Programme Team
    \item Meet the Food Team
    \item Meet the Site Services Team
    \item Meet Your Village Evening
    \item Code of Conduct III (Consent and Intoxicating Substances)
    \item Meet the Volunteer Support Team
    \item General Q\&A
\end{itemize}

\section{Meet Your Village Evening}
The biggest and most exciting session of the Online Pre-Camp series was by far the Meet Your Village Evening. This session was designed to announce the village allocations \& ensure village adults made contact with each other, a sort of first Village Meeting.\\ 

The session was regarded as a success - with all-bar-two groups booked present at the session, and some villages felt that their conversations had been so productive during the evening that they didn't require further meetings. From a coordination perspective, it was reassuring to know that key village adults had at least agreed on roles.
A follow up email was sent a few days after the evening, re-introducing those who had been at the session and introducing those who hadn't made it. This also shared links to the relevant forms which volunteers would need to fill in to say that they were taking on a role in the village or to request access to the Booking System. 

\subsection{Running Order}
Participants were given an overview of how villages would work and what would happen at camp. We also reinforced the aim of bringing everyone together and that this is our first Venturer Camp since before the pandemic. After a brief introduction and look at the map we showed the participants which groups would be in which villages and where villages would be on the site.\\

We spent some time talking about feeding central volunteers and explained which volunteers would be allocated where.\\

We then broke off into breakout rooms for each village, with a facilitator in each
\begin{enumerate}
    \item Introductions (names, pronouns, group \& role in group, all time favourite camp meal)
    \item Talk about equipment and who was supplying this 
    \item A space for the central volunteers to make themselves known
    \item Village Roles
    \begin{itemize}
        \item We said we needed names for: village coordinator, village KP (preferable more than one per village), Safeguarding lead, First Aid lead. Names don't have to be given tonight but note down ideas from people of who we could chase
        \item Optional extras (don't need names or need to dwell on this tonight): lost property keeper, KE/G (depends on where you're from), Fire, Programme, birthdays
    \end{itemize}
\end{enumerate}

We shared the form for village roles and said that these needed to be filled in as soon as possible. \\

Jeni, one of the people who ran this session, said it was Incredibly successful. We only had 2 districts coming to camp not turn up or send apologies
We need to have this session scheduled and in people's diaries as early as possible as it's really useful but more so if more people are there
