\chapter{Camp Census}
A Camp Census was carried out on Tuesday 8 August 2023. This census was optional for all attendees on site to complete; it was completed through attendees filling out a form printed which was then later digitised and compiled to produce results shown below. Of the 450ish people we had on site at the time of the Census, we received 334 responses, 51\% of those were from young people aged 13-15 years.

\begin{itemize}
    \item 45.3\% described their gender as female
    \item 39.5\% described their gender as male
    \item 7.2\% described their gender as non-binary
    \item 12.3\% described their gender being different from the one assigned at birth
    \item 15.2\% described themselves as having a disability
    \item 19.6\% described themselves as having a health condition which impacted on their lives
    \item 29.1\% described themselves as neuro-diverse
    \item 16.3\% said that they were disadvantaged due to their mental health
    \item 3.6\% of campers had experienced living in care
    \item 8.4\% described themselves as living in a low income household
\end{itemize}

\section{Summary}
\begin{itemize}
    \item Underrepresentation of people of colour at camp - 87.7\% white including British (73.6\%), Irish (3.3\%), Gypsy/Traveller (1.2\%) and other white (9.6\%). 2021 Census data suggests 82\% of people in England and Wales are white, and 18\% belong to a black, Asian, mixed or other ethnic group.
    \item Greater ethnic and gender diversity amongst our participants when compared to the volunteer team
    \item Greater reluctance to respond to questions around gender and mental health
    \item Woodcraft Folk has much higher levels of engagement by neurodiverse individuals, both as participants and volunteers (neurodiversity estimated to be 1 in 7 in the workplace 2016 study)
\end{itemize}

\section{Participants}
This section looks at responses from between 231 and 235 individuals
\begin{itemize}
    \item 7.2\% would describe themselves as non-binary with a further 4.3\% wishing to self-define their gender
    \item 68.4\% would describe themselves as white British, 11.5\% white other, 3.8\% Irish, 1.3\% Gypsy or traveller, 1.3\% Black or British Black, 5.1\% Asian or British Asian, 7.7\% mixed or multiple ethnic groups and 0.9\% other
    \item 16.2\% would describe themselves as having a health condition which impacts their health
    \item 13.4\% would described themselves as having a disability
    \item 26.5\% would describe themselves as neurodiverse
    \item 15.5\% would describe themselves as experiencing disadvantage due to their mental health
    \item 9.8\% would described themselves as having caring responsibilities
    \item 4.7\% would describe themselves as having lived in care experience
\end{itemize}

\section{DFs}
This section looks at between 26 and 27 responses from individuals aged between 18 and 21.
\begin{itemize}
    \item 11\% would describe themselves as non-binary with a further 7.4\% wishing to self define their gender
    81.5\% would describe themselves as white British, 11.1\% mixed or multiple ethnic groups with a further 7.4\% as white other
    22.2\% would describe themselves as having a health condition which impacts their health
    19.2\% would described themselves as having a disability
    40.7\% would describe themselves as neurodiverse
    23.1\% would describe themselves as experiencing disadvantage due to their mental health
\end{itemize}

\section{Kinsfolk}
This section looks at responses from Kinsfolk, aged 21 and above. There were approximately 71 to 72 responses.
\begin{itemize}
    \item 5.6\% would describe themselves as non-binary
    \item 87.5\% would describe themselves as white British, 2.8\% Irish, 1.4\% Gypsy or traveller, 1.4\% as Asian or Asian British, 1.4\% other ethnic group, 1.4\% mixed British Caribbean with a further 4.2\% as white other
    \item 29.2\% would describe themselves as having a health condition which impacts their health
    \item 19.4\% would described themselves as having a disability
    \item 33.3\% would describe themselves as neurodiverse
    \item 16.7\% would describe themselves as experiencing disadvantage due to their mental health
\end{itemize}

\section{Overall Observations}
It is interesting to compare the camp census responses to wider demographic monitoring across the organisation. For example, when reviewing demographic data shared by Venturer Camp volunteers against that submitted in an open call to monitor demographics amongst the volunteer team there are significant differences:
\begin{itemize}
    \item There is greater gender diversity amongst VCamp volunteers, and a higher representation of male volunteers, when compared to the wider volunteer network: 43.7\% male, 46.7\% female, 5.6\% non-binary compared to 28.4\% male, 67.9\% female and 2.6\% non-binary. 9.9\% of Venturer Camp volunteers suggested that their gender was different to that assigned at birth, compared to only 4.3\% of the wider volunteering membership
    \item Venturer Camp volunteers were also more likely to suggest that they had a disability 19.4\% compared to 12.7\% of the wider volunteering membership
    \item The ethnicity of the wider volunteering membership is slightly broader than that of the Venturer Camp volunteering team - 6.3\% of the whole volunteer membership described their ethnicity as something other than white compared to only 4.2\% of Venturer Camp volunteers. Both demonstrate a huge underrepresentation of people of colour within the organisation.
    \item Venturer Camp volunteers more likely to report a health condition - 29.2\% compared to 24.3\%
    \item Venturer Camp volunteers more likely to share that they have a disability - 19.4\% compared to 12.7\%
    \item Venturer Camp volunteers significantly more likely to share that they are neurodiverse - 33.3\% compared to 17.8\% of the wider volunteer team.
\end{itemize}
The difference shared above requires further investigation, the sample sizes are small and the response rates significantly different - only 9\% of volunteers responded to the demographic survey across the whole organisation, but 0.5\% equates to one respondent whereas 1.4\% equates to one respondent of the census completed at Venturer Camp. \\

The camp census also included an open question asking for suggestions of how Woodcraft Folk could be more inclusive. The majority of responses related to local group activities and the need for more marketing and promotion. Below are those statements which reference camp experiences, as provided by the young participants:
\begin{itemize}
    \item The need to make merchandise cheaper, as not all participants can afford to purchase
    \item Longer quiet hours - start at 10:30pm
    \item More drop-in workshops
    \item Cheaper camp cafe food
    \item More circle activities/team building activities
    \item Buy new tents
    \item Autism tent
    \item Clean showers more often
    \item More village activities
    \item Printed list of activities by day
    \item Mobile phone signal
    \item No camp currency
    \item Camp cafe is exclusive - whilst cheap is still making a profit and preventing some Venturers engaging due to financial barriers
    \item Clans should be made up of people known to the clan group - more comfortable working with friends
    \item Gender neutral toilets
\end{itemize}

Much of the comments made by DFs or volunteers focused on the wider organisation and the need to work in partnership with other organisations to increase and widen participation. All comments from the census have been shared with Woodcraft Folk's Equality, Diversity \& Inclusion Working Group.\\

There were also several comments on the food, unfortunately not all of the following comments are descriptive enough to inform future plans:
\begin{itemize}
    \item Better food
    \item More snacks
    \item Nicer food
    \item More food
    \item More diverse food
    \item Not enough meat on camp
    \item Make food better
    \item Vegan and non-vegan/vegetarian options being of equal value
\end{itemize}

\section{Some Pretty Graphs}
The below graphs use data from all respondents to the camp census. All questions had a Prefer Not To Say (PNTS) option. 
\begin{figure}[H]
    \begin{minipage}{0.4\textwidth}
        \centering
        \begin{tikzpicture}
            \pie[text=pin, radius=1.5]{45.2/Female,
            39.5/Male,
            7.2/Non-Binary,
            3.6/PNTS,
            4.5/Other
            }
        \end{tikzpicture}
        \caption{What Is Your Identified Gender?}
    \end{minipage}\hfill
    \begin{minipage}{0.4\textwidth}
        \centering
        \begin{tikzpicture}
            \pie[text=pin, radius=1.5]{6.6/PNTS,
            81/Yes,
            12.3/No
            }
        \end{tikzpicture}
        \caption{Is Your Identified Gender Same as Birth Gender?}
    \end{minipage}\hfill
\end{figure}

\begin{figure}[H]
    \begin{minipage}{0.4\textwidth}
        \centering
        \begin{tikzpicture}
            \pie[text=pin, radius=1.5]{61.2/No,
            29.1/Yes,
            7.2/Non-Binary,
            9.9/PNTS
            }
        \end{tikzpicture}
        \caption{Do You Consider Yourself to be Neurodiverse?}
    \end{minipage}\hfill
    \begin{minipage}{0.4\textwidth}
        \centering
        \begin{tikzpicture}
            \pie[text=pin, radius=1.5]{71/No,
            16.4/Yes,
            12.7/PNTS
            }
        \end{tikzpicture}
        \caption{Would you describe yourself as disadvantaged due to your mental health?}
    \end{minipage}\hfill
\end{figure}