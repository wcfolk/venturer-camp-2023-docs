\chapter{Sustainability and Decarbonisation}
\section{Team Structure}
The Sustainability and Decarbonisation team had a loose structure. Margaret and Paul Fleming took the lead, supported by Ann MacGarry and Bob Todd.\\

It was felt that young people were not sufficiently involved in the process. Ideally young people would drive Sustainability and Decarbonisation, with support from adults only. The Decarbonisation Working Group is focusing on this for future camps.\\ 

The team felt like they needed more engagement from other teams to ensure that sustainable choices were being made throughout the camp.
\subsection{Team Redesign}
The team felt the best future Sustainability and Decarbonisation team would be led by knowledgeable young people, who are identified well in advance of camp and trained. They felt that a University Student could use decarbonisation at a large Woodcraft Folk event as the topic for a dissertation.\\

The team doesn't need to necessarily attend every planning meeting, however they do need to advise on things. There needs to be some planning and consultation with the Sustainability and Decarbonisation team as well as other teams here to understand the best ways to work together for future events.

\section{Supporting Events}
\subsection{Pre-Camp}
The team came to On-Site Pre-Camp and found that a useful experience as it enabled them to meet other teams in person and discuss plans. They felt, however, that the transport to the site could have been arranged better - rather than everyone fending for themselves, arranging a taxi or minibus would have been more efficient.
\subsection{Working Week}
The team were kept very busy during Working Week as most members of the team overlapped with the Electricity team. This involved getting the power station, solar cinema and radio station sorted.

\section{On-Camp Operations}
\subsection{Daily Structure}
The team's days consisted of checking on the power station, trying to get media coverage for the news of a decarbonisation idea and trying to get young people involved in decarbonisation.
\subsection{Time Off}
The team managed a few half-days off when family visited.
\subsection{Support}
The team felt well supported by the Coordinator and Volunteer Support. However they noted that as a team they've been working together for quite some time, which means they know each other quite well. This enables them to support each other efficiently.

\section{Decarbonisation Specific Insights}
\subsection{Evaluation \& Future Camp Recommendations}
The programme team used bleach in the tie-die activity which is bad for the environment. They should be using water-based paints to avoid gasses.\\

Generally, the food was disappointing, in that it could have been more sustainable. There was a lot of processed food and not much processed stuff, not only in the central menu as this was found in the cafe too. It is suggested that we need to minimise travel off site to the shops every day and we should look for bigger fridges and freezers. However, this would require more power and battery capacity. \\

It's recommended to apply for grants to do specific decarbonisation projects, with a proposal for creating a solar-powered shower trailer, using panels, especially for Biblins during Summer Season on the table. Historic large Woodcraft Camps would do projects like this that encouraged more young people to get involved in Decarbonisation - something we should strive to return to.\\

Bikes and bike trailers are really useful for moving things around site! Grant bids should be put in for these to minimise the car use on site for transporting stuff.
