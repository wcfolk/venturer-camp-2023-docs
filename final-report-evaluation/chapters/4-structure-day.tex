\chapter{Structure of the Day}
We decided to have village mornings to enable for rest, clan and village activities, with the central area open in the afternoon and evening for centre activities, live music and dancing on alternate days and a cafe providing a chill out space. Each village also had a day where the participants got to do adventurous activities (climbing and canoeing) which were in the morning as well as the afternoon. This structure worked well as it enabled programme volunteers to get some rest and planning time, as well as giving participants and group leaders time in their villages.

{\RaggedRight \centering
\begin{longtable}{p{0.2\textwidth} p{0.25\textwidth} p{0.45\textwidth}}
\textbf{Start} & \textbf{End} & \textbf{Content} \\ 
\hline
\endhead

\multicolumn{3}{r}{\footnotesize\itshape continued on next page}\\
\endfoot 

\endlastfoot

& 14:30 & Village mornings (this will include rotating adventurous activities) \\ 
\hline
14:30 & 16:00 & Central Programme Slot 1 \\ 
\hline
16:00 & 16:30 & Break \\ 
\hline
16:30 & 17:30 & Central Programme Slot 2 \\ 
\hline
17:30 & 18:00 & Break \\ 
\hline
18:00 & 19:30 & Dinner \\ 
\hline
19:30 & 20:30 & News \\ 
\hline
20:30 & 22:00 & Evening Programme slot 1 \\ 
\hline
22:00 & 22:30 & Sign In \\ 
\hline
22:30 & 23:30 \newline (or 01:00 on 11/08/23) & Evening Programme slot 2 \\ 
\hline

\caption{Daily Structure of Venturer Camp 2023}
\end{longtable}
}% end of \RaggedRight

There were some comments that there wasn't enough time between central activities in the afternoon and the news to get dinner sorted and get prepared for the evening. But this is always an issue when we're trying to cram so much into the day and there isn't an obvious solution without removing some program which is not ideal. 

\section{Daily Meetings}
Village coordinators met with the Camp Coordinator, or Debs on his day off, every morning at 8:00am.  Thomas provided a printout of key pieces of info to take back to villages as well as going through the info verbally, to make sure nothing was forgotten. At request of the village coordinators - the weather was also included on this handout. This worked really well, and many coordinators were pleased to have these summary notes. 
These meetings were generally productive and a really good opportunity to check in with representatives from each village. Having limited numbers of people there (we only had 5 villages whereas an International Camp may have closer to 30) gave a lovely opportunity for the Village Coordinators to get to know each other as well as the Camp Coordinator. This bond helped when there were more difficult conversations to have or more complex minibus logistics to discuss!\\

Throughout the day, the Camp Coordinator made themself available in the cabin at set times should anyone want an audience with them. The timings for these sessions were defined in the village handbook however the timings were more fluid than those advertised. The general policy adhered to was: if the coordinator is in, you can talk to him. This worked well for the most part, except for when the camp coordinator wasn't on shift. This resulted in some difficult boundaries 
While the coordinator was elsewhere on-site with the notion that they were always available for questions. The concept of the On Call Duty Coordinator worked well here as those not on shift were able to divert questions to the person on shift.\\

As part of the food handover from the central team to the village KPs, a short meeting was held. It was compulsory for the Village KP to attend this meeting and to then collect the food, a decision taken to ensure that the right people knew what was going on with the food. A member of the wider food team ran these meetings as the food team didn't feel anyone had the right skills to deliver them. This is something which the Camp Coordinator was made aware of after the event, and something which should have been identified sooner to either upskill one of the Food team members or to identify the individual for them to deliver the meetings - as this was an unnecessary burden placed on them. 
