\documentclass[a4paper, 11pt]{report}

\usepackage{geometry}
\geometry{
a4paper,
total={170mm,257mm},
left=20mm,
top=20mm,
}
\setlength\parindent{0pt} % get rid of the stupid indent

% INCLUDE PACKAGES

\usepackage[utf8]{inputenc}
\usepackage[dvipsnames]{xcolor}
\usepackage{float}
\usepackage{graphicx}
\usepackage{tabularx}
\usepackage{fontawesome}
\usepackage[colorlinks=true, linkcolor=Magenta, breaklinks=true]{hyperref}
\usepackage{ragged2e}
\usepackage{cancel} % used to cancel out numbers in maths mode.
\usepackage{amssymb} % gives more maths symbols
\usepackage{enumitem} % gives ability to have different enumerate bullets.
\usepackage{multirow}

\usepackage{longtable}

\usepackage{fontspec}
\setmainfont[Ligatures=TeX]{Montserrat}
\setsansfont[Ligatures=TeX]{Poppins Bold}
\usepackage[raggedright,bf,sf]{titlesec}

\setcounter{tocdepth}{0}
\renewcommand{\chaptername}{Section}

\usepackage{fancyhdr}
\pagestyle{fancy}
\fancyhf{}
\fancyhead[R]{\leftmark}
\fancyfoot[C]{\thepage}
\renewcommand{\footrulewidth}{0.4pt}
\addtolength{\topmargin}{-1.59999pt} % This and line below changes margins to accomondate for headers and footers
\setlength{\headheight}{13.59999pt}

% NOW CREATE TITLE PAGE

\newcommand{\bookTitle}[4]{
    \begin{titlepage} % Suppresses headers and footers on the title page
	
        \centering % Centre everything on the title page
        
        \rule{\textwidth}{0pt} % Thick horizontal rule
        
        \vspace{0.3\textheight} % Whitespace between the top rules and title
        
        % \rule{\textwidth}{1pt}\\
        \vspace{15pt}
        {\LARGE \textbf{VENTURER CAMP 2023}}\\[1em] % Title line 1
        %{\Large - }\\[0.5\baselineskip] % Title line 2
        {\LARGE #1} \\[15pt]% Title line 3
    
        {\large #2 \\[5em]}
    
        {\large #3} % Date
        \vspace{15pt}
        % \rule{\textwidth}{1pt}
        \vspace{3em}
        %\vfill % Whitespace between the author name and publisher
        
        
        \begin{figure}[H]
            \begin{minipage}{0.45\textwidth}
                \centering
                \includegraphics[width=0.45\textwidth]{../vc23.png} %this will be replaced with vc23 logo when we have one
            \end{minipage}\hfill
            \begin{minipage}{0.45\textwidth}
                \centering
                \includegraphics[width=0.45\textwidth]{../wcf.png}
            \end{minipage}\hfill
        \end{figure}
    
        \vfill  
        
    \end{titlepage}

    \fancyhead[L]{#1}
    \fancyfoot[L]{\footnotesize{\texttt{#4}}}
}

\newcommand{\backPage}{
    \begin{titlepage}
        \centering
        \rule{\textwidth}{0pt}
        \vfill
        \begin{figure}[H]
            \begin{minipage}{0.45\textwidth}
                \begin{flushright}
                \includegraphics[width=0.1\textwidth]{../vc23.png} %this will be replaced with vc23 logo when we have one
                \end{flushright}
            \end{minipage}\hfill
            \begin{minipage}{0.45\textwidth}
                \includegraphics[width=0.1\textwidth]{../wcf.png}
            \end{minipage}\hfill
        \end{figure}
        \rule{\textwidth}{2pt}
        \small{Venturer Camp 2023, a project by \href{https://woodcraft.org.uk}{Woodcraft Folk}, will bring together 13-17 year olds from across the UK to camp together and live by the Woodcraft Folk values for a week in the summer of 2023.\\
        Check out our website (\href{https://venturercamp.org.uk}{venturercamp.org.uk}) and our social media pages for more information.}
    \end{titlepage}
}

% include header footer on new chapter pages by redefining \chapter command
\makeatletter
\renewcommand\chapter{\if@openright\cleardoublepage\else\clearpage\fi
\thispagestyle{fancy}%
\global\@topnum\z@
\@afterindentfalse
\secdef\@chapter\@schapter}
\makeatother

% \usepackage{draftwatermark}
% \SetWatermarkText{\textbf{PROOF}}
% \SetWatermarkScale{1}

\usepackage{ragged2e}
\usepackage{longtable}
\usepackage{multicol, multirow}

\printCopy

\begin{document}
\bookTitlebw{Village Handbook}{A handbook of useful Village Info}{18th July 2023}{village-handbook-v1}
\tableofcontents
\chapter{Welcome to the Village Handbook}
Hail and well met, brave adventurers!\nl

Welcome to Venturer Camp 2023, where the realms of fantasy come alive! Can you feel the magic pulsating through your veins?\nl

Prepare yourselves for an extraordinary journey as you delve into the pages of the Venturer Camp 2023 Village Handbook! Within this tome, you shall discover all the secrets of Venturer Camp, from its enchanting programs to a directory of valiant groups and so much more!\nl

Contained within these hallowed pages is a compendium of practical knowledge about Venturer Camp 2023, guiding you to a safe and joyous existence within its mystical grounds. Herein lies vital information on arrival and registration, traversing the fantastical terrain, and the very fabric of life at Venturer Camp 2023. However, be aware that with the dawning of Venturer Camp 2023 only a fortnight away, this sacred text may alter. These amendments shall be inscribed separately and inserted into the corresponding sections of the Village Handbook Folder.\nl

Let it be known that the awe-inspiring spectacle before you would not have come to pass without the indomitable valor of our five-star volunteer team, who have woven magic to bring this camp into existence. Witness their extraordinary efforts and revel in the joy that surrounds you!\nl

May the skies above remain azure, and may your adventures be filled with wonder and mirth.\nl

In the mystical depths of Venturer Camp,\nl
Thomas,\nl
Coordinator of Venturer Camp 2023\nl

\textit{When I found out how amazing the Village Handbook sounded when given to ChatGPT with the prompt ``Can you rewrite this with a high fantasy theme'', I had to include some of it! Thanks Joe from the Cafe \& Allergy Kitchen team for thinking of it!}


\chapter{Basic Site \& Event Information}
The full address of the site is as follows:\nl
Biblins Youth Campsite,\nl
The Doward,\nl
Witchurch,\nl
Ross-on-Wye,\nl
HR9 6DX\nl
OS Grid Reference: \texttt{SO 549 154}\nl
What3Words: \texttt{famous.twit.galloped}\nl

Driving directions to site can be found in \textit{Info Pack v2}, which is available on our website. 

\section{Safeguarding Contacts}
Throughout the camp, the Safeguarding team will be running drop-in sessions in the Camp Office. These will be at the following times daily:
\begin{itemize}
    \item 11:00 - 12:30
    \item 14:00 - 15:30
    \item 21:00 - 22:30
\end{itemize}
The Safeguarding team will also be on call 24 hours a day for the duration of the event. They can be contacted via any steward, the camp office or the camp info point. The Safeguarding inbox (\texttt{safeguarding@woodcraft.org.uk}) will also be monitored from off-site throughout the duration of the event.

\section{Coordinator Contacts}
There will always be an On-Call Duty Coordinator (OCDC), they will be contactable from the Camp Info Point or Stewards Office via Camp Walkie Talkie.\nl

At the following times, there will be a Coordinator Drop-In Session held in the Camp Office:
\begin{itemize}
    \item 09:00 - 09:45
    \item 15:00 - 15:45
    \item 20:30 - 21:15
\end{itemize}
There are a number of different people on-camp who will be taking on the role of On-Call Duty Coordinator. When a person is ``on-shift'' they will be wearing an orange hi-vis. Please respect the hi-vis and do not question those who have taken on the OCDC role when they are not wearing a hi-vis. 

\section{Emergency Contacts}
Nearest 24hr A\&E: Stonebow Road, Hereford, Herefordshire, HR1 2BN 01432 355444\nl

Nearest Urgent Treatment open 16:00 - 19:30: Grove Road, Lydney, Gloucestershire, GL15 5JE 0300 421 8722\nl

There is a landline phone on the front of the Camp Office. This will be available 24/7. \nl
If you call the emergency services and they come to the site, please alert the stewards so that they can be ready to guide the emergency vehicle to the correct part of the site. 

\section{Any Questions?}
If you have any questions, the first point of contact will be the Camp Info Point - this is one of the marquees opposite the Camp Office and will have lots of friendly, smiling faces in it!\nl
The Camp Info Point will be able to provide support with most things ranging from contacting Site Services about an issue with site infrastructure to telling you what programme is happening that afternoon!

\chapter{Schedules}
\section{Daily Schedule}
Each day will take the following structure:

{\RaggedRight \centering
\begin{longtable}{p{0.2\textwidth} p{0.2\textwidth} p{0.5\textwidth}}
\textbf{Start} & \textbf{End} & \textbf{Content} \\ 
\hline
\endhead

\multicolumn{3}{r}{\footnotesize\itshape continued on next page}\\
\endfoot 

\endlastfoot

& 14:30 & Village mornings (this will include rotating adventurous activities) \\ 
\hline
14:30 & 16:00 & Central Programme Slot 1 \\ 
\hline
16:00 & 16:30 & Break \\ 
\hline
16:30 & 17:30 & Central Programme Slot 2 \\ 
\hline
17:30 & 18:00 & Break \\ 
\hline
18:00 & 19:30 & Dinner \\ 
\hline
19:30 & 20:30 & News \\ 
\hline
20:30 & 22:00 & Evening Programme slot 1 \\ 
\hline
22:00 & 22:30 & Sign In \\ 
\hline
22:30 & 23:30 \newline (or 01:00 on 11/08/23) & Evening Programme slot 2 \\ 
\hline

\caption{Daily Structure of Venturer Camp 2023}
\end{longtable}
}% end of \RaggedRight

\section{Weekly Schedule}
The two tables below show the afternoon schedule for the week and the evening schedule for the week.\nl

The mornings will be village mornings where villages will run their own activities. There will be some consultation activities which villages are expected to run at set times throughout the camp. Details of these can be found further in the handbook. 
\begin{table}[H]
    \centering
    {\RaggedRight
    \begin{tabular}{p{0.2\textwidth} p{0.2\textwidth} p{0.07\textwidth} p{0.2\textwidth} p{0.08\textwidth} p{0.08\textwidth}}
        \textbf{Day} & \textbf{14:30} & \textbf{16:00} & \textbf{16:30} & \textbf{17:30} & \textbf{18:00} \\
        \hline
        Saturday 5th & \multicolumn{4}{l}{Arrivals} & Dinner \\
        \hline
        Sunday 6th & Centres (Hiroshima Day focus) & Break & Centres (Hiroshima Day focus) & Break & Dinner \\
        \hline
        Monday 7th & Centres and adventure activities & Break & Centres and adventure activities & Break & Dinner \\
        \hline
        Tuesday 8th & Centres and adventure activities & Break & Centres and adventure activities & Break & Dinner \\
        \hline
        Wednesday 9th & \multicolumn{3}{l}{Wide game} & Break & Dinner \\
        \hline
        Thursday 10th & AGM, centres and adventure activities & Break & AGM, centres and adventure activities & Break & Dinner \\
        \hline 
        Friday 11th & Centres and adventure activities & Break & Centres and adventure activities & Break & Dinner\\
        \hline
    \end{tabular}
    }% end of \RaggedRight
    \caption{Afternoon Programme Schedule}
\end{table}

\begin{table}[H]
    \centering
    {\RaggedRight
    \begin{tabular}{p{0.2\textwidth} p{0.1\textwidth} p{0.2\textwidth} p{0.1\textwidth} p{0.2\textwidth}}
    \textbf{Day} & \textbf{19:30} & \textbf{20:30} & \textbf{22:00} & \textbf{22:30} \\
    \hline
    Saturday 5th & News & Central Programme  - Band(s) & Sign in & Programme continues until 23:30 \\
    \hline
    Sunday 6th & News & Village Programme & Sign in &  \\
    \hline
    Monday 7th & News & Central Programme - Ceilidh & Sign in & Programme continues until 23:30 \\
    \hline
    Tuesday 8th & News & Village Programme & Sign in &  \\
    \hline
    Wednesday 9th & News & Merrymoot, with an exciting guest host & Sign in & Programme continues until 23:30 \\
    \hline
    Thursday 10th & News & Village Programme & Sign in &  \\
    \hline
    Friday 11th & News & Central Programme - Band(s) & Sign in & Programme continues until 1am\\
    \hline
    \end{tabular}
    }% end of \RaggedRight
    \caption{Evening Programme Schedule}
\end{table}

The Woodcraft Folk AGM will be taking place in the afternoon of Thursday 8th August. This will happen in the Main Marquee.

\subsection{Mini-Themes}
There will be 3 different mini-themes on camp, as part of the wider theme of mythology. Every two days there will be a different mini-theme working towards the central programme night, where this will be the dress up theme. There will also be some relevant workshops, and the Wide Game will fit into this too.

\section{Key Daily points for Village Officeholders}
On site during the event, Village Coordinator's will be invited to a morning meeting with Thomas Boxall, Camp Coordinator, where they will be able to find out the latest information on everything ranging from programme to waste collection. Morning circles within villages shouldn't be held until after the Village Coordinator's meeting to ensure the most up to date information is disseminated to young people. 
\begin{table}[H]
    \centering
    {\RaggedRight
    \begin{tabular}{p{0.2\textwidth} p{0.2\textwidth} p{0.5\textwidth}}
        \textbf{Start} & \textbf{End} & \textbf{Content} \\
        \hline
        08:00 & 08:15 & Village Coordinator's Meeting outside the Camp Office \\
        \hline
        11:00 & 11:30 & KP Briefing \& Food Collection @ Food Distribution Tent \\
        \hline
        11:00 & 11:45 & Central Clan to report to Site Services Tent\\
        \hline
    \end{tabular}
    }% end of \RaggedRight
    \caption{Key timings in the day for village officeholders}
\end{table}

\section{Sign-ins of Participants Throughout The Day}
Venturer leaders should check with the young people they are responsible for throughout the day, to ensure they are safe and happy. It is especially important to check in with them during the Morning Circle, after afternoon programme, at ``Sign In'' time (22:00) and again after central evening programme finishes. \nl

If a participant misses two or more consecutive of the check in points outlined above, stewards should be notified and the missing persons procedures will be followed. This should be communicated to participants at the start of camp.\nl

If a participant fails to return for Evening Sign In (22:00) or return to their village after Evening Programme finishes, the Stewards should be notified immediately. 

\section{Central Clan}
Every day, each village will need to send 7 people including at least one adult to the Site Services tent at 11:00 to take part in Central Clan.\nl

This role promotes a communal sense of responsibility for the shared areas of the campsite and may include supporting the Site Services team, Food team, production team or other members of the central team.

\section{Consent Session}
On Sunday 6th August, a consent workshop for all adults and young people will be delivered in each village at 10:00, by a pre-agreed facilitator. This is part of Woodcraft Folk's preventative approach to preventing harm and reducing peer-to-peer abuse. \nl

The pre-agreed facilitators will meet with each respective Village Coordinator on Saturday 5th August at some point in the afternoon / evening to check in about the plans for the session on Sunday 6th. \nl

All participants who have arrived on site by this point need to be at this session. 

\chapter{Arrivals \& Signing In}
\section{Arrivals}
\subsection{For arrivals on 5th August}
When you arrive to site, you will need to go to the Camp Info Point, situated opposite the Camp Office. Not all the people in the booking need to go to the sign in point, one volunteer can represent everyone in the group. 
\subsection{For arrivals after 5th August}
When you arrive to site, you will need to go to the Camp Office.
\section{Signing In}
When you sign in, everyone in the group will be given a Camp Wristband and a Camp Pocket Guide. The wristband must be worn at all times as this will be the primary method of identifying who should be on site and who shouldn't.\nl

Those who do not have photo consent will also be given a length of red ribbon, this should also be worn all the time while on site. 
\section{Getting to Site on 5th August}
On arrivals day (5th August), we expect over 400 people to want to get onto site! That's a lot of people so we've tried to make it as easy as possible for you to be able to find your camping pitch quickly and safely!\nl

If you're arriving after August 5th, then the arrangements will be slightly different. If arriving by car / van, you'll be able to drive directly onto site and park straight in the car park.\nl

If you have any questions or require access arrangements to be made, please contact the Coordination \& Event Administration team via \texttt{info@venturercamp.org.uk}

\subsection{I'm dropping someone off by car\ldots}
When you arrive onto site, there will be stewards directing you to the drop-off area. This drop-off area is Biblins' usual car park and we will be operating a one-way system within it. You will be able to park up and walk the person you are dropping off to the relevant area of the site then return to your car and leave site.\nl

Please note that the walk from the drop off area to the entrance of site is about 200m down a forestry track, not suitable for wheeled suitcases.

\subsection{I'm coming by car / van / lorry and leaving it on site\ldots}
When you reach the drop off area, let the stewards know that you're keeping your vehicle on site and they will direct you onto site. Please be extra careful on the first day of camp of people walking up / down the track.\nl

After you have unloaded your vehicle, it will need to be parked in the designated car park on site. This will be located in pitch 4, next to the bridge. No vehicles should be left in villages or parked adjacent to villages unless pre-agreed with the Camp Coordinator.

\subsection{I'm coming by train and shuttle bus\ldots}
By now, you should have been contacted by the Coach Coordinators. They will arrange for you to have a place on one of the shuttle buses from Hereford station to site.\nl

The shuttle bus will drop you off on the Wales side of the river, you will then need to carry your belongings over the river to the campsite. 

\subsection{I'm coming by Coach or vehicle over 7.5 tonnes\ldots}
The main entrance track for Biblins is unsuitable for large vehicles or those weighing over 7.5 tonnes. If you are coming in either of these types of vehicle, then please contact \texttt{info@venturercamp.org.uk} so alternative arrangements can be made.

\chapter{Villages}
The campers at Venturer Camp 2023 will be split between five villages.  The five villages at Venturer Camp are named for mythical, magical places from ancient traditions and beliefs.
\section{Groups Directory}
{\RaggedRight \centering
\begin{longtable}{p{0.3\textwidth} p{0.6\textwidth}}
    \textbf{Name} & \textbf{Containing}\\
        \hline
        \endhead

        \multicolumn{2}{r}{\footnotesize\itshape continued on next page}\\
        \endfoot 
        
        \endlastfoot
        \multirow{6}{*}{Asgard} & Newham Watersmeet\\*
        & Cambridge\\*
        & St Albans\\*
        & Eastern Region\\*
        & Sheffield Derwent\\*
        & Birkenhead\\
        \hline

        \multirow{10}{*}{Benben} & Clapham\\*
        &  Highgate \& Holloway \\*
        &  Teddington \\*
        &  New Barnet \\*
        &  Camps For All \\*
        &  Banbury \\*
        &  Stroud \\*
        &  Food Team \\*
        &  Woodcraft Folk Staff \\*
        &  International Volunteers \\*
        \hline

        \multirow{9}{*}{Camelot} & Lewisham \& Greenwich\\*
        & Exeter \\*
        & Brighton \& Hove Central \\*
        & Watford \\*
        & Eastbourne \\*
        & Tyne \\*
        & Leeds \\*
        & Coach Coordinators \\*
        & Camp Coordinators \\
        \hline

        \multirow{6}{*}{Dinas Affaraon} & Oxford\\*
        & Hackney \\*
        & Brighthelmstone \\*
        & Bath \\*
        & MEST-UP Centre Team \\*
        & Adult Volunteers \\
        \hline

        \multirow{7}{*}{Elysium} & Scotland\\*
        & Machynlleth \\*
        & Cardiff \\*
        & Manchester \\*
        & Southampton \\*
        & Ealing \\*
        & Bromley \\
        \hline

\caption{Village names and contained groups}
\end{longtable}
}

\section{Central Team in Each Village}
At Venturer Camp 2023, rather than having a ``Central Village'', a number of the central coordination team will be placed in each village.\nl

These members of the central coordination team may need accommodations making for them, this could include:
\begin{itemize}
    \item Making meals available at non-standard times
    \item Not rota-ing them into clan
    \item Not rota-ing them into responsibilities within the village
\end{itemize}
Conversations should happen between Village Coordinators and members of the Central Coordination Team in your village either in advance of camp or on the first day of camp to ensure that everyone can have a good experience at Venturer Camp 2023.\nl

A list of those who may need accommodations made within the village is available in the Village Handbook Folder.

\subsection{Making Sure the Central Team gets Fed}
At previous large Woodcraft Folk camps, there have been reports of members of the Central Team or core volunteers within the camp not being able to access food within their own villages. This makes for a very unpleasant experience for the people who are working extremely hard to make the camp happen.\nl

We want to make sure that everyone has a positive experience at Venturer Camp 2023 so it is suggested that members of the central coordination team ``buddy-up'' with a volunteer in the village who can collect their portion of food and make sure that it is saved, if they are unable to make it to the meal. Members of the central coordination team should also communicate to the Village KP and their buddy if they are eating elsewhere on site, or eating off site. 

\section{Tent Layout}
As a fire safety precaution, there needs to be 2m between any structure or tent on the site. This is extended to 10m between any heat source and a sleeping tent. \nl

Every village needs to ensure they have created an access point of 3m so it is wide enough for an ambulance to fit through into the middle of the circle of tents. This should be in addition to any decorative Village Gates to ensure there is sufficient headroom for a large vehicle to enter the Village.  \nl

Large structures should be placed alongside the track, creating a barrier between the footpath and villages which hopefully will deter walkers and members of the public who are using the footpath from entering the villages. 

\chapter{Off-Site Communications}
Communication off-site at Venturer Camp 2023 presents a unique challenge. This is because there is no phone signal on the site and the WiFi we have is extremely limited.
\section{Wi-Fi Access}
The Camp Office has a WiFi connection which will be made available, where absolutely necessary, to volunteers for them to communicate off-site with parents / carers of young people. \nl

WiFi will also be made available, in special circumstances, to participants. These are included but not limited to:
\begin{itemize}
    \item Receiving exam results (incl. Scottish Highers)
    \item Communicating with parents where there is a need
\end{itemize}

When someone needs access to the WiFi, they should go to the Camp Office where the password will be given out. The password must not be given to all the young people in a group as this will result in the connection being overloaded and not working.\nl

There may be times of the day during which the WiFi is unavailable. Please try again another time if this is the case.\nl

The WiFi password will be changed frequently.

\section{Contacting the Site}
The following methods of communication will reach the on-site team. WhatsApp or Email are the preferred mechanisms.
\begin{table}[H]
    \centering
    \begin{tabular}{p{0.3\textwidth} p{0.6\textwidth}}
        \hline
        Email & info@venturercamp.org.uk\\
        \hline
        WhatsApp & +44 7716 372651\\
        \hline
        Phone & 01600 890 850\\
        \hline
    \end{tabular}
    \caption{Contact methods during the event}
\end{table}

\section{Where to Find Mobile Phone Signal}
Mobile signal can generally be found a short walk up the forestry track, towards the Doward Campsite.\nl

Volunteers and Participants need to sign out from site if they are going to find signal as they are leaving the site. This will need to be done at the Stewards Office.

\chapter{Site Services}
The Site Services team will be based from their tent in the Car Park. If you need to contact them and no one is available, the people at the Camp Info point will be able to radio for them. 

\section{Waste \& Recycling}
There are waste and recycling bins throughout the site. Please ensure you put your rubbish in the correct bins. There is no glass or food recycling available. Please ensure your recycling waste is washed and clean before putting it into the recycling bin. 

\section{Gas}
Gas cylinders need to be staked into the ground. Fix the cylinders upright next to the kitchen tent. Ensure the gas cylinders are not located next to the public path. No smoking or naked lights near the gas cylinders. Keep the cylinders well ventilated and only authorised people to access the gas and keep a log of those who do so. Do not take shortcuts and carry gas cylinders through tents and be mindful of guy ropes and pegs.\nl

Please bring both screw-fit and clip-on regulators for gas equipment where possible and do not run every appliance off one gas bottle.\nl

Please contact Site Services between 11:00 - 12:00 if you need a replacement.

\section{Fires and Wood}
Please only use the fire circle which is clearly marked in each village. To ensure we have enough wood, the maximum each village can take is one wheelbarrow per night from the woodpile. Please do not store more than this in your village. \nl

Please ensure you have a bucket of water near the fire to be able to extinguish the fire. Do not leave fires unattended and ensure the fire is put out before going to bed.

\section{Toilets \& Showers}
Please keep the toilets and showers clean. If you notice they are dirty, please clean them up before you leave, like you would do in your own home.\nl

Please put sanitary products in the waste bins provided. If you are using a portaloo, please put it into the loo.\nl

If you noticed a blockage, please inform the site service team.\nl

Each village will be on duty to help with the maintenance of the toilets and showers. \nl

\section{Water}
There are water taps throughout the site and they are clearly marked on the map. Please do not wash up in sink basins. Washing up is to be done only in the allocated areas or in the villages.\nl

Please be water wise. If you notice a tap is running, and not used, please turn it off.

\section{Backstage, Power Station and Solar Array}
You must not go backstage in the main marquee, into the power station tent or within the boundary around the solar array. There will be cables laying along the ground and it may be dark.\nl

If you need access to any of: backstage in the main marquee, the power station tent or within the boundary of the solar array - please speak to the Site Services, Production or Electrical team.


\section{Site Care}
Please look after the site like your own home. We are in the public eye representing Woodcraft Folk and we want to create a good impression on the local population.\nl

Please only enter a village via the village gate. Take care around guy ropes and pegs. \nl

If you see litter, please pick it up and put it in the nearest litter bin.\nl

Report any issues to your leader, the Site Services team or the Camp Info Point.

\section{Restrictions to access of Wooded Areas}
A large amount of the wooded areas surrounding Biblins is a Site of Special Scientific Interest, this means access is restricted.\nl

If you go into the woods, please stay on the marked tracks unless otherwise told (for example, as part of an activity).

\section{Spare Equipment}
Nobody is to take any equipment from any store cupboards on site without consulting with the Site Services team. The Site Services team will be more than happy to support you with finding equipment / materials if needed.

\section{Waste at the end of the event}
Things break, we definitely learnt this at Common Ground! If something breaks, please don't leave it on site for us to clear up - Biblins has limited waste disposal which we will already be pushing to its limits from the numbers of people on site so adding broken tents, air mattresses or other equipment to this will make the job really tough for the Site Services team \& staff team at Biblins. 

\section{Smoking \& Vaping}
Smoking and vaping is not permitted in any village or central area. The permitted smoking and vaping area is in the car park up the track and at the Canoe turning circle adjacent to Pitch 11. Please put your rubbish in the bins provided. At the end of each day, a smoker / vaper needs to empty the rubbish bin into one of Biblins large red waste bins.

\chapter{Incidents, Emergencies \& Evacuation}
In the event of an incident occurring, the first priority is the safety of those on site.\nl

If you see something that doesn't look right, please speak to a steward or member of the central team as soon as possible.

\section{In the event of Fire}
In the event of a fire, the alarm should be raised by operating the nearest fire alarm and, if safe to do so, the fire should be extinguished with equipment provided.\nl

If the fire services are required, the On-Call Duty Coordinator and On-Shift Head Steward should be alerted. The fire services should be telephoned, there is an emergency phone available on the front of the Camp Office.\nl

Assembly Points are located at:
\begin{itemize}
    \item the car park beside Herf's Barn
    \item the fire box adjacent to pitch 8
    \item the fire box adjacent to pitch 3
    \item the fire box located near Camp Koodoo
\end{itemize}

The Head Steward and On-Call Duty Coordinator are responsible for ensuring that the access track is clear \& all gates are open so that an emergency vehicle can enter the site.\nl

There are no planned fire drills at Venturer Camp 2023.

\section{If a flood warning is received}
In the unlikely event that a flood warning is received for Biblins Campsite, the flood plan will be followed. The Head Stewards will instruct campers on where to move themselves to and where vehicles should be relocated to.\nl

Competent adults may be needed to assist with preparing the site for a flood, if the need for this arises - the Site Services team, Biblins site team and Head Stewards will ask. 

\section{If a Safeguarding Incident occurs}
If a safeguarding incident occurs or a disclosure is made, the Venturer Camp 2023 Safeguarding Plan should be followed. A copy of this document and the Incident \& Disclosure form can be found in the Village Handbook Folder.\nl

The Safeguarding team will be in the Camp Office at set times of the day, see the Safeguarding Contacts section for more information.

\chapter{Stewards}
Venturer Camp stewards are here to help, to make the event more enjoyable for everyone and to make the event safe for everyone. They are not a police force.
\section{What will they do?}
Stewards will patrol around the site, monitor the main entrance to the site and will ensure that any large gatherings of campers are safe. They will also be around during the central evening programme to assist where necessary.\nl

Some of our stewards are first aid trained, but all of them are able to contact a first aider through their Walkie Talkies. \nl

Contact a steward if you have any issues during camp and if they can't help you then they will be able to tell you who can!

\section{What do stewards look like \& where can I find one?}
When a steward is on-shift, they will be wearing dashing yellow hi-vis'.\nl

They will be patrolling around the site and will be at larger gatherings including central evening programme.  There is also a Steward's office outside the Camp Office which will have a steward in most of the time. 

\section{Can I steward?}
If you are interested in getting involved in stewarding or have anyone in your village who is interested then send them over to the Steward's Office.\nl

Stewards must be over the age of 16.

\chapter{Safety on Site}
\section{The River Wye}
Many of our camping pitches back directly onto the river, while the riverbank is clearly marked, it is not fenced off. \nl

The river is fast moving and not suitable for swimming in. Canoes may launch from the launch on the eastern end of the site. Access to the river from anywhere else along the river bank is not permitted. 

\section{Vehicles on Site}
Groups are asked to keep driving on site to the absolute minimum needed to arrive/leave site with their group.  There is a speed limit of 10 miles per hour and drivers should be aware that the path through the site is very popular with local walkers as well as being used by other campers.\nl

Vehicles must be parked in the designated car park and not adjacent to villages during the event, unless expressly agreed in advance, e.g. for disabled access.
\section{Members of the public passing through}
Biblins has a public footpath which runs the length of it, along the track as well as across the bridge. We expect lots of members of the public to be walking through the site during camp.\nl

If members of the public stray into villages or into areas of the site we are using, they should be politely challenged and asked to return to the footpath. We expect some interest in what we are doing, please point them in the direction of the Info Point (in the central area) if they have lots of questions which you don't want to answer.\nl

A set of leaflets will be given to each village which can be given out to members of the public, more copies of these will be available from the Camp Info Point, should you need them. \nl

Members of the public often stop to have a break and make use of vacant camping pitches for this, if this happens please politely challenge them and ask them to move on.\nl

All dogs brought onto site must be kept on leads, again if members of the public are not respecting this - please politely challenge them and ask them to do so. If a dog owner does not pick up after their dog, please report this to Site Services so the excrement can be dealt with. \nl

There is a public toilet at the West Toilet Block (next to Camp Koodoo), please direct any members of the public looking for a toilet to this one. They must not use any of the other toilets on site.

\section{Bivvying}
We do not have a designated bivvying space on site.\nl

If you wish to do bivvying as an activity, then please use space in your village or adjacent to your village on a camping pitch.

\chapter{Medication}
\section{First Aiders}
Each village will have a nominated individual acting as a Village First Aider. They should be first point of call for village level first aid incidents.\nl

If an incident requires further medical attention but not enough to call an ambulance, we have a number of medical professionals at camp with us who can be contacted via the Camp Office / Info Point.

\section{Medication Storage}
Village Coordinators and delegation leaders should know who in their village requires medication. Medication belonging to under-18s should be stored with a responsible adult who can administer it at the required times. This adult should be one which the young person is familiar with.\nl

If medication needs to be kept cool, there will be a medication fridge in the Camp Office. Medication cannot be stored in the food distribution facility. 

\chapter{Consultation Activities}
General Council, Camp 100 and the Heading to 100 teams would really value feedback and suggestions from campers - as such we have prepared some consultation session plans to seek the ideas, thoughts and priorities of young members and volunteers.\nl

The consultation session plans are seeking your feedback on:
\begin{itemize}
    \item Camp 100, 27th July-6th August 2025
    \item How should Woodcraft Folk celebrate its centenary?
    \item Woodcraft Folk's next strategic plan: what should our priorities be 2025-2030?
\end{itemize}

Please make time to facilitate these sessions in your village, they can be done as part of your morning village programme, discussions over dinner or around the campfire. You are welcome to adapt the session plan, but please note the session aims and try to gather responses to the questions the teams have asked (you will find these listed at the top of the session plan).

You will find all the session plans in the Village Handbook folder and additional materials will be given out at the first Village Coordinators meeting.\nl

Please return your responses to the Camp Office for Debs McCahon.\nl

If you need any help or have any questions please don't hesitate to find Debs in Benben village.

Thanks in advance.\nl

\chapter{Camp Census}
Woodcraft Folk's Equality Diversity \& Inclusion Working Group was established at the end of 2022. Its focus is to explore and address barriers to participation, both for young members and volunteers. \nl

On Tuesday morning all campers will be asked to complete a simple tick box demographic form. Paper copies will be given to Village Co-ordinators during the morning meeting. The form asks a series of questions, including:
\begin{itemize}
    \item Age 
    \item Gender
    \item Ethnicity
    \item Disability \& health related questions
    \item Caring responsibilities
\end{itemize}

The results of which will be shared on camp news, and will also help the working group better understand who participates in national events and importantly who is under-represented. Further censuses will be carried out across the movement to provide a comparison.\nl

For more information about the working group contact \texttt{debs@woodcraft.org.uk} The group is always keen to welcome new members!

\chapter{Decarbonisation}
\section{Summary}
We are trying to reduce the greenhouse gas emissions associated with Woodcraft Folk operations, buildings and events. We should try and not create the greenhouse gas emissions in the first place and then to minimise whatever emissions are related to our work.\nl

The major greenhouse gas emissions are associated with the use of energy for mobility, heat, light and power. Also, with the stuff we buy to carry out our work and the food that we provide at our events.\nl

The actions can be summarised as
\begin{enumerate}
    \item Reduce the need to travel or travel collectively in the most low-carbon form of transport.
    \item Reduce the heat, light and power demand
    \item Don't buy too much stuff and make sure any waste is minimised re-used or composted
    \item Make sure that food for events is purchased with low carbon in mind
\end{enumerate}

\section{What does this mean in the Villages and in the Central Area?}
\subsection{Cars}
Avoid driving cars whilst at camp. Realistically the only people needing to regularly go offsite by car will be those carrying out KP and Site Services duties. We will provide details of the many lovely walks in the area. We will also aim to have some bicycles to borrow, the ride to Monmouth on the `Peregrine trail' is magnificent. There are two pubs at Symonds Yat within an easy half hour walk and both have WiFi.
\subsection{Electrical Power}
Because Biblins is not on the national electricity grid all the electrical power will be provide by renewable energy. Biblins already has solar panels on the warden's cabin and the toilet blocks. The Edinburgh PowerPod provides electricity to the Bunkhouse and Bunkhouse toilets. You will see a temporary solar power station in the central area to meet the needs of the centres and the main stage and you might see smaller solar power kits in villages to provide light and phone charging (some villages will have this and some won't have this). Because the amount of solar power is limited (because it is still quite expensive) you will see energy efficient lights, amplifiers, etc to keep electricity demand as low as possible. It will not be possible to use things like electric kettles, hair dryers or straighteners that use a lot of energy. Check with the power station before plugging in anything other than a phone charger.
\subsection{Liquefied Petroleum Gas (LPG)}
The largest single source of carbon emissions at camp will be associated with the use of LPG. Biblins uses LPG for heating the water in the toilets and showers and villages will use LGP for cooking. The best way to reduce these emissions is to use less. For example, use less hot water in the shower and don't leave gas hot water boilers on all the time. Here are some hints and tips to use less gas
\begin{enumerate}
    \item Bring a Kelly kettle to camp. Make pots of tea and coffee using `tinder' and kindling. Further information on Kelly Kettles can be found here: \texttt{https://www.kellykettle.com/how-to-use-the-kelly-kettle}.
    \item Try retained heat (hay box) cooking. There is lots of information on line. This is a good link \texttt{https://www.lowimpact.org/categories/retained-heat}. We will also have a couple of books available at camp, together with basic equipment (insulating materials / boxes and fabric) so villages can experiment and try out some options.
\end{enumerate}

\subsection{Water}
Reducing water use (both hot and cold but particularly hot) will help reduce the associated carbon emissions. Again, use less water in the showers and also try to minimise water use when cooking and washing up. Since Biblins is also not part of the sewerage network, waste from toilets and showers goes into septic tanks. Sometimes these overflows when lots of people are on site and too much water is used.

\subsection{Food}
The camp is trying to minimise carbon emissions associated with the food we will be eating. There will be more information about this during Camp itself, but please try to avoid food waste. This (\texttt{https://myemissions.green/food-carbon-footprint-calculator/}) is a great free website that can quickly provide information re the embedded carbon in our food. Our cafes will use the data to give a daily carbon list on their menus. 

\subsection{Waste}
Recycling facilities are provided at Biblins, so please use them if you have waste. Try to avoid generating the waste it the first place if you can. We are trying to deal with composting, but a whole camp site composting system is not in place yet.

\section{How can we further Decarbonise our events?}
Please talk to your Village Co-ordinators, the Power Station Volunteers or any of the Venturer Camp Committee about what more you think we can do at VCamp, at Biblins and at other Woodcraft Folk Centres and future Camps.

\chapter{Volunteer Support}
The volunteer support centre will be open daily from 14:30-17:30, located in the marquee in Camp Koodoo. It will be a space for any volunteers to come to if they are feeling stressed, overwhelmed or upset. There will be someone  there to talk to as well as the option of refreshments and relaxation techniques to help alleviate stress. \nl

In addition to the drop-in space, the volunteer support centre will host two workshops - one where leaders can share ideas and good practice around supporting young DF groups and another looking at Woodcraft Folk's Equality, Diversity and Inclusion work. Timings tbc.

\chapter{Programme}
\section{Centers}
\begin{description}
    \item[Activism] a space full of activities about the climate, anti-racism, LGBTQI+ rights, feminism, and more. This centre aims to encourage people to challenge the 'black and white, good or bad' view often found in activist circles and communities and instead explore issues from a more holistic point of view. There will be lots of exciting special guests running workshops as well as our excellent volunteers.
    \item[Arts] a tent to explore and practise all kinds of art: visual, performance, couture, classic and modern. At drop in time it will be a calm space for Venturers to chill out and sit with their own crafts. Workshops will mostly be a bit more active with a different disciplinary focus each time. The centre will become a gallery of itself as camp goes on.
    \item[Media] a space where you can be guided through the process of drafting, filming and editing together videos. These will be watched by everyone on camp in the main marquee, as part of the News each night.
    \item[MEST-UP] Mediation, Education, Support Team Umbrella Project, to put it simply, we are DFs who have been trained to provide support to our peers at DF camps and although we are a small team this year, we will have a presence at VCamp too! Whether it be information/support surrounding consent, sex, friendships, drugs, alcohol, mental health, boundaries, friendships, or just a general chat, we're here! We have a safe space for anyone needing some alone time in a quiet space. We also provide free condoms and period products. Pop by the MEST UP centre for a general nosy, chat, or if you want some information on DF's, the 16-21 year old woodies! We will be running workshops and Drop Ins throughout the week and would love to see you there :)
    \item[Mythology] a centre to explore the camp theme of mythology through lots of  different kinds of activities, from craft to games to discussions.
    \item[Solar Cinema] a centre where you can see blockbusters and smaller releases, including some bigger events where discussions and Q\&A sessions follow the film. Some viewings may be outside of typical centre hours, but we will make timings clear in the programme attendees are given on site.
    \item[Radio] a centre where Venturers can run their own radio programme, be that a talk show, an interview, simply djing, or a mixture. There will be regular radio programmes throughout camp, so you should bring an FM radio from home if you have one (we will provide some to villages and centres but more would be good)!
\end{description}
\section{Centre Activities}
We will give sign up sheets to village coordinators each morning in the village coordinator meeting so that Venturers can sign up for the day's activities (other than drop in sessions). There will be limited spaces for each workshop per village to ensure the workshops are safe and enjoyable. If there are any spaces left for workshops these will be available on a first come first serve basis once activities start at 14:30.

\section{Adventurous Activities}
Each village will be assigned a day to take part in climbing and canoeing activities in 3 time slots throughout the day: 09:00 - 11:00, 11:30 - 13:30 and 14:30 - 16:30 to give all the young people in each village a chance to have a go. Sign up will happen the night before within the village(s) taking part in the activities. 
\begin{table}[H]
    \centering
    \begin{tabular}{ll}
    \textbf{Day} & \textbf{Village(s)}\\
    \hline
    Monday 7th & Asgard \\
    \hline
    Tuesday 8th & Benben \& Camelot \\
    \hline
    Wednesday 9th & No Activities - Wide Game \\
    \hline
    Thursday 10th & Dinas Affaraon \\
    \hline
    Friday 11th & Elysium \\
    \hline
    \end{tabular}
    \caption{Village allocation for Adventurous Activities}
\end{table}

\section{The Wide Game}
The wide game will take place on Wednesday 9th, starting in the morning with a brief pause for lunch then returning after lunch to finish off and have reflection on the events of the wide game. This means that lunch should be prepped with breakfast on Wednesday (This will be in the KP handbook).\nl

The wide game team will be looking for DFs and Kinsfolk to play roles and help out in the wide game on camp so make sure to let Anya (programme coordinator) know of any volunteers. Email Anya at \texttt{programme@venturercamp.org.uk} or find her on site in \& around the Central Area.

\section{Bringing Decorations for the Central Area}
If your district/village has any decoration e.g: bunting that you can lend the central area for the duration of camp please bring them with you to be put up when you get to camp, let Anya know when you get to camp and putting up these decorations can be arranged. 

\chapter{Ambrosia Café}
Introducing Ambrosia: Your Legendary Café Retreat.\nl

Escape the hustle and bustle of VCamp's workshops and wonders, discover a hidden gem nestled within its enchanting grounds. Welcome to Ambrosia, a café that blends the charm of a cosy tavern with the allure of legends. As you step into our realm, prepare to embark on a culinary journey that will tantalise your taste buds and soothe your soul.\nl

Amidst the melodic tunes of relaxing music, Ambrosia beckons you to indulge in a delightful selection of hot and iced drinks. Savour the rich aromas of freshly brewed coffees, revel in the refreshing flavours of our signature fruit smoothies, and let the warmth of our handcrafted teas envelop you. Whether you seek a rejuvenating boost or a moment of tranquil repose, Ambrosia has the perfect libation to suit your desires.\nl

Throughout the day, our delectable array of homemade biscuits, pastries, and cakes will entice you with their irresistible allure. Each bite is a symphony of flavours, crafted with love and devotion. Indulge in the buttery crumble of our biscuits, savour the flaky perfection of our pastries, and lose yourself in the divine sweetness of our cakes. These delectable treats are the stuff of legends themselves, prepared with care to ensure every mouthful transports you to a realm of pure bliss.\nl

As the sun sets and the stars begin to twinkle, Ambrosia undergoes a transformation, embracing the magic of the evening. Prepare your taste buds for a culinary adventure with our piping hot toasted sandwiches, crafted with a medley of delectable ingredients that will leave you craving more. And what would an enchanting evening be without the timeless delight of freshly popped popcorn, the perfect companion for indulging in tales and creating unforgettable memories.\nl

Ambrosia's ambiance is as captivating as the legends that inspire it. Immerse yourself in a tavern adorned with subtle nods to mythical tales, where every detail whispers of ancient stories and timeless enchantment. Be it the soft glow of candlelight, the comfortable seating arrangements, or the gentle melodies that serenade your senses, our café promises a haven of tranquillity and camaraderie.\nl

Step outside and discover our charming outdoor seating area, where you can bask in the moonlight and enjoy the company of fellow camp-goers. Engage in conversations that span realms, share laughter under the stars, and forge connections that will linger, long after the camp songs fade away.\nl

At Ambrosia, we believe in fostering a sense of community and embracing the spirit of adventure. Unplug from the digital world, indulge in the simple pleasures of life, and immerse yourself in the rich tapestry of legends that surround you. Join us during the camp's days which have “chill nights” from 9 AM to 4 PM and days which have central evening programme from 2 PM to 10 PM to experience the magic of Ambrosia firsthand.\nl

As an added touch of excitement, the café occasionally transforms when Venturers take over, creating their own culinary delights for you to discover. These surprises will captivate your senses and introduce you to a world of gastronomic wonders that unfold before your eyes.\nl

Visit Ambrosia at the most eastern marquee of the central area (the Camelot, Dinas Affaraon and Elysium side) and let us become the centrepiece of your legendary adventure. Unwind, replenish, and let your taste buds be seduced by the magic that awaits within our cosy haven.

\chapter{V-Coin}
Due to the lack of mobile signal and WiFi on site, and the aim to reduce cash use on-site, Venturer Camp 2023 will see the return of an on-camp currency, called the \textit{V-Coin}. This allows us to exchange currency in one location and spend on-site currency throughout the camp.\nl

V-Coin will be available to be exchanged from GBP during the event and exchanges back to GBP will be available at the end. The various denominations of V-Coin will be represented by different colour plastic chips. The camp Bureau de Change will be open for a small window each day, exact opening hours will be posted on the front of the Bureau de Change at Camp. \nl

The exchange rate between V-Coin and GBP will be 1:1. You will be able to exchange GBP for V-Coin using either cash or card, although we would prefer card (or Google Pay / Apple Pay) where possible as this reduces administrative burden and is more secure.\nl

At the end of camp, any unspent V-Coin over £5 can be changed back to GBP at no cost. All V-Coin tokens will need to be returned however as Woodcraft is renting these and will be charged for any which are not returned.\nl

V-Coin will be the only currency for use in the cafe and the shop to purchase refreshments and merchandise.


\backPagebw
\end{document}